\section{Cuentas por cobrar en moneda extranjera}
\begin{itemize}
    \item Se debe evaluar si cuendo al hacer transacciones en otras monedas para ver si perdí por la \textbf{fluctuación cambiaria}.
    \item Pasos:
    \begin{itemize}
        \item Reconozco la venta, \textbf{Clientes} contra \textbf{ventas}
        \item Cerrar libros: si el cliente va a pagar posteriormente, entonces voy al banco de GT y evalúo el tipo de cambio del día, entonces si el caso es ese que el precio fluctuó, tengo que cobrar más por el tipo de cambio, tengo que registrar una \textbf{ganancia por fluctuación} con contra partida de \textbf{clientes}.
        \item Paga la deuda: \textbf{Bancos} contra \textbf{clientes}, Si no establecí cuando brindé el crédito una taza por fluctuación debo registrar a la hora que me paguen una \textbf{pérdida por fluctuación}.
    \end{itemize}
    \emph{\textbf{Ojo: considerar lo siguiente...} La mayoría de contadores (creo que todos) presentan estados mensuales, para motivos fiscales se pagan mes a mes el impuesto }
\end{itemize}

\section{Saldos negativos de clientes}
\begin{itemize}
    \item Media vez el cliente nos pague anticipadamente, el cliente tiene un derecho y yo una obligación a brindarle el producto.
    \item Pasos:
    \begin{itemize}
        \item \textbf{Bancos} contra \textbf{anticipo de clientes} 
    \end{itemize}
    \emph{\textbf{Ojo: considerar lo siguiente...} muchas veces en GT un caso peculiar de los cambios de libras a Quetzales, usualmente se pasa de Libras $\Rightarrow $ Dólares $\Rightarrow $ Quetzales.}
\end{itemize}

\section{Inventario}
\begin{itemize}
    \item Costo de ventas:  
    \begin{itemize}
        \item PEPS: Primeras entradas, primeras salidas; 
        \item UEPS: Ultimo en entrar primero en salir; \emph{\textbf{Ojo: considerar lo siguiente...} El ISR dicta que no se puede utilizar UEPS por el hecho que aumenta el costo de ventas y a mayor costo de ventas menor impuesto.}
        \item Promedio ponderado: Costo promedio por existencia disponible durante todo el año. Este método las empresas que tienen artículos de mucho valor no utilizaría el método de promedio ponderado. 
    \end{itemize}
\end{itemize}

