\section{Activos fijos}
\begin{itemize}
    \item \emph{\textbf{Ejemplo:}Una computadora por ejemplo si me sirve a mí en mis operaciones, son bienes \textbf{tangibles}}
    \item \textbf{Nos preguntamos:} ¿Por qué hay depreciación? \emph{\textbf{La respuesta a esta pregunta es: }Se registra para motivos de llevar un historial de gastos que uno incurre al transcurrir el tiempo o usar los activos fijos}.
    \item \textbf{Nos preguntamos:} ¿Cómo deprecio los activos? \emph{\textbf{La respuesta a esta pregunta es: }deprecio con algún método de depreciación}. \emph{\textbf{Observación: }Siempre en los ejercicios se nos brindará la vida útil.}
    \item Línea recta:
        \begin{itemize}
            \item El valor en libros:
                \begin{center}
                   \[
                     \text{costo} - \text{depreciación acumulada} = \text{valor en libros}
                   \]
                \end{center}
            
            \item Depreciación anual: 
                \begin{center}
                \[
                    \text{depreciación anual} = \frac{\text{valor depreciable}}{\text{vida útil}}
                \]
                \end{center}
            
            \item \emph{\textbf{Observación: }Si no te dicen valor de rescate el valor depreciable es 0 y el monto depreciable se deprecia a 0 respecto a su monto neto.}
        \end{itemize}

    
    \item Unidades producidas o recorridas:
        \begin{itemize}
            \item Depreciación por unidad:
                \begin{center}
                    \[
                        \text{depreciación por unidad} = \frac{\text{valor depreciable}}{\text{vida útil}}
                    \]
                \end{center}
        \end{itemize}
            
    
    \item Números digitos crecientes:
        \begin{center}
            \[
               = \frac{\text{Número de año en el cual de está depreciando}}{\text{suma de los años ascendente} \equiv  \sum_{i=1}^{n=\text{número de año}}\text{año}}
            \]
        \end{center}
    
    
    \item Númerp de dígitos decrecientes: igual al anterior pero en lugar de empezar en 0 empiezo en el último año.
\end{itemize}

\section{Balance general}
\begin{itemize}
    \item Todos los métodos se utilizan los métodos, se usan gasto por depreciación (dimensional como mensual o anual) y depreciación acumulada.
\end{itemize}
