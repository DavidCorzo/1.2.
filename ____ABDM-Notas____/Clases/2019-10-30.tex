\section{Capital contable}
\begin{enumerate}
    \item El capital es la diferencia entre el activo y pasivo.
    \item El patrimonio de los accionistas.
    \item Dentro del capital contable: 
        \begin{itemize}
            \item 
        \end{itemize}
    
    \item Tipos de capital: 
        \begin{itemize}
            \item Autorizado: el máximo que se puede tener sin modificar la constitución.
            \item Suscrito: pagaron realmente 
            \item Exhibido: ya se tiene
            \item Donaciones: contribuciones en efectivo que realizan los accionistas.
            \item Acciones preferentes: acciones que valen más pero a la hora de votar no valen tanto. Tienen ciertos beneficios, ellos siguen acumulando acciones aun que esté perdiendo y a ellos se le pagan dividendos prioritariamente. Ellos son los que reciben \textbf{primero}. 
            \item Acciones Ordinarias: tienen más voto pero menos prioridad, a ellos se les paga posteriormente a los preferentes. En GT la mayoría de accionistas son ordinarios. 
            \item Dividendos: pagos a los accionistas como contribución.
            \item ROE:Return On Equity. \[
              \frac{\text{Utilidad neta}}{\text{Capital contable}}
            \]
        \end{itemize}
    
    \item Valor par o nominal: 
        \begin{itemize}
            \item \begin{center}
            \begin{tabular}{ | p{5cm} | p{5cm} | p{5cm} | }
             \hline
             Bancos & [monto] & \\
             Capital & & [monto] \\ 
             Prima en accciones & & monto \\ 
             \hline
            \end{tabular}
            \end{center}
        
        \item En GT se necesita dos personas para hacer una sociedad.
        \item Valor par: 
        \end{itemize}
\end{enumerate}
%%%%%%%%%%%%%%%%%%%%%%%%%%%%%%%%%%%%%%%%%%%%%%%%%%%%%%%%%%%%%%%%%%%%%%%%%%%%%%%%%%%%%%%%%%%%%%%%
\section{Próximamente: flujo, temas contables}
Preliminares del final y las presentaciones finales.
\begin{enumerate}
    \item 11 \& 13 son presentaciones.
    \item Hacer un resumen de los elementos de la empresa.
    \item Hacer gráficas, explicar los activos.
    \item Finales en la semana de finales.
\end{enumerate}
