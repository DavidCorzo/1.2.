\documentclass{article}
\author{David Gabriel Corzo Mcmath}
\title{Contabilidad Financiera - Capítulo \#1 - Resumen}
\date{2019-08-23 18:12}
% % % % % % % % % % % % % % % % % % % % % % % % % % % % % % % % % % % % % % % % % % % % % % % % % % %
\usepackage[margin = 1in]{geometry}
\usepackage{graphicx}
\usepackage{fontenc}
\usepackage[spanish]{babel}
\usepackage{amsmath}
\usepackage{amsthm}
\usepackage[utf8]{inputenc}
\usepackage{enumitem}
\usepackage{mathtools}
\usepackage{import}
\usepackage{xifthen}
\usepackage{pdfpages}
\usepackage{transparent}
\usepackage{color}
% % % % % % % % % % % % % % % % % % % % % % % % % % % % % % % % % % % % % % % % % % % % % % % % % %%

\begin{document}
\maketitle



\section{Introducción}
\begin{itemize}
    \item Vivimos rodiados de información, para usar correctamente se necesita entender la información financiera.
    \item Se pretende enseñar al lector a tomar decisiones a partir de información financiera. 
\end{itemize}

\section{Evolución de la información financiera}
\begin{itemize}
    \item En la antigüedad se usaban herramientas como:
    \begin{itemize}
        \item Tablillas de barro 
        \item Escritura cuniuniforme 
        \item Sistemas de pesas y medidas 
        \item Operaciones matemáticas
        \item Escritura pictográfica
        \item Jeroglífica
        \item Papiro y la moneda
    \end{itemize}
    
    \item En la edad media se dan avances en la coordinación del mercado, dado a los avances en la ccordinación de los individuos \textbf{se desarrolla la necesidad de un sistema de contabilidad funcional.}

    \item Al llegar el renacimiento e inventarse la imprenta se comenzaron a regstrar formalmente operaciones mercantiles, esta herramienta ayudó a perfeccionar la técnica contable.
    \item La revolución industrial dio validez a lo que en ese entonces era \textbf{la profesión contable}
    \item El reglamento que rodea la actividad de la contabilidad es algo contemporaneo. 

    
    \item Problemas que surgen: al globalizar la economía, la contabilidad se ve bajo una problemática ya que las personas necesitaban \textbf{rendición de cuentas} entonces la contabilidad tuvo que adaptarse a las nuevas necesidades.

    \item   Acontecimientos importantes:

        \begin{center}
        \begin{tabular}{ | p{5cm} | p{5cm} | p{5cm} | } 
         \hline
        1494 & 1930 & 2002 \\
        Fray Luca Pacioli, el registro dual. & EEUU, la gran recesión, necesidad de orden contable para estabilidad. & Fraude financiero en EEUU por no tener reglmentos de contabilidad. \\ 
         \hline
        \end{tabular}
        \end{center}

    
    \item Fray Luca Pacioli inventa la partida doble, básicamente los conseptos del \textbf{debe \& haber}

    
    \item EEUU ``The great depression'', a raíz de esta catástrofe financiera se necesitó \textbf{establecer de una vez por todas las normas de contabilidad para prevenir fraude a los inversionistas.}

    
    \item EEUU, 2002, se da origen al movimiento \textbf{gobierno corporativo (ley Sarbnes-Oxley)} que buscaba aumentar la calidad, transparencia y confianza de la información financiera de las empresas a raíz de lo que pasó por los fraudes financieros de muchas empresas. 
\end{itemize}

\subsection{Organizaciones económicas}
\begin{itemize}
    \item Están formadas por aquellos que ejercen la función empresarial y coordinan el mercado ellos despues derivando un beneficio de su coordinación.
\end{itemize}

\section{Objetivo de las organizaciones económicas}
\begin{itemize}
    \item Objetivo: \textbf{Servir al cliente, beneficiar al cliente, y al hacerlo beneficiarse a consecuencia del valor que crean.}
    \item Está en el interés de los administradores de las organizaciones económicas la satisfacción de sus clientes, \textbf{solo eso los hace rentables.}
    \item Misión y Visión de una empresa.
\end{itemize}

\subsection{Tipos de organizaciones económicas}
\begin{itemize}
    \small
    \item Organizaciones lucrativas
    \item Organizaciones no lucrativas
    \item Organizaciones gubernamentales
\end{itemize}

\subsubsection{Organizaciones lucrativas}
\begin{itemize}
    \small
    \item Son las más numerosas
    \item Ofrecen servicios, productos o mercancías para comercializar y así obtener \textbf{utilidad}
\end{itemize}

\subsubsection{Organizaciones no lucrativas}
\begin{itemize}
    \small
    \item \textbf{No persiguen lucro}
    \item Tienen usualmente fines sociales, \textbf{sí tienen utilidades} solo que no son para el uso personal del reclamante residual si no que son reinvertidas para la causa.
\end{itemize}

\subsubsection{Organizaciones gubernamentales}
\begin{itemize}
    \small
    \item Son financiados por impuestos que ingresan al gobierno.
    \item Los impuestos son muy cuantativos por ende pueden gastar mucho más.
    \item Requieren información financiera que \textbf{facilite la toma de decisiones}.
    \item Los criterios de la elaboración de información financiera \textbf{difiere sustancialmente de las organizaciones lucrativas y no lucrativas.}
\end{itemize}

\section{Tipos de organizaciones económicas lucrativas}
\begin{itemize}
    \item Los cuatro tipos de operaciones de las organizaciones lucrativas:
    \begin{itemize}
        \item Servicios
        \item Comercialización de bienes    
        \item Manufactura o transformación
        \item Giros especializados
    \end{itemize}
\end{itemize}


\begin{enumerate}
    \item \textbf{Empresas de servicios:}
    \begin{itemize}
        \item Prestan actividades \textbf{intangibles}
    \end{itemize}

    
    \item \textbf{Empresas de comercialización de bienes o mercancias:}
    \begin{itemize}
        \item Compra bienes o mercancías para su posterior venta (arbitrage).
        \item Se usa el concepto de inventarios y mercaderías que representa la mercancía que comercializa el negocio.
    \end{itemize}

    
    \item \textbf{Empresas manufactureras o de transformación:}
    \begin{itemize}
        \item Se dedica a la compra de materia prima para transformarla a productos derivados mediante la utilización de \textbf{mano de obra, tecnología, etcétera}.
        \item Implicaciones contables mayores  ya que tiene que llevar la contabilidad de productos en tres diferentes estados, \emph{en estado de materia prima, en proceso y final.}
        \item También determinan el costo de producción y el precio de venta.
    \end{itemize}

    
    \item \textbf{Empresas de giros especializados:}
    \begin{itemize}
        \item Estas son una mezcla de todas las especies de empresas mencionadas anteriormente.
        \item Entre estas están incluidas las siguientes:
        \begin{itemize}
            \item Empresas de servicios financieros \textbf{\emph{Definición:}} \emph{brindan servicios de inversión, financiamiento, ahorro, almacenamiento y resguardo de valores, tipo bancos, aseguradoras, empresas de factoraje financiero}
            \item Empresas de comercio electrónico \textbf{\emph{Definición:}} \emph{ ofrecen servicios a través de el internet tipo amazon.}
            \item Empresas del sector primario \emph{\textbf{Definición:} Promueven insumos primarios a los consumidores tipo ganadesias, agrícolas, pescaderías.}
            \item Empresas de la construcción: \emph{\textbf{Definición:} edifican obras civiles y/o explotan recursos que se encuentran en el subsuelo tipo minas, constructoras, metalúrgicas, petroleras}
        \end{itemize}
    \end{itemize}
\end{enumerate}


\section{Formas de ordenar las organizaciones económicas lucrativas}
\begin{enumerate}
    \item Forma de: ``La persona física''
    \begin{itemize}
        \item Una sola persona es suficiente para formar una entidad económica, esta busca realizar un Objetivo, esta sola persona es suficiente para llamarse una entidad económica, esta está \textbf{funcionalmente completa y puede operar de manera adecuada}.
        \item Fiscalmente se le denomina ``\emph{persona física con actividades empresariales}''
        \item Cualquier persona puede inciar un negocio propio, sin requerir de socios, solo debe llenar los trámites correspondientes. Es lo que en EEUU son los ``independant contractors''.
    \end{itemize}
    
    \item Forma de: ``Sociedad''
    \begin{itemize}
        \item Cuando hay varias personas como copropietarios o socios del negocio y todos obtienen utilidades de los beneficios empresariales que obtienen se llama sociedad.
        \item Existen dos tipos: 
        \begin{itemize}
            \item Sociedad de personas:
            \begin{itemize}
                \item Es una democracia entre los socios en los que cada voto para las decisiones cuenta por igual.
            \end{itemize}

            \item Sociedad de capitales:
            \begin{itemize}
                \item La validez del voto de cada socio se toma en cuenta en función de cuánto aporta. ``Mientras más aporten más podrán influir en la administración''.
            \end{itemize}
        \end{itemize}
    \end{itemize}
\end{enumerate}

\subsection{Sociedad anónima}
\emph{\textbf{Definición de ``Sociedad anónima":} es una entidad legal con personalidad jurídica propia, independiente a la de sus socios.}
\begin{itemize}
    \item Esta puede ser tratada ante la ley como una ``persona física'' por eso puede comprar bienes, operar negocios como que si fuera una persona física.
    \item Operar de esta manera deriva ciertas ventajas, entre estas están las siguientes:
    \begin{itemize}
        \item Como es una entidad autónoma, los dueños y accionistas están claramente diferenciados de esta, esta característica limita el derecho de los acreedores a los montos de los accionistas.
        \item Es más fácil la obtención de capital mediante la venta de acciones
        \item La vida de la sociedd anónima no es afectada por quiénes la operen o se adueñen de ella, por ende es facilmente transferible de una persona a otra, por ejemplo de herencia a los hijos. Solo se puede acabar la sociedad anónima si los actuales dueños deciden hacerlo.
    \end{itemize}
    
    \item Operar de esta manera conlleva las siguientes desventajas:
    \begin{itemize}
        \item Están altamente reguladas y son requeridas de presentar más papelería.
        \item Aquellas que ofrecen sus acciones al público accionista, debe presentar cuentas de sus operaciones de forma más amplia.
        \item Debe contar con una administración más compleja debido a la demanda de información financiera detallada que ellos deben producir.
    \end{itemize}
\end{itemize}

\section{Constitución de una sociedad anónima}
\begin{itemize}
    \item Se debe obtener autorización del gobierno mediante una solicitud emitida por una o más personas (socios fundadores).
    \item Una vez aprobada la solicitud se extiende un \textbf{permiso de constitución} que proporciona el nombre oficial de la sociedad anónima y a qué se dedica, también cantidad de acciones autorizadas.
    \item A los participantes de la sociedad anónima se le conoce como \textbf{accionistas}.
\end{itemize}

\subsection{Requisitos mínimos para formar una sociedad anónima}
\begin{itemize}
    \item Tener dos socios como mínimo.
    \item Una aportación de capital en efectivo.
\end{itemize}


\subsection{Tras haber expendido el permiso de constitución se hace lo siguiente}
\begin{itemize}
    \item Aprueban reglas bajo las cuales la sociedad se regirá.
    \item Elegir a los administradores
    \item Decidir la emisión y venta de acciones.
    \item Fijar procedimientos operativos más importantes para la compañía. 
\end{itemize}


\subsection{Con lo que queda registrado para propósitos legales y para dar formalidad a la sociedad se emite un documento llamado ``Acta'' o ``Escritura constitutiva''}
Elementos relevantes del acta o escritura constitutiva:
\begin{itemize}
    \item Nombre, nacionalidad, domicilio de las personas físicas
    \item Objetivo de la sociedad
    \item Razón social o denominación
    \item Duración
    \item Importe de capital social
    \item Domicilio de la sociedad
    \item Manera de distribuir utilidades y pérdidas entre los miembros que integran la sociedad.
\end{itemize}


\section{Administración de la sociedad anónima}
\begin{itemize}
    \item La asamblea de accionistas es la más alta en la jerarquía de la sociedad anónima, en ella se toman importantes decisiones acerca de la administración del negocio, ratificación de actos y operaciones, lo acordado en la misma debe de ser cumplido por la sociedad anónima.
\end{itemize}

\begin{itemize}
    \item La asamblea se reúne por lo menos una vez por año.
    \item La asamblea de accionistas designa quiénes formarán parte de el consejo de administración, pueden ser accionistas o personas ajenas a la sociedad, pero estos son responsables por la información financiera que se produce y se presenta y por cumplir las decisiones acordadas por la asamblea de accionistas.
    \item Genralmente el consejo de administración se estructura así:
    \begin{itemize}
        \item Presidente de consejo: este es el principal funcionario, solo la asamblea de accionistas tiene más poder sobre él.
        \item Secretario: este elabora las actas donde constan lo acuerdos tomados en reuniones de consejo o asambleas de accionistas, al igual que lleva control de las actas que se han ratificado o emitido.
        \item Consejeros: ellos aconsejan con una opinión crítica acerca de las decisiones que toma la asamblea y el consejo, pueden ser personas ajenas o miembros de la sociedad anónima. 
    \end{itemize}
\end{itemize}

\section{Vigilancia de la sociedad anónima}
\begin{itemize}
    \item Está a cargo de uno más comisarios temporales que pueden ser socios o personas ajenas a la sociedad.
    \item Sus obligaciones son las siguientes:
    \begin{itemize}
        \item Exigir a los administradores información financiera mensual.
        \item Realizar una verificación de las operaciones y registros al igual que otras herramientas comprobatorias.
        \item Presentar anualmente un informe con la información financiera, con su propia opinión acerca de las políticas y criterios contables, cómo estos se han aplicado y qué tan eficaz son, al igual que presentar evidencias que la información financiera es suficiente y veráz para tener la confianza que la información es la correcta y sí en efecto refleja la situación financiera de la sociedad anónima.
    \end{itemize}
    
    \item El que ratifica las actividades de un comisarios es un \textbf{auditor}, este examina toda la información de administración y verifica que la información si refleje el estado financiero de la organización.
\end{itemize}

\section{Administración de las sociedades desde la perspectiva del gobierno corporativo}
\begin{itemize}
    \item Las leyes que componen el gobierno corporativo son precisamente para la mejor administración de las sociedades mercantiles públicas (esta es una empresa que ha emitido acciones de capital o instrumentos de deuda a través del mercado de valores).
    \item Las normas del gobierno corporativo regula la forma que administran y controlan las sociedades.
    \item El gobierno corporativo busca:
    \begin{itemize}
        \item Administrar honstamente
        \item Proteger los derechos de los accionistas
        \item Definir la responsabilidad del consejo
        \item Definir la responsabilidad de la administración
        \item Dar fluidez a la información  
        \item Regular las relaciones de los gruos de interes
    \end{itemize}

    
    \item Los principios del gobierno corporativo:
    \begin{enumerate}
        \item Responsabilidad: porque promueve la identificación clara de los accionistas de una sociedad, también la forma de el consejo y sus responsabilidad.
        \item Independencia: porque los auditores y los ejecutivos son independientes e imparciales para asegurar la información producida. 
        \item Transparencia: porque conlleva la reproducción de reportes altamente detallados que dan a conocer realmente cuál es la situación de una empresa.
        \item Igualdad: porque promueve la conformidad de derechos de los accionistas en relación con los asuntos de la sociedad.
    \end{enumerate}

    
    \item Funciones del consejo de administración según el gobierno corporativo:
    \begin{itemize}
        \item Estableces la visión estratégica de la sociedad
        \item Garantizar a los accionistas y al mercado información.
        \item Establecer mecanismo de control interno.
        \item Verificar que la sociedad cumpla conlas disposiciones legales aplicables.
        \item Evaluar regularmente el desempeño del director general y de funcionarios de alto nivel.
    \end{itemize}

    
    \item Las pracitas del gobierno corporativo sugieren que se tengan entre 5 a 15 consejeros propietarios en el consejo de administración, y almenos 20\% de ellos independientes.
    \item Sugieren asímismo que se maneje el consejo de administración con tres comités:
    \begin{enumerate}
        \item \textbf{Comité de evaluación y compensación:} su función comprende sugerir al consejo procedimientos para proponer al director general y a funcionarios de alto nivel, además criterios de evaluación y remuneración de los mismos.  
        \item \textbf{Comité de auditoría:} comprende la función de recomendar candidatos, condiciones y desempeño de los auditores externos, asímismo como coordinar las labores de los auditores externos e internos también la de los comisarios.
        \item \textbf{Comité de finanzas y planeación:} su función comprende sugerir y evaluar políticas de inversión y financiamiento de la sociedad, así como la planeación estratégica, los presupuestos y la identificación de los factores de riesgo.
    \end{enumerate}
\end{itemize}

\section{La información financiera: idioma de los negocios}
\begin{itemize}
    \item El objetivo de la contabilidad es generar información útil para los que interese, para reflejar la situación financiera de una organización.
    \item Que esté representado de una manera correcta y represente la situación financiera correspondiente a la realidad se presentarán datos de suma importancia para interesados como accionistas y acreedores.
\end{itemize}

\section{La información financiera como herramienta de competitividad}
\begin{itemize}
    \item La interpretación de información financiera proporciona una muy importante herramienta en el ámbito de la competencia ya que no se puede competir sin tener un sistema contable eficiente
    \item Se estas interpretaciones se derivan las decisiones que radican en las siguientes tres especies de decisiones en un negocio:
    \begin{itemize}
        \item Decisiones de operación: se refieren a interpretar datos financieros relacionados con las operaciones, cosas como ver el monto de ventas que se genrearon, margen de utilidad, todas estas cosas interrogan información para tomar decisiones operativas del negocio.
        \item Decisiones de financiamiento: un negocio requiere de financiamiento para comenzar a operar y continuar de acuerdo con sus planes, busca la interpretación de información financiera para tomar decisiones de financiamiento.
        \item Decisiones de inversión: evalúa el beneficio derivado de un potencial inversión basado en información financiera proporcionada, la toma de decisiones a partir de esta información implica la adquisición de nuevos bienes.
    \end{itemize}
    
    \item Esta información brinda a las empresas mejor estrategia por ende es de carácter indispensable que se tenga buena información financiera ya que es una herramienta muy útil en la competencia con otras empresas.
\end{itemize}


\section{Tipos de usuarios}
La información financiera proporcionada es útil para dos tipos de usuarios, el usuario externo y el usuario interno.

\section{Usuarios externos}
\begin{itemize}
    \item \textbf{Inversionistas presentes:} accionistas que aportan a la empresa.
    \item \textbf{Inversionistas potenciales:} inversionistas nuevos que potencialmente puedan invertir.
    \item \textbf{Poveedores y otros acreedores comerciales:} para evaluar la capacidad de pago de los compromisos financieros contraídos.
    \item \textbf{Clientes:} para evaluar dependencias comerciales.
    \item \textbf{Empleados:} para evaluar la capacidad de pago de las remuneraciones pactadas organización a la que prestan servicios.
    \item \textbf{Órganos internos o externos de revisión:} para los auditores internos y externos.
    \item \textbf{Gobiernos:} principalmente para ver cuántos impuestos deben pagar.
    \item \textbf{Organismos públicos de supervisión financiera:} necesitan la información financiera por motivos de regulación.
    \item \textbf{Analistas e intermediarios financieros:} para monitorear el desempeño, muchas veces para evaluar la capacidad de pago, para conocer la capacidad de pagar un préstamo.
    \item \textbf{Usuarios de gobierno corporativo:} los miembros de consejo de administración que son sugerencia del gobierno corporativo.
    \item \textbf{Público en general:} para propósitos estadísticos, académicos y de cultura financiera.
\end{itemize}

\section{Usuarios internos}
\begin{itemize}
    \item Se requiere para los miembros de la organización información financiera, en este caso más detallada que la que se le entrega a los usuarios externos, para directores generales, miembros de los altos mandos de la organización, estos reciben información más detallada.
\end{itemize}


\section{Tipos de contabilidad}
\begin{itemize}
    \item \textbf{Contabilidad financiera:} comprende elementos como normas de registro, formas de presentación, etcétera; para examinar las transacciones al igual que acontecimientos económicos de una entidad, esta contabilidad es más para usuarios externos. \emph{Regida por reglas aplicables a la elaboración de la información.} \textbf{Los ingresos son vistos como qué tanto le ingresa a una empresa por lo que haya vendido o prestado servicios en.}
    \item \textbf{Contabilidad fiscal:} es útil para las autoridades gubernamentales para determinar a base de la información proporcionada cuánto se debe pagar en impuestos. \emph{Regida  por leyes fiscales de cada región.} \textbf{Los ingresos son vistos con efectos de impuestos según la ley}   
    \item \textbf{Contabilidad administrativa:} es útil para los usuarios internos de la organización ya que es información que va orientada para la toma de decisiones internas.
\end{itemize}

\textbf{Los impuestos se deducen de la utilidad contable, que se saca a partir de esta fórmula:}
\begin{center}
\begin{tabular}{ p{1cm}p{4cm} } 
&  Ingresos \\
-& Gastos \\  
 \hline
 =  & Utilidad contable 
\end{tabular}
\end{center}

\textbf{Entonces el monto sobre el cual se cobran los impuestos se saca a partir de esta fórmula:}
\begin{center}
\begin{tabular}{ p{1cm}p{5cm} } 
 &Ingresos acumulados  \\
-  & Deducciones autorizadas\\ 
 \hline
=  & Base gravable \\ 
- & Impuesto sobre la renta \\ 
\end{tabular}
\end{center}
No todas las partidas contables se consideran para el cálculo de la base gravable fiscal.

\section{Diferencias entre contabilidad financiera y contabilidad administrativa}
\begin{itemize}
    \item Ambas generan información útil, lo que difiere es para quién va dirigida.
    \item La contabilidad administrativa está para generar información útil para los integrantes internos de la organización; va enfocada hacia el futuro, pero también produce información para usuarios externos que es la información del pasado sucesos ya realizados por la organización.
    \item Contabilidad financiera, está regulada por las normas NIIF.
    \item La contabilidad administrativa no está regulada con nada, es más para las personas internas de la organización. Además la contabilidad administrativa a diferencia de la contabilidad financiera sí interactúa con otras disciplinas tales como las de estadística y economía para la más precisa producción de información.
\end{itemize}

\section{La profesión contable}
\begin{itemize}
    \item Empezó como oficio y escaló a ser cada vez más demandado hasta llegar a considerarse una profesión.
    \item El ejercicio profesional de la contabilidad está divido en dos, el dependiente y el independiente.
\end{itemize}

\section{Ejercicio profesional independiente:}
\begin{itemize}
    \item Ellos no trabajan para algúna organización en específico, son independientes y restan sus servicios al público en general.
    \item Principalmente comprende auditores que se dedican a verificar la información financiera para fines de impuestos, análisis y verificar si se han aplicado correctamente las políticas contables.
\end{itemize}

\section{Ejercicio profesional dependiente}
\begin{itemize}
    \item Ellos trabajan para una organización en específico, y velan por producir información financiera útil para esa organización.
    \item Proporciona a los funcionarios información para la toma de decisiones.
\end{itemize}

\section{Organización de la profesión contable}
\begin{itemize}
    \item Asociaciones profesionales de contadores públicos:
    \begin{itemize}
        \item International Federation of Accountants (IFAC): con la finalidad de mejorar la calidad de la profesión contable.
        \item Asociación Interamericana de Contabilidad (AIC): este organismo agrupa a 1,400,000 de profesionales que pertenecen a 33 organismos nacionales de contadores públicos.
        \item Instituto Mexicano de Contadores Públicos (IMCP): agrupa contadores en México y actualmente está encargado de certificar contadores públicos Mexicanos.
        \item American Institute of Certified Public Accountants (AICPA): encargado de acreditar a los profesionales en EEUU en el área de contabilidad.
    \end{itemize}
\end{itemize}

\section{Organismos responsables de la emisión de normas de información financiera}
\begin{itemize}
    \item Esta comprende cómo deben de presentar los contadores las normas para cumplir con las características necesarias:
    \begin{itemize}
        \item International Accounting Standards Board (IASB): fundada en 1973, busca mejorar los informes financieros a través de las NIIF. Las normas emitidas por el IASB son requeridas en más de 100 países.
        \item Consejo Mexicano para la Investigación y Desarrollo de Normas de Información Financiera (CINIF): fundado en 2001, promueve la investigación, desarrollo y la difusión de las normas de información financiera.
        \item Finantial Accounting Standards Board (FASB):responsable de elaborar y emitir normas de contabilidad financiera en EEUU.
    \end{itemize}
\end{itemize}

\section{Organismos de supervisión financiera}
\begin{itemize}
    \item Estas instituciones se encargan de verificar que la información esté estructurada y redactada de acuerdo a las normas de contabilidad vigentes:
    \begin{itemize}
        \item International Organization of Securities Commissions (IOSC): busca la efectiva vigilancia de las transacciones mundiales de valores.
        \item Comisión Nacional Bancaria y de Valores (CNBV): busca regular el mercado de valores emitiendo normas contables diseñadas para proteger a los inversionistas y al gran público que participa en dicho mercado.
        \item Securities and Exchange Commission (SEC): su objetivo es el mismo que el de el CNBV solo que en EEUU. 
    \end{itemize}
\end{itemize}

\rule{16cm}{1pt}

\end{document}
