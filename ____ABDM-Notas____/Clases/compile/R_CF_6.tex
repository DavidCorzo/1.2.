\documentclass{article}
\author{David Gabriel Corzo Mcmath}
\title{Contabilidad Financiera - Capítulo 6}
\date{2019-08-30 20:57}
% % % % % % % % % % % % % % % % % % % % % % % % % % % % % % % % % % % % % % % % % % % % % % % % % % %
\usepackage[margin = 1in]{geometry}
\usepackage{graphicx}
\usepackage{fontenc}
\usepackage[spanish]{babel}
\usepackage{amsmath}
\usepackage{amsthm}
\usepackage[utf8]{inputenc}
\usepackage{enumitem}
\usepackage{mathtools}
\usepackage{import}
\usepackage{xifthen}
\usepackage{pdfpages}
\usepackage{transparent}
\usepackage{color}
% % % % % % % % % % % % % % % % % % % % % % % % % % % % % % % % % % % % % % % % % % % % % % % % % %%

\begin{document}
\maketitle
%%%%%%%%%%%%%%%%%%%%%%%%%%%%%%%%%%%%%%%%%%%%%%%%%%%%%%%%%%%%%%%%%%%%%%%%%%%%%%%%%%%%%%%%%%%%%%%%

\section{Principales diferencias entre una empresa de servicios y una comercializadora de mercancías}
\begin{itemize}
    \item Las dos organizaciones económicas lucrativas sobresalientes son las:
    \begin{itemize}
        \item Prestadoras de servicios
        \item Las comercializadoras de mercancías 
    \end{itemize}
    Dado a este hecho la contabilidad para cada una \textbf{es diferente}
\end{itemize}


%%%%%%%%%%%%%%%%%%%%%%%%%%%%%%%%%%%%%%%%%%%%%%%%%%%%%%%%%%%%%%%%%%%%%%%%%%%%%%%%%%%%%%%%%%%%%%%%


\subsection{Diferencias en el estado de resultados (empresas prestadoras de servicios vs. empresas comercializadoras de mercancías)}

\begin{enumerate}

    \item Diferencias de \textbf{ingresos}:
    \begin{center}
    \begin{tabular}{ | p{5cm} | p{9cm} | }
     \hline
     Empresa de servicios & Empresa comercial\\
     \hline
     Se utiliza la cuenta ingresos para registrar ingresos & Se utiliza la cuenta de \textbf{ventas}, al sustraerle las cuentas de \textbf{descuentos} y la cuenta de \textbf{devoluciones o bonificaciones} se obtiene la venta neta. \\ 
     \hline
    \end{tabular}
    \end{center}


    \item Costos y gastos orientado a costo de ventas:
    \begin{center}
    \begin{tabular}{ | p{5cm} | p{9cm} | }
     \hline
     Empresa de servicios & Empresa comercial\\
     \hline
    No hay ya que no trabajan con nada más que con nada tangible & Se registra en la cuenta de \textbf{costo de ventas.}\\ 

     \hline
    \end{tabular}
    \end{center}
    
    
    
    \item Costos y gastos orientado a compras:
    \begin{center}
    \begin{tabular}{ | p{5cm} | p{9cm} | }
    \hline
    Empresa de servicios & Empresa comercial\\
    \hline
    No hay nada que comprar por que no se dedica a nada tangible  & Se registra en cuentas como \textbf{Compras, Fletes sobre compras, descuentos sobre compras, devoluciones y bonificaciones sobre compras} la suma de estas cuentas es la \textbf{compra neta} \\ 
    \hline
    \end{tabular}
    \end{center}                                                                                
    
    
    
    \item Costos y gastos orientado a gastos:
    \begin{center}
    \begin{tabular}{ | p{5cm} | p{9cm} | }
    \hline
    Empresa de servicios & Empresa comercial\\
    \hline
    Las cuentas de gastos se presentan trás haber registrado los ingresos, después se calcula la utilidad & Los gastos se registran en las cuentas de \textbf{gastos de venta, gastos de administración \& gastos financieros}, \newline  utilidad bruta = (Ventas netas - costo de ventas) - gastos \\ 

    \hline
    \end{tabular}
    \end{center}
    
    
    \item Inventario:
    \begin{center}
    \begin{tabular}{ | p{5cm} | p{9cm} | }
    \hline
    Empresa de servicios & Empresa comercial\\
    \hline
    No existe inventario para esta empresa & Se registra la mercancía disponible en la cuenta de \textbf{inventario.}\\ 

    \hline
    \end{tabular}
    \end{center}

\end{enumerate}
%%%%%%%%%%%%%%%%%%%%%%%%%%%%%%%%%%%%%%%%%%%%%%%%%%%%%%%%%%%%%%%%%%%%%%%%%%%%%%%%%%%%%%%%%%%%%%%%

\section{Registro de transacciones en empresas comercializadoras}
\begin{itemize}
    \item Metodología de estudio, analizar, registrar, clasificar los movimientos en el \textbf{mayor general}, enfoque en las siguientes cuentas:
    \begin{itemize}
        \item Intentario de mercancías 
        \item Compras y cuentas relacionadas
        \item Ventas y cuentas relacionadas 
    \end{itemize}
\end{itemize}
%%%%%%%%%%%%%%%%%%%%%%%%%%%%%%%%%%%%%%%%%%%%%%%%%%%%%%%%%%%%%%%%%%%%%%%%%%%%%%%%%%%%%%%%%%%%%%%%
\section{Inventario de mercancías}
\begin{itemize}
    \item En la cuenta \textbf{Inventario de mercancías} se registran las sobras del inventario que no se vendió al final de un periodo.
    \item El inventario \textbf{final} del periodo = El inventario \textbf{inicial} del siguiente periodo.
    \item La cuenta \textbf{Inventario de mercancías} es un \textbf{activo circulante}.
\end{itemize}
%%%%%%%%%%%%%%%%%%%%%%%%%%%%%%%%%%%%%%%%%%%%%%%%%%%%%%%%%%%%%%%%%%%%%%%%%%%%%%%%%%%%%%%%%%%%%%%%

\section{Sistema de registro de inventario}
Hay dos sistemas de contabilidad válidos para registrar inventario en la contabilidad.
\begin{itemize}
    \item Sistema perpetuo: los datos están actualizados, se puede saber qué tantas mercancías hay en existencia en cualquier momento. En este sistema solo se necesita la cuenta \textbf{inventario de mercancías} y cuando se compra inventario esta aumenta y cuando se vende disminuye. \emph{\textbf{(Paréntesis: No se utilizan cuentas como fletes,devoluciones,compras,bonificaciones.}\textbf{)}}
    \item Sistema periódico: No mantiene datos actualizados de las mercancías en existencia, se actualiza la información al final del periodo y las transacciones se registran en las cuentas \textbf{compras, fletes, devoluciones o bonificaciones.}
\end{itemize}
%%%%%%%%%%%%%%%%%%%%%%%%%%%%%%%%%%%%%%%%%%%%%%%%%%%%%%%%%%%%%%%%%%%%%%%%%%%%%%%%%%%%%%%%%%%%%%%%

\section{Costo de mercancía vendida y utilidad bruta}
Para el método periódico se utiliza la siguiente fórmula para determinar el costo de la mercancía vendida.
\begin{itemize}
\item Para determinar el costo de las mercancías disponibles: \newline 
\begin{tabular}{ | c |}
\hline
+ Compras brutas \\
+ Fletes sobre compras \\ 
- Descuentos sobre compras \\ 
- Devoluciones y bonificaciones sobre compras  \\  
\hline
= Costo de las mercancías disponibles\\ 
\hline
\end{tabular}

\item Para determinar el costo de ventas: \newline 
\begin{tabular}{ | c |}
\hline
+ Costo de mercancías disponibles\\
- Inventario Final \\
\hline  
= Costo de ventas \\ 
\hline
\end{tabular}

\item Para determinar utilidad bruta y utilidad neta: \newline 
\begin{tabular}{ | c |}
\hline
 + Costo de mercancías vendida\\
 - Ventas netas \\ 
\hline 
= Utilidad bruta \\  
- Otros gastos del periodo \\ 
\hline
= Utilidad neta \\ 
\hline
\end{tabular}

\end{itemize}
%%%%%%%%%%%%%%%%%%%%%%%%%%%%%%%%%%%%%%%%%%%%%%%%%%%%%%%%%%%%%%%%%%%%%%%%%%%%%%%%%%%%%%%%%%%%%%%%
\section{Compras y cuentas afines}
\begin{itemize}
    \item Compras:
    \begin{itemize}
        \item En el periódico el costo de mercancía adquirida se registra en la cuenta de \textbf{compras}, \emph{\textbf{(Paréntesis: La cuenta de compras es solo para mercancía que se va a vender, si se compra un activo para la operación de negocio se registra en otra cuenta del activo.}\textbf{)}}
        \item En el perpetuo se registra el costo de mercancía adquirida en la cuenta de \textbf{inventario}.
        \item Al ser vendida la mercancía se clasifica en la cuenta de costo de ventas y al cerrar el periodo se compara con la contrapartida \textbf{la cuenta de pérdidas y ganancias}.
        \item Tras haber documentado las transacciones en el \textbf{diario general} se registra en el  \textbf{mayor general}.
        \item Ejemplo pg. 5
    \end{itemize}
        
    \item Devoluciones sobre compras:
    \begin{itemize}
        \item Si por alguna razón la mercancía no viene en condiciones aptas para venderse se contacta el proveedor y en el sistema perpetuo se carga a proveedores, se abona a devoluciones y bonificaciones sobre compras; en el sistema perpetuo solo se registra en la cuenta de inventarios.
        \item La cuenta de devoluciones y bonificaciones sobre compra es una cuenta diseñada para la deducción de costo de mercancía en caso venga defectuosa y cierra al finalizar el periodo con su contra cuenta de \textbf{pérdidas y ganancias}.
        \item Ejemplo pg. 6
    \end{itemize}


    
    \item Descuentos sobre compras:
    \begin{itemize}
        \item Para incentivar a que los clientes paguen pronto sus deudas se hace un descuento de la deuda si se cancela en un intervalo de tiempo después de la factura.
        \item Estos descuentos se registran en la cuenta \textbf{descuentos sobre compras} en el sistema periódico.
        \item Ejemplo pg. 7
        \item Condiciones de crédito:
        \[
          \underbrace{2}_{\text{Descuento}} / \underbrace{\text{10}}_{\text{Días}}, \underbrace{\text{n}}_{\text{Neto}} / \underbrace{30}_{\text{11-30 días}}
        \]
    \end{itemize}
\end{itemize}

%%%%%%%%%%%%%%%%%%%%%%%%%%%%%%%%%%%%%%%%%%%%%%%%%%%%%%%%%%%%%%%%%%%%%%%%%%%%%%%%%%%%%%%%%%%%%%%%

\section{Gastos adicionales que forman parte del producto}
Hay gastos no claramente incurridos en la venta del producto que deben de ser contemplados en el precio al que se venderá la mercancía.
\begin{itemize}
    \item Fletes: aparecen en factura como ``condiciones de embarque'' o  ``condiciones de envió'', se registra esta transacción usando el método periódico en la cuenta \textbf{fletes sobre compras}, en el sistema perpetuo en la cuenta de \textbf{inventarios}:
    \begin{itemize}
        \item Libre A Bordo (LAB) punto de embarque: el comprador debe pagar los costos de envío (el comprador registra en las cuentas correspondientes).
        \item Libre A Bordo (LAB) punto de destino: el vendedor debe pagar todos los costos de envió (el comprador no debe realizar ningún registro ya que todo lo absorbe el vendedor).
        \item Ejemplo pg. 8
    \end{itemize} 
    Los pagos del embarque se registran en los dos sistemas en la cuenta \textbf{fletes sobre ventas}, esto es un gasto y se registra posteriormente en los gastos generales en el estado de resultados.


    
    \item Seguros:
    \begin{itemize}
        \item Se contrata un seguro que cuesta dinero por ende es un gasto que debe de estar contemplado en el precio final del producto.
        \item Se contrata un seguro para reducir incertidumbre,
    \end{itemize}

    
    \item Impuestos de importación.
    \begin{itemize}
        \item Se hace otro gasto adicional el de impuestos y aranceles, este gasto también debe de estar contemplado en el precio final del producto.
    \end{itemize}
\end{itemize}

%%%%%%%%%%%%%%%%%%%%%%%%%%%%%%%%%%%%%%%%%%%%%%%%%%%%%%%%%%%%%%%%%%%%%%%%%%%%%%%%%%%%%%%%%%%%%%%%
\section{Ventas y cuentas afines}
\begin{itemize}
    \item Ventas:
    \begin{itemize}
        \item Registra ventas en los dos sistemas se hacen de la siguiente forma: 
        \begin{itemize}
            \item Periódico: cuando se vende mercancía se hace un cargo a bancos o clientes, si es a crédito se abona a \textbf{ventas}
            \item Perpetuo: de la misma manera que el periódico, el perpetuo hace un registro adicional ese es el cargo a costo de ventas y abono a la cuenta de inventario. 
        \end{itemize}
        Ejemplo pg.203, \emph{\textbf{(Paréntesis} Si la empresa vende un activo que no estaba destinado para la venta se debe registrar como un cargo a bancos o cuentas por cobrar y se abona a la cuenta de activo que se vende.\textbf{)}}
    \end{itemize}
    
    \item Devoluciones sobre ventas: En las dos se registra la devolución a la cuenta de \textbf{devoluciones y bonificaciones sobre ventas}, y se disminuye posteriormente la cuenta de clientes.
    \begin{itemize}
        \item Periódico: no se registran ingresos de las mercancías por las devoluciones. Solo se registra la en la cuenta especial \textbf{devoluciones y bonificaciones sobre ventas} y se hace un cargo a \textbf{clientes}.
        \item Perpetuo: sí se debe registrar el ingreso de las mercancías devueltas en la cuenta \textbf{inventarios} y su contrapartida \textbf{costo de ventas}; después se registra la cuenta especial de \textbf{devoluciones y bonificaciones sobre ventas} para conocer en cualquier momento la cantidad de productos devueltos, y se hace un cargo a clientes. 
        \item Ejemplo pág. 204 \emph{\textbf{(Paréntesis:}Las devoluciones son costosas y representan un gasto extraordinario, de ocurrir muy amenudo se debe tomar medidas para reducir las devoluciones.\textbf{)}} 
    \end{itemize}
        
        \item Descuentos sobre ventas: se hace el mismo procedimiento que el de descuentos sobre compras solo que esta vez en la cuenta \textbf{descuentos sobre ventas}, esta es una deducción de la cuenta ``ventas''. Esta cuenta nace a partir de haber incentivado descuentos por pronto pago desde el punto de vista del que vende.

        
        \item Descuentos comerciales: Los descuentos comerciales se hacen para incentivar al comprador a comprar más, generalmente a la compra al por mayor se le hace un descuento por este motivo, y estos \textbf{no se registran en ninguna cuenta especial} se hace el registro a \textbf{ventas} por el precio de venta como tal (precio - descuento) y esto se hace en \textbf{los dos sistemas de inventario}.
\end{itemize}

%%%%%%%%%%%%%%%%%%%%%%%%%%%%%%%%%%%%%%%%%%%%%%%%%%%%%%%%%%%%%%%%%%%%%%%%%%%%%%%%%%%%%%%%%%%%%%%%
\section{Clasificación de cuentas de ingresos y gastos}
Principales clasificaciones de ingresos y gastos en el estado de resultados:
\begin{enumerate}
    \item Ingresos ordinarios: incluyen \textbf{todos los ingresos derivados a partir de la venta de un servicio o la prestación de un servicio.}
    \item Costo de ventas: costo de producir todos los artículos vendidos.
    \item Gastos generales: se dividen en dos,
    \begin{itemize}
        \item Gastos de venta: todo gasto que se relacione con la función de la venta, (salarios del personal, comisiones sobre ventas, publicidad, alquileres, etcétera).
        \item Gastos de administración: Los gastos que se relacionen con la administración (ya sea salario de oficinistas, alquiler, servicios públicos, etcétera).
    \end{itemize}
    \item Otros ingresos y gastos: ingresos o gastos que no se relacionen con las principales actividades del negocio. (Ejemplo intereses, ganancias o pérdidas por venta de activos fijos.) 
    \item Resultado integral de financiamiento : representan el total de ingresos o  gastos por interéses (ganancias o pérdidas de los mismos asi mismo como la fluctuación cambiaria).
    \item Participación en los resultados de subsidiarias no consolidadas y asociadas: subsidios  
    \item Partidas no ordinarias: Son aquellos gastos que no son incurridos frecuentemente como pérdidas por catástrofes naturales por ejemplo.
\end{enumerate}

\begin{center}
\begin{tabular}{ | p{0.5cm} | p{9cm} | }
 \hline
 \multicolumn{2}{|c|}{Estado de resultados}\\
 \hline
 &  Ventas o ingresos netos \\ 
 \hline
 - & Costo de ventas \\ 
 \hline
 = & Utilidad o pérdidas bruta \\ 
 \hline
 - & Gastos generales (de venta y de administración) \\ 
 \hline
 $\pm$ & Otros ingresos y gastos\\
 \hline
 $\pm$ & Resultado integral de financiamiento \\ 
 \hline
 $\pm $ & Partidas no ordinarias \\ 
 \hline
 = & Utilidad o pérdida antes de impuestos a la utilidad \\ 
 \hline
 - & Impuestos a la utilidad \\ 
 \hline
= & Utilidad antes de operaciones discontinuas \\ 
\hline
- & Operaciones discontinuas \\ 
\hline
= & Utilidad o pérdida neta \\ 
\hline 
\end{tabular}
\end{center}


%%%%%%%%%%%%%%%%%%%%%%%%%%%%%%%%%%%%%%%%%%%%%%%%%%%%%%%%%%%%%%%%%%%%%%%%%%%%%%%%%%%%%%%%%%%%%%%%
\section{Procedimiento a cierre contable}
\begin{itemize}
    \item El ciclo contable se realiza en cuatro procesos básicos, y estos se siguen independientemente de qué tipo de empresa sea (servicio o comercializadora), a excepción de el último, el \textbf{cierre} ahí si varían \emph{\textbf{(Paréntesis:}difieren en el cierre por que las empresas comercializadoras tienen dos sistemas de llevar inventario el perpetuo y el periódico.\textbf{)}}: 
    \begin{enumerate}
        \item Registro de transacciones 
        \item Ajustes 
        \item Estados financieros 
        \item Cierre 
    \end{enumerate}
    
    \item A partir de la elaboración de los estados financieros se emite una \textbf{balanza de comprobación ajustada.}
\end{itemize}

\begin{enumerate}
    \item Cierre de las cuentas con saldo acreedor: Se deja en \textbf{cero} aquellas cuentas de resultado con \textbf{saldo acreedor}.
    \item Cierre de las cuentas con saldo deudor: Se debe de dejar en \textbf{cero} las cuentas de resultado con \textbf{saldo deudps}.
    \item Baja del inventario inicial: Su objetivo es dar de baja al inventario inicial.
    \begin{itemize}
        \item Perpetuo: Solo se realiza si existen diferencias entre el inventario registrado y el inventario real.
        \item Periódico: Se agarra la cuenta \textbf{pérdidas y ganancias} y su contra partida el \textbf{inventario}.
    \end{itemize}
    
    \item Alta del inventario final: 
    \begin{itemize}
        \item Perpetuo: Solo se realiza si existen diferencias entre el inventario registrado y el inventario real.
        \item Periódico: Se agarra la cuenta \textbf{inventario} y su contra partida el \textbf{pérdidas y ganancias}. (al revés)
    \end{itemize}

    
    \item Cierre de la cuenta de dividendos: se realiza el cierre de la misma manera en los dos sistemas de inventario, [cuentas son \textbf{utilidades retenidas y su contrapartida la de dividendos}].

    
    \item Cierre de la cuenta de pérdidas y ganancias: se realiza el cierre de la misma manera en los dos sistemas, [cuentas con \textbf{pérdidas y ganancias  y su contra partida utilidades retenidas}].
\end{enumerate} 

Tras haber hecho el registro a los asientos de cierre, se clasifican en el mayor general, luego se elabora la balanza de comprobación al cierre.











\end{document}
