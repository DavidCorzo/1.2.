\documentclass{article}
\author{David Gabriel Corzo Mcmath}
\title{Contabilidad Financiera Cap. 7}
\date{2019-09-09 22:18}
% % % % % % % % % % % % % % % % % % % % % % % % % % % % % % % % % % % % % % % % % % % % % % % % % % %
\usepackage[margin = 1in]{geometry}
\usepackage{graphicx}
\usepackage{fontenc}
\usepackage[spanish]{babel}
\usepackage{amsmath}
\usepackage{amsthm}
\usepackage[utf8]{inputenc}
\usepackage{enumitem}
\usepackage{mathtools}
\usepackage{import}
\usepackage{xifthen}
\usepackage{pdfpages}
\usepackage{transparent}
\usepackage{color}
\usepackage{enumitem}
% % % % % % % % % % % % % % % % % % % % % % % % % % % % % % % % % % % % % % % % % % % % % % % % % %%

\begin{document}
\maketitle
\section{Introducción}
\begin{itemize}
    \item A continuación se muestran las cuentas involucradas en la elaboración del balance general.
    \item La elaboración según las normas NIC
    \item El estado financiero presentado de la manera indicada proporciona información acerca de los siguientes elementos de la entidad te permiten evaluar la situación financiera de una empresa:
    \begin{itemize}
        \item Activos
        \item Pasivos
        \item Patrimonio neto
        \item Gastos e ingresos en los que se incluyen pérdidas y ganancias.
        \item Flujos de efectivo.
    \end{itemize}
    
    \item La información proporcionada sirve para \textbf{reducir} incertidumbre.
    \item El activo mostrando los recursos con los que cuenta la organización.
    \item El pasivo las obligaciones.
    \item Capital las aportaciones y los recursos internos que dispone un organismo.
    
\end{itemize}


\subsection{El activo}
\begin{itemize}
    \item Se analiza de manera jerárquica dividiéndose en activo circulante y no circulante, se clasifica de esta forma para saber qué tanta capacidad tiene dicho activo en cuestión de volverse en efectivo. \emph{\textbf{(Paréntesis ``efectivo'':}el efectivo aquí tiende a ser confundido a ser la cantidad de dinero que tiene una empresa en efectivo pero se estará refiriendo a esto como \textbf{efectivo disponible}, que se divide en dos, \emph{efectivo en caa y efectivo en inversiones a corto plazo.}\textbf{)}}
\end{itemize}

\section{Efectivo en caja y bancos}
La cuenta \textbf{efectivo} que posee una empresa se divide en dos partidas:
\begin{itemize}
    \item Caja chica: \emph{\textbf{Definición de ``caja chica":} fondo normalmente limitado para hacer transacciones que tienen que ser en efectivo o son muy pequeñas para incurrir en la molestia de hacerlo por medio de otra forma de pago}.
    \item Bancos: \emph{\textbf{Definición de ``bancos":} es una forma de mantener efectivo por medio de \textbf{cheques} es instituciones bancarias.} \emph{\textbf{(Paréntesis:} OJO si las cuentas producen intereses se les denomina \textbf{cuentas productivas} y otras que \underline{no} producen intereses se les denomina \textbf{cuentas de cheques}}\textbf{)}
\end{itemize}


\section{Inversiones a corto plazo}
\begin{itemize}
    \item Cuando hay excedentes de efectivo se aprovecha poniendolo en una \textbf{cuenta de inversiones} ya que estas dan beneficios que no dan las de cheques normales. 
    \item La suma de el dinero en bancosy en caja chica es el \textbf{efectivo disponible} 
\end{itemize}

\section{Relación del efectivo y las inversiones temporales con el ciclo de operación}
\begin{itemize}
    \item ``Estados de flujo dde efectivo'' según el NIC el efectivo está constituido por \textbf{la caja, los depósitos bancarios \& las inversiones a corto plazo con gran liquidez}.
    \item La norma menciona la terminología \emph{equivalentes al efectivo} esto son inversiones a corto plazo \textbf{fácilmente} convertibles a efectivo que conllevan relativamente bajo nivel de riesgo.
    \item Para una inversión ser calificada como quivalente a efectivo debe de cumplir lo anteriormente descrito. La inversión es equivalente a efectivo cuando tiene \textbf{pronta fecha de caducidad}, se establece esta como \textbf{menor a tres meses desde la fecha de adquisición}. 
    \item La cuenta \textbf{saldo de la cuenta de efectivo} comprende las transacciónes derivadas de las \textbf{actividades de operación} (ingresos y egresos de la entidad), \textbf{actividades de inversión} (adquisición y venta de activos a largo plazo), \textbf{actividades de financiación}.
    \item Casos especiales a considerar: se tiende a confundir con efectivo las cuentas \textbf{cheques posfechados} y \textbf{vales de caja}, recordar que los cheques posfechados (cheques que se emiten con una fecha de cobro posterior al presente) son \textbf{cuentas por cobrar} y vales de caja (vales firmados cuando se les da a los empleados efectivo para la realización de algo en específico) se registran en la cuenta \textbf{deudores diversos}. 
\end{itemize}


\section{Objetivos de control del efectivo}
\begin{itemize}
    \item Es importante el control del efectivo ya que con el se cubren los gastos de mercancías o servicios. 
    \item Hay una tendencia a ser mal invertido.
    \item La administración del efectivo y las dos áreas que comprende:
    \begin{itemize}
        \item El presupuesto de efectivo: sirve para la planeación.
        \item El control contable: para asegurar que el efectivo se usa únicamente en función de la empresa. 
    \end{itemize}
\end{itemize}

\section{Control interno}
\begin{itemize}
    \item Se pretende proteger el efectivo de uso no empresarial \textbf{salvaguardando contra desperdicio}, \textbf{asegurando correcta contabilización, analizar el cumplimiento de las políticas de la empresa, medir eficiencia operativa.}, Todo en función de proteger el efectivo de desperdicio, malos usos, robo, etc.
\end{itemize}

\section{Efectivo en caja y bancos}
\subsection{Fondo de caja chica: registro contable}
\begin{itemize}
    \item Los gastos menores se incurren con la caja chica ya que redactar cheques por sumas pequeñas es costos e impráctico.
\end{itemize}

\subsection{Sistema de fondo de la caja chica y registros contables}
Funcionamiento de la caja chica, pasos:
\begin{enumerate}
    \item Creación del fondo de caja chica: \textbf{determinar la cantidad de dinero necesaria para la caja}, tras hacer esta determinación se deduce de la cuenta \textbf{bancos} y se acredita a la cuenta \textbf{caja chica}.
    \begin{center}
    \begin{tabular}{ | p{9cm} | p{3cm} | p{3cm} | }
     \hline
    Cuenta & Debe & Haber \\
    \hline
    Caja chica & [La cantidad dinero determinada] & \\ 
    \hline
    Bancos & & [La cantidad dinero determinada] \\ 
     \hline
    \end{tabular}
    \end{center}
    Después se le entrega a la persona responsable del control de la caja chica.
    
    \item Erogaciones a través del fondo de la Caja Chica: El que es responsable de la Caja Chica debe de llevar el control total sobre la cantidad de dinero encomendada, debe de emitir comprobantes para \textbf{rendir cuentas del dinero que pueda desembolsar}.

    
    \item Reposición del fondo de caja chica: Cuando el fondo de la caja chica \textbf{está por agotarse} se debe solicitar que se reponga los desembolsos. Para ello se genera un reporte describiendo cada cosa que se desembolso, para determinar la cantidad que tiene que remunerar la cuenta de \textbf{bancos}.
    \begin{center}
    \begin{tabular}{ | p{9cm} | p{3cm} | p{3cm} | }
     \hline
    Cuenta & Debe & Haber  \\
    \hline
    Gasto 1 & 10 & \\ 
    Gasto 2 & 10 & \\ 
    Gasto 3 & 10 & \\ 
    Bancos & & 30\\ 
    \hline
    \end{tabular}
    \end{center}

    \item Incremento o disminución del fondo de la caja chica:  Si la cantidad de dinero determinada anteriormente (paso 1) no es suficiente para abastecer las necesidades o en lo contrario es excesiva, se debe de reajustar esa cantidad, para ello se emiten los siguientes asientos:
    \begin{itemize}
        \item Asiento en disminución:
        \begin{center}
        \begin{tabular}{ | p{9cm} | p{3cm} | p{3cm} | }
         \hline
        Cuentas & Debe & Haber \\
        \hline
        Bancos & [cantidad excedente en la caja chica] &  \\ 
        Caja chica & & [cantidad excedente en la caja chica] \\ 
         \hline
        \end{tabular}
        \end{center}

        
        \item Asiento en aumento:
        \begin{center}
        \begin{tabular}{ | p{9cm} | p{3cm} | p{3cm} | }
         \hline
         Cuentas & Debe & Haber \\
         Caja chica & [cantidad que se necesita] & \\ 
         Bancos & & [cantidad que se necesita] \\ 
         \hline
        \end{tabular}
        \end{center}
    \end{itemize}
\end{enumerate}

Comentario: Ojo que si el responsable de la caja chica produce desembolsos que no tienen comprobantes no se registra la cantidad desembolsada en caja chica si no en \textbf{gastos varios.}


\section{Ejemplo pg. 240}
Ver ejemplo en el pdf adjunto.

\section{Efectivo en bancos (cuentas de cheques): Registro contable}
\begin{itemize}
    \item Al abrir una cuenta de \textbf{cheques} el banco pide una tarjeta de firmas de todas las personas que usarán firmas para poder tener una referencia si la firma de un cheque no parece a las registradas.
    \item Tras la apertura de la cuenta, el banco entrega cheques con el nombre, dirección y número de cuenta de la empresa.
    \item Al final del mes el banco manda a la empresa un estado de cuenta que contiene el resumen de las transacciónes realizadas.
    \item Las transacciones de la cuenta no relacionadas con la empresa se clasifican por el banco con una letra especial (cobro por manejo de cuenta por ejemplo).
    \item \textbf{Cheques de depósito} se preparan siempre doble, uno para el banco una para la empresa, este tipo de cosas se hacen por motivos de registrar historial y si llegase el caso por auditoría.
\end{itemize}

\section{Procedimiento para la conciliación bancaria}
A raíz que muchas veces el efectivo mostrado en el estado de cuenta no concuerda con las cifras registradas por la empresa a continuación muestro una lista de razones responsables de este fenómeno:
\begin{itemize}
    \item Depósitos en tránsito: producido por el retardo en la sincronización de la cuenta de banco con los libros de la compañía.
    \item Cheques pendientes de cobro: mientras los cheques emitidos no sean presentados al banco no se toman en cuenta en el estado de cuenta.
    \item Errores del banco: que el banco se confunda con modificaciones que corresponden a otra compañía por ejemplo.
\end{itemize}

Cinco razones fundamentales de la discrepancia de la cifras que tiene la compañía y el banco:
\begin{enumerate}
    \item Cargos por servicios bancarios: Se descuenta cargos por servicios automáticamente en la cuenta del banco.
    \item Depósito de cheques sin fondos: Cheques que se encuentran sin fondos.
    \item Cobro de documentos:El banco cobra honorarios por el servicio de cobrar a los clientes.
    \item Pago de documentos:Cuando la compañía autoriza al banco que un documento pendiente se ha pagado en la fecha de su vencimiento, Este es un cargo el banco paga desde la cuenta de la empresa automáticamente.
    \item Errores en los libros:La compañía puede cometer un error al registrar una transacción y no darse cuenta hasta compararlo con el estado de cuenta del banco.
\end{enumerate}


\section{Elaboración de conciliación bancaria}
Tras recibir el estado de cuenta se intenta sincrinizar la cuenta con los libros locales de la empresa.
Tres formas de conciliación:
\begin{itemize}
    \item A partir del saldo de los registros contables de la empresa se llega al saldo del estado de cuenta. (Libros $\Rightarrow $ Estado de cuenta)
    \item Con base en el saldo del estado de cuenta se llega al saldo en libros. (Estado de cuenta $\Rightarrow$ Libros)
    \item Por conciliación cuadrada, en la que se parte de ambos saldos para llegar a un saldo conciliado. 
\end{itemize}

\textbf{Dos etapas de la conciliación para llegar a un consenso del saldo correcto:}
\begin{enumerate}
    \item Determinar \textbf{diferencias en el estado del banco}.
    \item Precisar \textbf{diferencias en el saldo en libros}.
\end{enumerate}

\section{Procedimiento para conciliar el saldo en bancos}
\begin{enumerate}
    \item Anotar el saldo en estado de cuenta.
    \item Determinar los depósitos en tránsito.
    \item Determinar los cheques pendientes y los que ya fueron pagados por el banco.
    \item Se revisa para encontrar errores y se produce el \textbf{saldo de banco conciliado}.
\end{enumerate}

\section{Procedimiento para conciliar el saldo en libros}   
\begin{enumerate}
    \item Anotar el saldo en libros.
    \item Determinar cobros de documentos e interéses realizados por el banco, se agregan al estado de cuenta.
    \item Verificación de deducciónes no contempladas como:
    \begin{itemize}
        \item Servicios bancarios 
        \item Cobro de documentos 
        \item Pago de documentos \& intereses 
        \item Depositados de cheques sin fondos de clientes.
    \end{itemize}
    
    \item Se revisa para encontrar errores y se ratifican.
    \item Se contemplan todas las deducciones o adiciones correspondientes y se produce el \textbf{saldo en libros conciliado} \emph{\textbf{(Paréntesis:}\text{Saldo en libros conciliado == Saldo bancario conciliado}\textbf{)}}
\end{enumerate}


\section{Registro contable para actualizar el saldo en bancos}
Las deducciones y adiciones que se efectuan se cargan o debitan todas a la cuenta de bancos.

\section{Ejemplo de una conciliación pg. 244}
\emph{\textbf{(Paréntesis:}\textbf{)}La conciliación bancaria es un \textbf{reporte} que se presenta en forma mensual y se guarda en los archivos permanentes de la compañía.}\newline 
La conciliación debe contener lo siguiente:
\begin{itemize}
    \item Nombre de la compañía 
    \item Nombre del estado de conciliación bancaria
    \item Fecha de conciliación
\end{itemize}


\section{NIF aplicables a la partida de efectivo}
Normas: NIC 1  presentación de estados financieros.
NIC 7 estados de flujo de efectivo.


\emph{\textbf{Definición de ``Concepto de efectivo según NIF":} El efectivo está integrado por caja, depósitos bancarios a la vista y los equivalentes al efectivo (inversiones a corto plazo fácilmente convertibles en efectivo).} \newline 


Reglas:
\begin{itemize}
    \item El efectivo y su equivalente es un activo corriente.
    \item Equivalentes a efectvo son para cumplir compromisos a corto plazo así mismo es fácilmente convertible a efectivo, y es de poco riesgo.
    \item Los sobregiros bancarios son componentes de efectivo $\equiv$ al efectivo.
    \item Se debe de informar flujo de efectivo mediante al estado de flujo de efectivo.
\end{itemize}

\section{Inversiones temporales}
\begin{itemize}
    \item Cuando hay un excedente en la estimación de efectivo de una empresa ese efectivo hay que \textbf{sacarle provecho} la manera para hacer eso es por medio de inversiones temporales, estas deben cumplir con las siguientes características:
    \begin{enumerate}
        \item Fácilmente convertibles 
        \item Se debe de tener como objetivo volverlas de nuevo a efectivo dentro del ciclo de operación.
    \end{enumerate}
    
    \item Las empresas derivan beneficios marginales de estas inversiones financiadas por excedente de efectivo por ende se tiende a preferir a invertir este efectivo que a que se quede ahí sin producir nada.
\end{itemize}

\section{Costo de adquisición}
\emph{\textbf{Definición de ``costo de adquisición":} es el costo con el que se adquirió un activo. }\newline 
La transacción referente a inversiones temporales debe mostrar cada tipo de inversión, a continuación los tipos de inversión que se deben registrar:
\begin{itemize}
    \item El costo de adquisición
    \item La fecha 
    \item Número de acciones 
    \item Obligaciones 
    \item Certificados 
    \item Valores poseídos.
    \item Costo por unidad
\end{itemize}
A continuación la partida de inversiones temporales vs. bancos:
\begin{center}
\begin{tabular}{ | p{9cm} | p{3cm} | p{3cm} | }
 \hline
Cuenta & Debe & Haber \\
\hline
Inversiones temporales & [monto de inversión] & \\ 
Bancos & & [monto de inversión] \\ 
 \hline
\end{tabular}
\end{center}

\section{instrumentos de inversión comunes}
Opciones a invertir efectivo excedente para el publico inversionista:
\begin{itemize}
    \item Opción de \textbf{invertir en sociedades de inversión} 
    \item Opción de \textbf{inversión de acciones }
    \item Opción de \textbf{instrumentos gubernamentales}
\end{itemize}

\section{Sociedad de inversión}
\emph{\textbf{Definición de ``sociedad de inversión":} son sociedades anónimas que emiten acciones para ser adquiridas por el gran público inversionista, el dinero obtenido por la venta de acciones se utiliza en sí para la compra de acciones de otras empresas industriales, comerciales, bancarias de servicio, etc.}
La peculiaridad de estas sociedades es que aceptan a inversionistas que aportan poco, perfecto para inversión de excedente del efectivo para la caja chica.

\section{Registro de acciones en sociedad de inversion}
Ver ejemplo pg.248
\begin{itemize}
    \item Se hace un cargo a bancos y un abono a inversiones temporales al registrar la inversión, después el dinero más los beneficios son registrados en bancos y se deduce de la cuenta de inversiones temporales e ingresos.
\end{itemize}

\section{Inversión de acciones}
\emph{\textbf{Definición de ``Inversión de acciones":} cuando se adquiere un conjunto de acciones de diferentes empresas se debe calcular el precio de cada acción adquirida para monitoriar momentos de ganancia o pérdida}
\begin{itemize}
    \item Al comprar: Se hace un cargo a bancos y un abono a inversiones temporales.
    \item Al vender: Se hace lo contrario a la compra y adicional se registran los ingresos por intereses.
    \item Ingresos Ganados == (precio de venta - Precio de compra) x Número de acciones
\end{itemize}

\section{Instrumentos gubernamentales}
\emph{\textbf{Definición de ``Instrumentos gubernamentales":}  cuando los gobiernos emiten instrumentos de inversión para financiar sus operaciones.} \newline 
Se caracterizan por ser seguros y poco riesgosos por su pago nominal estar asegurado por el gobierno federal, tiene fácil convertibilidad a efectivo.
\begin{itemize}
    \item En EEUU se llaman ``Treasury bills''
    \item En Mex. se llaman ``Certificados de la federación''
\end{itemize}

\section{Normas de información financiera aplicables a inversiones temporales}
Igual que las NIF regulan la cuenta de efectivo regulan igual la cuenta de \textbf{inversiones temporales}.\newline 
\newline 
Normas: 
NIC 32 Instrumentos financieros: presentación de información a revelar.


\emph{\textbf{Definición de ``concepto de inversiones temporales según NIF":} Las inversiones temporales son un equivalente a efectivo, son fácilmente transformadas a efectivo, a un riesgo significativamente reducido.}

Reglas:
\begin{itemize}
    \item Son clasificadas como \textbf{activos corrientes}.
    \item Se registran con su valor de mercado.
    \item Es un equivalente a efectivo.
    \item La diferencia entre valuación de un periódo a otro deberá registrarse en el estado de resultados del periodo en el que ocurre.
    \item Los costos de rendimiento deben de ser reconocidos en el estado de resultados del periodo que devengan.
\end{itemize}

\section{Ánalisis financiero}
Las partidas de \textbf{efectivo} e \textbf{inversiones temporales} tienen relación con la liquidez que una empresa tiene, recordar la razón circulante:
\[
  \text{Razón Circulante} = \frac{\text{Activos circulantes}}{\text{Pasivos circulantes}} 
\]
Recordar que la razón circulante sirve para medir la liquidez de una empresa.\newline 
Recordar la prueba del ácido.
\[
  \text{Prueba del ácido} = \frac{\text{Activos circulantes - Inventario}}{\text{Pasivos circulantes}} 
\]

\rule{16cm}{1pt}

\section{Discusión de clase acerca del capítulo - 2019-09-18}
\begin{itemize}
    \item Diferencia entre ``cheque'' y ``cheque de caja'' en el cheque de caja se garantiza que hay fondos en el cheque.
    \item \emph{\textbf{Observación: }Cada cheque tiene un número, \textbf{Nos preguntamos:} ¿porque?por que hay gente que que falsifica la firma y las cheques.}
    \item \emph{\textbf{Ojo: considerar lo siguiente...} El activo circulante es a corto plazo, la no circulante es para largo plazo.}
    \item Activo circulante:
    \begin{itemize}
        \item Bancos
        \item Efectivo 
        \item Inversiones temporales
    \end{itemize}
    
    \item Activo no circulante:
    \begin{itemize}
        \item Equipo de cómputo 
    \end{itemize}

    \item Efectivo en caja y bancos:
    \begin{itemize}
        \item Billetes
        \item Cuentas monetarias 
        \item Cuenta de ahorro
    \end{itemize}
    
    \item Inversiones a corto plazo:
    \begin{itemize}
        \item Cuentas que dan rendimiento 
        \item Certificados de depósito plazo fijo.
    \end{itemize}
    
    \item \emph{\textbf{Observación: }Circulante y corriente son lo \textbf{mismo}.}
    \item Control interno:
    \begin{itemize}
        \item Proteger ante desperdicios
        \item Legitimizar la contabilidad, que sea real.
        \item Alentar y medir el cumplimiento de las políticas de la empresa.
        \item Juzgar la eficiencia operativa.
    \end{itemize} 
    
    \item \emph{\textbf{Observación: }Cámara de compensación, para trasladar fondos entre el un banco y otro.}
\end{itemize}





\end{document}
