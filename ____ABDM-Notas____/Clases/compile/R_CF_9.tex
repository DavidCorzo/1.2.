\documentclass{article}
\author{David Gabriel Corzo Mcmath}
\title{Contabilidad financiera - Capitulo 9 }
\date{2019-09-24 21:03}
% % % % % % % % % % % % % % % % % % % % % % % % % % % % % % % % % % % % % % % % % % % % % % % % % % %
\usepackage[margin = 1in]{geometry}
\usepackage{graphicx}
\usepackage{fontenc}
\usepackage[spanish]{babel}
\usepackage{amsmath}
\usepackage{amsthm}
\usepackage[utf8]{inputenc}
\usepackage{enumitem}
\usepackage{mathtools}
\usepackage{import}
\usepackage{xifthen}
\usepackage{pdfpages}
\usepackage{transparent}
\usepackage{color}
% % % % % % % % % % % % % % % % % % % % % % % % % % % % % % % % % % % % % % % % % % % % % % % % % %%

\begin{document}
\maketitle


\section{Introducción}
\begin{enumerate}
    \item La partida inventarios está constituida por los bienes de una empresa destinados a la venta 
\end{enumerate}

\section{Relación del inventario con el ciclo de compras y pagos}
\begin{enumerate}
    \item El inventario es importante por que es la principal fuente de ingreso para la empresa.
    \item Es como un puente entre Compras y ventas.
\end{enumerate}

\section{Costo de inventarios}
\begin{enumerate}
    \item NIC número 2:
    \item \textbf{Costo de inventarios}: Son todos los costos que se relacionan con la adquisición y transformación que produce la condición actual del inventario.
    \item \textbf{Precio de compra se compone por}:  los aranceles de importación y otros impuestos, los transportes, el almacenamiento y otros costos directamente atribuibles a la adquisición de las mercaderías, los materiales o los servicios.
\end{enumerate}


\section{Sistemas de registro, métodos de valuación y de estimación de inventarios}
Sirven para mejorar, agilizar y producir información financiera.
\begin{enumerate}
    \item Sistema de registro:
    \begin{itemize}
        \item La empresa decide cuándo registrar sus transacciónes de inventario, son dos: \textbf{inventario perpetuo, inventario periódico}.
    \end{itemize}
    \item Métodos de valuación:
    \begin{itemize}
        \item Dado a que los precios fluctuan durante un periodo se necesita determinar el costo que se asignará a las mercancías vendidas, se debe seleccionar el que represente mejor la \emph{utilidad neta y la actividad del negocio}.
        \item Métodos:
        \begin{enumerate}
            \item Costos identificados
            \item PEPS 
            \item UEPS 
            \item Promedio ponderado
        \end{enumerate}
    \end{itemize}
    \item Sistema de estimación:
    \begin{itemize}
        \item Cuando por razones no contempladas se debe producir un estado financiero y urge y no se puede hacer un recuento físico de todos los productos disponibles se intenta \textbf{estimar}, métodos son:
        \begin{enumerate}
            \item Precio de menudeo o detallista
            \item Utilidad bruta
        \end{enumerate}
    \end{itemize}
\end{enumerate}


\section{Sistema de registro de inventario}
\begin{enumerate}
    \item Sistema perpétuo:
    \begin{itemize}
        \item Se registran la mayoría de cuentas en \textbf{inventario}, (cuentas como: compras, fletes sobre compras, devoluciones y bonificaciones sobre compras o descuentos sobre compras)
        \item Mantiene un saldo actualizado de la mercancía disponible
        \item Se conoce el costo de ventas en todo momento
        \item No se acostumbra a hacer un recuento físico, es más preciso
    \end{itemize}
    \item Sistema periódico:
    \begin{itemize}
        \item No se mantiene actualizado las mercancias.
        \item Se usan muchas cuentas más ( como)
        \item Se hace un conteo física al final del periodo.
        \item Se calcula el \textbf{costo de la mercancía vendida} hasta el final del periodo no cada vez que se vende. \newline 
        \begin{center}
            \begin{tabular}{ | p{5cm} | p{5cm} | p{5cm} | }
                \hline
               Inventario inicial  \\ 
               + Compras  \\ 
               + Fletes sobre compras \\ 
               - Devoluciones y bonificaciones sobre compras \\ 
               - Descuentos sobre compras \\ 
               = Costo de las mercancías disponible \\ 
               - Inventario Final \\ 
               = Costo de ventas \\  
                \hline
               \end{tabular} 
        \end{center}

        \item Se hace el cirre de cuentas de compras, ventas.
        \item Se hace una estimación del inventario.
    \end{itemize}
\end{enumerate}


\section{Compras y cuentas afines}
\begin{enumerate}
    \item Compras: cuando se compra mercancía se registra en \textbf{compras},con contra partida proveedores.
    \item Devoluciones sobre compras: cuando se hacen devoluciones se hace un registro a \textbf{devoluciones sobre compras} con contrapartida de \textbf{proveedores}.
    \item Descuentos sobre compras: para incentivar a clientes a pagar pronto se hacen descuentos, tipos de descuento,: \newline 
    Aclaraciones preliminares, tienen la forma porcentage\_de\_descuento/día\_límite\_para\_pagar, n(monto\_neto\_/última\_fecha\_para\_pagar\_el\_monto\_neto).
    \begin{itemize}
        \item 2/10, n/30
        \item 2/10, 1/15, n/30
        \item 2/10 FDM, n/60 (FDM se refiere en este caso a los primeros diez días del siguiente mes, de lo contrario monto neto)
    \end{itemize}
\end{enumerate}

\section{Gastos adicionales que forman parte del producto}
\begin{enumerate}
    \item Fletes: Se registra en fletes sobre compras o fletes sobre ventas en el periódico y en inventario en perpétuo, son los costos de envío. Son costo de embarque y los convenios de embarque son:
    \begin{itemize}
        \item Libre a bordo (LAB) punto de embarque: el comprador paga todo, se registran en fletes sobre compras.
        \item Libre a bordo (LAB) punto destino: el vendedor paga todo y no se registra nada de parte del comprador.
    \end{itemize}
    Estos gastos se registran en el estado de resultados como \textbf{gastos generales}.


    \item Seguros: es una garantía de calidad de la mercancía, para cubrir riesgos se contrata un seguro, y esto froma parte del costo del producto.

    \item Impuestos de importación: son impuestos que se aplican cuando se importa la mercancía, comprenden dos tipos: 
    \begin{itemize}
        \item Impuesto sobre ventas locales
        \item Derecho a trámite aduanero
    \end{itemize}
\end{enumerate}


\section{Ventas y cuentas afines}
\begin{enumerate}
    \item Ventas: cuando se vende mercancía se hace un bono a \textbf{ventas} y se hace un cargo a \textbf{clientes} (en los dos sistemas de inventario).
    \item Costo de ventas (solo en el perpetuo se registra ya que el periódico se estima al final de periodo): se hace un cargo a \textbf{inventarios} con contrapartida de \textbf{costo de ventas}.
    \item Devoluciones sobre ventas: implica dos cuentas \textbf{devoluciones y bonificaciones sobre ventas} y \textbf{clientes} y se registra en los dos sistemas de la misma manera.
    \item El registro de la entrada de la mercancía de inventario (solo se registra en el perpétuo), implica \textbf{inventarios} y \textbf{costo de ventas}.
    \item Descuentos sobre ventas: implica tres cuentas, \textbf{bancos, descuentos sobre cobras y clientes} se registra de la misma manera en los dos sistemas de inventario.
    \item Descuentos comerciales: implica dos cuentas y se registra igual en los dos sistemas, implica las cuentas de \textbf{clientes y ventas}. El costo de la mercancía vendida se registra solo en el perpetuo ya que en el periódico se hace una estimación al final del periodo, el costo de mercancía vendida implica las cuentas \textbf{costo de ventas e inventarios}.
\end{enumerate}


\section{Valuación del inventario}
\begin{enumerate}
    \item Cuando se compra mercancía se registra con el precio de costo sin importar si tiene un descuento comercial, registro únicamente lo que me costo a mi con todo y el descuento. \newline 
    \begin{center}
        \begin{tabular}{ | p{5cm}| }
            \hline
            Compras \\
            + Fletes \\ 
            - Descuentos sobre compras \\ 
            = Costo de compras \\ 
            \hline
           \end{tabular}
    \end{center}
    
    \item Se puede calcular el costo de las mercancías no vendidas, solo se selecciona un método de estimación (como UEPS, PEPS, promedio ponderado, costos no identificados) y se registra, es importante escoger un método correcto ya que el inventario final afecta el estado de resultados.
\end{enumerate}

\section{Costos no identificados}
\begin{enumerate}
    \item Se rastrea cada venta con un número único para identificar cada producto vendido, de esta forma es posible determinar la factura correspondiente al artículo vendido.
    \item Es impráctico para empresas muy grandes por su alto costo del trabajo que conlleva dicho método.
    \item A pesar de esto es \textbf{exacto}.
    \item La suma del costo de ventas y el inventario final debe de ser igual al costo total.
\end{enumerate}


\section{Primeras entradas, primeras salidas, (PEPS)}
\begin{enumerate}
    \item Las primeras mercancias compradas son las primeras que se venden.
    \item Las mercancías al final del periodo serán las últimas, es decir, las de compra más reciente, valoradas al precio actual o al último precio de compra.
    \item La suma del costo de ventas y el inventario final debe de ser igual al costo total.
\end{enumerate}

\section{Últimas entradas, primeras salidas}
\begin{enumerate}
    \item Se supone que las últimas mercancías compradas son las primeras que se venden.
    \item La mercancía residual al final del periodo representa la que se encontraban en existencia en el inventario inicial.
    \item El inventario final debe valuarse según el primer precio de compra o el más antiguo.
    \item El UEPS tiende a dar valores de inventario final menores (subvaluado) y costos de ventas mayores (sobrevaluado)
    \item Las entidades IASB \& CINIF no permiten el uso de este método. 
\end{enumerate}


\section{Promedio ponderado}
\begin{enumerate}
    \item Reconoce que los precios varían por eso busca valuar cada artículo en el promedio de la fluctuación de precios de la mercancía en existencia durante todo el año.
    \item Pasos de este método: Determinar el costo promedio por unidad, después aplicar el promedio al número de unidades del inventario final.
\end{enumerate}

\section{Comparación de los métodos de valuación}
El método UEPS provee una ventaja de tener el costo de ventas a los valores más recientes, todos son aceptables contablemente (depende de la entidad reguladora de contabilidad).

\section{Congruencia de los métodos de valuación}
\begin{enumerate}
    \item Cada compañía puede optar por el mejor método que le convenga.
    \item Son importantes ya que proveen información financiera, pero el contador debe de aclarar qué método se usa para la más precisa interpretación de la información producida.
    \item Si se cambia el método el contador debe especificar y justificar por qué se cambió. 
\end{enumerate}

\section{Sistemas de estimación de inventarios}
Cuando con urgencia y con pocos recursos se debe saber el valor que hay en inventario se usan los métodos de estimación. 
\begin{enumerate}
    \item El método de precios al menudeo o detallista: usualmente usado por las empresas mayoristas, las cadenas de tiendas.
    \begin{itemize}
        \item Utiliza el costo y el precio al menudeo de los productos disponibles para la venta, para obtener una relación al inventario final a precios al menudeo para obtener el inventario final al costo.
        \item Usa dos columnas: costo y precio al menudeo. \newline 
        \begin{center}
            \begin{tabular}{ | p{5cm} | p{5cm} | p{5cm} | }
                \hline
                & Costo & Precio al menudeo \\ 
               Inventario inicial &  [Montos] & [Montos]\\
               + Compras del año & [Montos] & [Montos] \\ 
               + Gastos de fletes & [Montos] & \\ 
               Total & [Montos totales] & [Montos totales] \\ 
               - Devoluciones sobre compras & [Montos] & [Montos] \\ 
               = Mercancía disponible para la venta, estimada & [Montos totales] & [Montos totales] \\ 
               - Ventas netas & & [Montos] \\ 
               Inventario final al precio de menudeo & & [Montos] \\ 
               Relacion costo a precio, al menudeo & & [Porcentaje] \\
               = Inventario al costo estimado & & [Montos totales] \\ 
                \hline
               \end{tabular}
        \end{center}
    \end{itemize}
    \item El método de utilidad bruta: Usualmente usado por razones que se ha destruido o perdido la mercancía. Pasos para usar este método:
    \begin{enumerate}
        \item Costo de compras + costo de inventario inicial + cargos por fletes = Subtotal.
        \item Subtotal - devoluciones sobre compras = costo de mercancías disponibles para la venta.
        \item Ventas netas - Devoluciones sobre ventas = costo de la mercancía vendida.
        \item Ventas netas - utilidad bruta estimada = mercancías disponibles para la venta, posteriormente determinar el inventario final estimado a precios de costo. 
    \end{enumerate}
\end{enumerate}

\section{Efecto de los errores de inventarios}
\begin{itemize}
    \item Los errores de inventario afecta la utilidad neta que aparece en el estado de resultados, también estará presente en los activos circulantes del balance general del periodo en curso.
    \item Cuando el inventario final muestra una cantidad de inventario mayor a la existencia real el costo de mercancías vendidas será menor $\Rightarrow$ resultado más alto de la utilidad neta.
    \item Cuando la cantidad sea menor a la real en existencia $\Rightarrow$ el costo de las mercancías vendidas sera más alto, por ende refleja utilidades netas menores.
\end{itemize}

\section{Aplicación de la recla de costo o valor neto realizable: el menor}
Los inventarios  pueden sufrir importantes variaciones de costo debido a cambios de los precios de mercado, obsolescencia u otras razones. Por tal motivo es necesario aplicar la regla de costo  o valor neto realizable : el menor.
\begin{enumerate}
    \item Menor de partida por partida: De acuerdo con este método, se valúa partida por partida del inventario y a cada una se le asigna el menor valor entre el costo y valor de mercado
    \item Método de categorías: Según este método, los artículos se agrupan por categorías y el costo total histórico y el valor neto realizable se comparan para determinar el menor de ellos
\end{enumerate}

\section{Normas de información financiera aplicables a inventarios}
\begin{itemize}
    \item NIC 2: Son los activos poseídos para ser vendidos en el curso normal de la operación; en proceso de producción de cara a tal venta; en forma de materiales o suministros para ser consumidos en el proceso de producción, o en el suministro de servicios.
    \item Reglas:
    \begin{itemize}
        \item Son medidos al costo o valor neto realizable según sea menor.
        \item Costo de inventarios debe de incluir todos los costos derivados de su adquisición.
        \item Se deben revelar en los estados financieros qué métodos se emplean en relación a inventarios.
    \end{itemize}
\end{itemize}



\section{Análisis financiero}

\subsection{Rotación de inventario}
Indica el número de veces que el inventario es vendido en su totalidad y ha sido repuesto nuevamente.
\[
  \text{Rotación de inventario} = \frac{\text{Costo de ventas}}{\text{inventarios}}
\]

Se recomienda usar el inventario promedio.

\subsection{Los días de inventario se calculan a partir de la rotación de inventarios}
Representa los días que tarda el inventario en renovarse completamente, se calcula de dos maneras:

\begin{align*}
    \text{Días de inventarios} & = \frac{365}{\text{Rotación de inventarios}}\\ 
    \text{Días de inventario} & = \frac{\text{Inventarios}}{\text{Costo de ventas} * 360} \\ 
\end{align*}


\section{Efecto de inflación sobre inventario}
Cuando hay un cambio significativo en la inflación se debe actualizar el valor de inventario, (se empieza a considerar el cambio si el cambio es mayor a 20\%) y para actualizar la partida de inventarios se utiliza algunos de los siguientes métodos:

1. Índice general de precios (IGP): Indica crecimiento o decaimiento de la inflación de un país, se actualiza con la siguiente fórmula: 
\[
   \text{Valor actualizado } =  \frac{\text{IGP a la fecha de elaboración de los estados financieros }}{\text{IGP a la fecha de adquisición del inventario}}
\]

Posteriormente \dots 
\[
  \text{Valor actualizado} - \text{Valor histórico} = \text{Actualización}
\]

2. Valor actual, costo de reposición o método de índices específicos: 
Representa el costo en que incurriría la empresa en la fecha de elaboración de los estados financieros para adquirir o producir un artículo igual al que integra su inventario. Este método se utiliza cuando la empresa decide que el IGP no representa el incremento que han sufrido los artículos de su inventario







\end{document}
