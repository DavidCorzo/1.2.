\documentclass{article}
\title{Contabilidad financiera Capitulo \# 13}
\author{David Gabriel Corzo Mcmath}
\date{2019-Oct-29 18:33:07}
%%%%%%%%%%%%%%%%%%%%%%%%%%%%%%%%%%%%%%%%%%%%%%%%%%%%%%%%%%%%%%%%%%%%%%%%%%%%%%%%%%%%%%%%%%%%%%%%%%%%%%%%%%%%%%%%%%%%%%%%%%%%%%%%%%%%%%%%%%%%%%%
\usepackage[margin = 1in]{geometry}
\usepackage{graphicx}
\usepackage{fontenc}
\usepackage{pdfpages}
\usepackage[spanish]{babel}
\usepackage{amsmath}
\usepackage{amsthm}
\usepackage[utf8]{inputenc}
\usepackage{enumitem}
\usepackage{mathtools}
\usepackage{import}
\usepackage{xifthen}
\usepackage{pdfpages}
\usepackage{transparent}
\usepackage{color}
\usepackage{fancyhdr}
\usepackage{lipsum}
\usepackage{sectsty}
\usepackage{titlesec}
\usepackage{calc}
\usepackage{lmodern}
\usepackage{xpatch}
\usepackage{blindtext}
\usepackage{bookmark}
\usepackage{fancyhdr}
\usepackage{xcolor}
\usepackage{tikz}
\usepackage{blindtext}
\usepackage{hyperref}
%%%%%%%%%%%%%%%%%%%%%%%%%%%%%%%%%%%%%%%%%%%%%%%%%%%%%%%%%%%%%%%%%%%%%%%%%%%%%%%%%%%%%%%%%%%%%%%%%%%%%%%%%%%%%%%%%%%%%%%%%%%%%%%%%%%%%%%%%%%%%%%
\begin{document}
\maketitle


\section{Introducción}
\begin{enumerate}
    \item El último de los tres conceptos que conforman el balance general es el capital contable.
    \item \emph{\textbf{Definición de ``Capital contable":} representado por los recursos de que dispone la entidad, aportado por los propietarios, accionistas socios y que fueron acumulado mediante las utilidades generadas, etc.}
\end{enumerate}

%%%%%%%%%%%%%%%%%%%%%%%%%%%%%%%%%%%%%%%%%%%%%%%%%%%%%%%%%%%%%%%%%%%%%%%%%%%%%%%%%%%%%%%%%%%%%%%%
\section{Concepto de capital contable}
\begin{enumerate}
    \item \emph{\textbf{Definición de ``capital en contabilidad":} la diferencia entre el activo y pasivo}
    \item El capital contable se distingue del capital social, el capital social es la aportación de los accionistas en la empresa.
    \item Marco conceptual de las NIIF delega al patrimonio neto que se puede subdividir para efectos de presentación en el balance general
\end{enumerate}










\end{document}
