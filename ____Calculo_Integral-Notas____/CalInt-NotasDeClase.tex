\documentclass[openany]{book}
\author{David Gabriel Corzo Mcmath}
\title{\huge Cálculo Integral \normalsize \\ Notas de clase }
\date{2019-09-20 08:07}
% % % % % % % % % % % % % % % % % % % % % % % % % % % % % % % % % % % % % % % % % % % % % % % % % % %
\usepackage[margin = 1in]{geometry}
\usepackage{graphicx}
\usepackage{fontenc}
\usepackage{pdfpages}
\usepackage[spanish]{babel}
\usepackage{amsmath}
\usepackage{amsthm}
\usepackage[utf8]{inputenc}
\usepackage{enumitem}
\usepackage{mathtools}
\usepackage{import}
\usepackage{xifthen}
\usepackage{pdfpages}
\usepackage{transparent}
\usepackage{color}
\usepackage{fancyhdr}
\usepackage{lipsum}
\usepackage{sectsty}
\usepackage{titlesec}
\usepackage{calc}
\usepackage{lmodern}
\usepackage{xpatch}
\usepackage{blindtext}
\usepackage{bookmark}
\usepackage{fancyhdr}
\usepackage{xcolor}
\usepackage{tikz}
\usepackage{blindtext}
%%%%%%%%%%%%%%%%%%%%%%%%%%%%%%%%%%%%%%%%%%%%%%%%%%%%%%%%%%%%%%%%%%%%%%%%%%%%%%%%%%%%%%%%%%%%%%%%
\input{settings/page.tex}
\input{settings/chapter_head_font_whole_page.tex}
% \newcommand*\circled[1]{\tikz[baseline=(char.base)]{
%             \node[shape=circle,fill=gray!50,inner sep=2pt] (char) {#1};}}

% % header style
% \pagestyle{fancy}
% \fancyhf{}
% \fancyhead[EL]{\nouppercase\leftmark}
% \fancyhead[OR]{\nouppercase\rightmark}
% \fancyfoot[C]{\circled{\thepage}}
\newcommand*\circled[1]{\tikz[baseline=(char.base)]{
            \node[shape=circle,fill=gray!50,inner sep=2pt] (char) {#1};}}

% header style
\pagestyle{fancy}
\fancyhf{}
\fancyhead[EL]{\nouppercase\leftmark}
\fancyhead[OR]{\nouppercase\rightmark}
\fancyfoot[C]{\circled{\thepage}}

\fancypagestyle{plain}{%
  \fancyhf{}
  \fancyfoot[C]{\circled{\thepage}}
  \renewcommand{\headrulewidth}{0pt}
}



% % % % % % % % % % % % % % % % % % % % % % % % % % % % % % % % % % % % % % % % % % % % % % % % % % %
\begin{document}
\maketitle
\tableofcontents


\part{Material de clase, Notas}

\chapter{Antiderivadas, integrales indefinidas, notación de integral, reglas básicas de integración, integrales definidas, Teorema fundamental del cálculo}
\includepdf[pages=-,pagecommand={\thispagestyle{plain}}]{pdf/MC_01-2019-07-23.pdf}
\includepdf[pages=-,pagecommand={\thispagestyle{plain}}]{pdf/DA_3.pdf}


\chapter{Área, desplazamiento y propiedades, sobre vuelo acerca de la regla de la integral definida (inversión de límites con signo negativo), propiedades de integrales definidas, fórmula de área \& fórmula de desplazamiento.}
\includepdf[pages=-,pagecommand={\thispagestyle{plain}}]{pdf/MC_02-2019-07-25.pdf}


\chapter{Desplazamiento \& Distancias, presencia de las integrales en la economía, aceleración, velocidad, desplazamiento a partir de una función integrable; propiedad de funciones pares e impares.}
\includepdf[pages=-,pagecommand={\thispagestyle{plain}}]{pdf/MC_03-2019-07-30.pdf}


\chapter{Teorema fundamental del cálculo parte I \& parte II, ¿Cómo utilizo este teorema para derivar integrales con límites?}
\includepdf[pages=-,pagecommand={\thispagestyle{plain}}]{pdf/MC_04-2019-08-01.pdf}


\chapter{Regla de la sustitución, equivalente a la regla de la cadena en derivadas solo que con integración}
\includepdf[pages=-,pagecommand={\thispagestyle{plain}}]{pdf/MC_05-2019-08-06.pdf}


\chapter{Integración por partes (IPP), ILATE ó LIPET, IPP para definidas}
\includepdf[pages=-,pagecommand={\thispagestyle{plain}}]{pdf/MC_06-2019-08-08.pdf}


\chapter{Integrales trigonométricas, explorar las formas y los casos especiales}
\includepdf[pages=-,pagecommand={\thispagestyle{plain}}]{pdf/MC_07-2019-08-13.pdf}


\chapter{Continuación de Integración trigonométricas, curioso: Área de un circulo unitario, introducción a sustitución trigonométrica}
\includepdf[pages=-,pagecommand={\thispagestyle{plain}}]{pdf/MC_08-2019-08-20.pdf}


\chapter{Integración por sustitución trigonométrica}
\includepdf[pages=-,pagecommand={\thispagestyle{plain}}]{pdf/MC_09-2019-08-22.pdf}


\chapter{Repaso de sustitución trigonométrica} %Continuación integración por sustitución trigonométrica
\includepdf[pages=-,pagecommand={\thispagestyle{plain}}]{pdf/MC_10-2019-08-27.pdf}


\chapter{Repaso del parcial simulacro}
\includepdf[pages=-,pagecommand={\thispagestyle{plain}}]{pdf/MC_11-2019-08-29.pdf}


\chapter{Integrales impropias, tipo uno \& tipo dos}
\includepdf[pages=-,pagecommand={\thispagestyle{plain}}]{pdf/MC_12-2019-09-03.pdf}


\chapter{\textbf{Área entre curvas}, pasos para sacar área entre curvas irregulares}
\includepdf[pages=-,pagecommand={\thispagestyle{plain}}]{pdf/MC_13-2019-09-05.pdf}


\chapter{Aplicación de la integración, integración respecto a y para encontrar áreas entre curvas, integración respecto a x para encontrar áreas entre curvas, introducción a \textbf{volúmenes}}
\includepdf[pages=-,pagecommand={\thispagestyle{plain}}]{pdf/MC_14-2019-09-10.pdf}


\chapter{Volúmenes, sólidos en revolución, cilíndros \& solidos huecos}
\includepdf[pages=-,pagecommand={\thispagestyle{plain}}]{pdf/MC_15-2019-09-12.pdf}

\chapter{Continuación de volúmenes, volúmenes con un cascarón cilíndrico.}
\includepdf[pages=-,pagecommand={\thispagestyle{plain}}]{pdf/MC_16-2019-09-17.pdf}


\chapter{Valor promedio de una función}
\includepdf[pages=-,pagecommand={\thispagestyle{plain}}]{pdf/MC_17-2019-09-19.pdf}

\chapter{Longitud de arco}
\includepdf[pages=-,pagecommand={\thispagestyle{plain}}]{pdf/MC_18-2019-09-24.pdf}

\chapter{Probabilidades, función de densidad, distribuciones de probabilidad comunes(tipo uniforme, tipo normal, tipo exponencial)}
\includepdf[pages=-,pagecommand={\thispagestyle{plain}}]{pdf/MC_20-2019-09-26.pdf}

\chapter{Continuación Probabilidad}
\includepdf[pages=-,pagecommand={\thispagestyle{plain}}]{pdf/MC_21-2019-10-01.pdf}

\chapter{Fracciones parciales \\ Caso 1: Factores lineales distintos \\ Caso 2: Factores lineales repetidos}
\includepdf[pages=-,pagecommand={\thispagestyle{plain}}]{pdf/MC_22-2019-10-03.pdf}

\chapter{Fracciones parciales \\ Caso 3: Factores cuadráticos irreducibles \\ Caso 4: Factores cuadráticos repetidos }
\includepdf[pages=-,pagecommand={\thispagestyle{plain}}]{pdf/MC_23-2019-10-08.pdf}





%%%%%%%%%%%%%%%%%%%%%%%%%%%%%%%%%%%%%%%%%%%%%%%%%%%%%%%%%%%%%%%%%%%%%%%%%%%%%%%%%%%%%%%%%%%%%%%%






\end{document}
