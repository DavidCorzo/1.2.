\section{Revisión de el examen}
Bajo la valoración objetiva es imposible quebrar una empresa, por oto lado en el valor subjetivo se produce un proceso libre.
\newline 
\textbf{Definition: }La función empresarial crea información que se propaga en el mercado.
\newline 
En el marginalismo se valoran los bienes que mejor satisfagan las necesidades existentes.
\newline 

\subsection{Información relevante}
\textbf{\emph{Definición} Incertidumbre:} el empresario
\textbf{\emph{Definición} Riesgo:} las aseguradoras

\section{Noticia}
La noticia de cómo afectaría la ingesta de 1 millón de inmigrantes al país. Se estima que la curva de la oferta se desplarará a la derecha aumentando la oferta significativamente.

\section{Discusión de clase}
\begin{itemize}
    \item Ejemplo de españa de inmigración musulmana; los latinos suelen integrarse muy bien a la sociedad española.
    \item Ley de sey, para demandar cosas hay que ofrecer primero \textbf{\emph{Definition} Ley de Sey:} una oferta es una demanda, si se ofrece algo se ofrece por que el oferente esta demandando otra. 
    \item Los inmigrantes producen cosas para demandar cosas, entonces es posible que los inmigrantes absorbidos en Guatemala, desplacen no solo la oferta para la derecha si no que también la demanda aumentará también.
\end{itemize}

\section{Función empresarial}
\textbf{Nos preguntamos:} ¿es el empresario maximizador? y si es ¿de que? \textbf{Nos preguntamos:} ¿por qué el empresario ve hacia el futuro como un historiador?
\emph{Paréntesis: La teoría de marxismo opera en cuanto las gafas que te pongas(confirmation bias)} el empresario debe evaluar con información presente posibles futuros que el debe imaginarse. El que pudo dar a la humanidad la capacidad de llamar a las personas a distancia se volvió rico por que se imaginó en el futuro. \newline 
\emph{Respuesta:El empresario si maximiza ya que por interés propio busca maximizar su beneficio}, \textbf{Ejemplo: } El petroleo, no solía servir para nada antes de el desarrollo del carro, la información producida de la función empresarial, se encontro un nuevo fin a un mismo bien, gracias a la función empresarial; también se puede producir mediante nuevos medios, como el iPhone es un nuevo medio de desarrollo tecnológico que te permite cumplir tus fines de una manera más efectiva. \newline \newline 
Entonces el empresario está creando nuevos fines y nuevos medios, \textbf{Ejemplo: } Henry Ford, \emph{Citación:"No le pregunta a los clientes qué quieren por que los clientes no saben qué quieren"}

\subsection{Conocimiento práctico}
El conocimiento práctico es conocimiento que es difícilmente articulable, no es tipo científico. La economía es un conocimiento cientifico articulado sobre reacción y transmisión de conocimiento no científico. En otras palabras dar fundamento científico a el conocimiento práctico no articulable.\newline

El problema de la religión es que no es articulable, como decir que las hormigas nos estudien a los humanos, es dar testimonio de algo que es mucho más complejo que nosotros. La ciencia es una articulación. El clima es otra cos que es mucho mas complejo. La empresarialidad es un conocimiento tipo práctico, solo se hace aprende por medio de experiencia.\newline 

La empresarialidad es un conocimiento puramente práctico. \emph{Paréntesis:La empresarialidad es ariesgada y tanto así que hay personas que financian 10 proyectos y con uno que sea exitoso es suficiente para financiar todos los 9 más fracaso.} El conocimiento práctico es científico, lo práctico no es articulable.

\subsection{El conocimiento práctico no es articulable}
\emph{Paréntesis:en la universidad las clases son tóricas porque la practica se aprende fuera de la universidad, en la universidad solo se aprenden conocimiento científico articulable no práctico.}

\textbf{\emph{Definición} : Conocimiento subjetivo y práctico} Subjetivo en el sentido que es relativo a las circunstancias de tiempo y lugar; es subjetivo también por la parte de interpretación de señales, cuando alguien especula subjetivamente, \emph{Paréntesis: Serigo Ramos, quebro "lemon brothers" ya que lo compró especulando ganancia}, el tipo de conocimiento articulable no te capacita en el conocimiento práctico.  \newline 
Si ya sabemos que tenemos un sesgo o scope que nos priva de saber otras circunstancias ajenas a las mias. 
\textbf{\emph{Definición} Conocimiento privativo y disperso:} hay una información y cantidad de conocimiento enorme en cualquier sociedad, cada uno de nosotros solo poseemos una cantidad diminuta unos <<bits>> de este conocimiento. Este conocimiento esta muy disperso entre la sociedad, es subjetivo por la interpretación, es privativo porque cada uno posee información única. Es uno de los problemas de la centralización ya que es disperso el conocimiento. \emph{Paréntesis:La centralización es solo posible si Dios la hace, no si una persona la hace.}
