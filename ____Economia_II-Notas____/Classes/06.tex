\section{Clase de visita: con Nico Nascshki}
\textbf{Sistemas económicos comparados (capitalism: myth and reality):} \newline 
Hay un debate en economía entre capitalismo y socialismo, pero el enfoque es en qué se componen cada una y diferenciar los mitos de los hechos:
\newline Capitalismo
\begin{itemize}
    \item Capitalismo
\begin{itemize}
    \item Propiedad privada de los productos finales o intermedios
    \item Funciona bajo el ``rule of law'':
    \begin{itemize}
        \item Nadie tiene privilegios sobre los demas
        \item Capitalismo no signiifica que el capitalista tiene beneficiiso a coste de los demas
    \end{itemize}
\end{itemize}
    \item socialismo
    \begin{itemize}
        \item El socialismo utópico; Sait-Simon
        \item Strong socialism: El socialismo cientifico
        \begin{itemize}
            \item No hay propiedad privada de los medios de produccio. \emph{\textbf{Paréntesis:}Si no hay propiedad privada no se puede vender ni mucho menos ganar que son requisitos de la función empresarial.} \textbf{\emph{Ejemplo:}} Milton Friedman y su sistema de precios en su casa. \emph{\textbf{Respuesta}:Cuando se habla de asignar recursos están hablando de una asignación de cosas de parte de gente que no sabe}
            \item La economía (como se allcan los recursos) es centralmente planeado.
        \end{itemize}
        \item Weak Socialism: Post calculation debate socialism: Es una version de el free market con una inclinación socialista.
        \begin{itemize}
            \item Maintain private property
            \item Fix income distribution
        \end{itemize}
    \end{itemize}
\end{itemize}

\section{Debate}
El debate de hoy; se piensa que el capitalista es inmoral,produce mas pero es eticamente inferior \newline 
\begin{itemize}
    \item Capitalism is unethical (exploiter)
    \item Capitalism leaves people behind.
\end{itemize}
Note a significant change:
\begin{itemize}
    \item El socialista conoce que se necesita propiedad prvada para producir
    \item Nada es excempto a las criticas
\end{itemize}

% \newline 
\begin{enumerate}
    \item Myth 1: Capitalismo crea pobreza
    \begin{itemize}
        \item Poverty and income distribution are diferent issues
        \item Poverty
        \begin{itemize}
            \item Low level of income
            \item Standard es 1 o 2 dollares al día se considera pobre
        \end{itemize}
        \item Realidad: los datos muestran que el capitalismo de hecho baja los indices de pobreza, desde 1820 ha bajado exhorbitantemente.
        \item África y Asia (continentes con tendencias al socialismo) son los más pobres del mundo
        \item El capitalismo al elevar la población se fue reduciendo el índice de pobreza.
        \item Los paises más libres tienen un ingreso mayor que los socialistas, no se pueden tener buenos ingresos con instituciones socialistas, si eres pobre en una sociedad socialista vas a tener 10 veces menos ingreso que los pobres en las los paises capitalista.
        \item Recordar el ``hockey stick of time'', la humanidad estalló como función exponencial, se dió  como derivada de la revolución industrial.
    \end{itemize}


    \item Myth 2: Income distribution
    \begin{itemize}
        \item GDP es mejor en paises libres, los ingresos están concentrados en unas pocas personas, income distribution worsens
        \item Be carefull with:
        \begin{itemize}
            \item Income distribution (relative income) is ot poverty (absolute income)
            \item Income mobility: Statistics are a photo, reality is a movie
            \item Income matters in terms of consumption, how about ``consumption distribution''? Grandes diferencias en la distribución no es necesariamente diferente
            \item La redistribución de la riqueza da lugar a la pobreza
            \item \textbf{Nos preguntamos:} ¿de donde viene la idea que tener diferencias de distribución es malo? No lo es, \emph{\textbf{Respuesta}:la redistribución de más equidad pero los jode a todos y todos se van para abajo en sus ingresos.}
            \item Income and consumption inequality, los paises mas libres tienden a tenes mas ingresos, los paises menos libres tienden a tener menos ingresos \emph{\textbf{Paréntesis:}tu ingreso es el equivalente sin importar que tan libre sea donde vives es de aproximadamente 2\%}
            \item Siempre se va a tener el caso de distribución del ingreso, se cree que el socialismo y capitalismo y sus efectos son muy diferentes \emph{\textbf{Paréntesis:}Hay que mirar la mayor cantidad de datos posibles}
            \item \textbf{\emph{Ejemplo:}} dos dollares están generalizadas ya que se asume la \textbf{paridad de poder de compras}
        \end{itemize}
    \end{itemize}

    \item Myth 3: Capitalism is exploitative
    \begin{itemize}
        \item Se piensa que en el capitalistas hay un 1\% a el coste de los trabajadores
    \end{itemize}
    \begin{itemize}
        \item An economic agent can have more than one source of income 
        \begin{itemize}
            \item worker may invest in stocks and bonds
            \item An individual can use his capital and become a capitalist and worker
            \item Clases sociales \textbf{NO} son exclusivas
        \end{itemize}
    \end{itemize}
    \begin{itemize}
        \item En el capitalismo se maximiza la producción, ocurre lo opuesto de lo que predecía Marx. Los datos muestran lo contrario.
        \item Cual es el problema con la distribucion de ingresos:
        \begin{itemize}
            \item \textbf{Nos preguntamos:} ¿por qué ser exitoso es malo? Un empresario aporta valor.
            \item Supongamos que mañana amanecemos con la misma distribución del ingreso, \textbf{Nos preguntamos:} ¿cuánto tiempo va a pasar para que tengamos que redistribuirlos de nuevo en el nombre de igualdad?
        \end{itemize}
    \end{itemize}

    \item Myth 4: Socialism is posibles
    \begin{itemize}
        \item Weak socialism is posible 
        \begin{itemize}
            \item Mantiene la eficiencia del capitalismo pero con una ``ética'' distribución del ingreso.
            \item Se dan el ejemplo de los paises nórdicos, pero el éxitos de los paises nórdicos es a raíz del capitalismo, Los paises nordicos eran ricos antes de hacer sus ``reformas socialistas'', siguieron siendo libres a pesar de las ``reformas socialistas'', hay libre mercado, hay libertad
            \item \textbf{\emph{Ejemplo:}} Iceland, su caída de la riqueza por la crisis.
        \end{itemize}
    \end{itemize}

    
    \item Why dows socialism persist?
    \begin{itemize}
        \item Aspectos importantes:
        \begin{itemize}
            \item Rational ignorance: los costos de información es muy alto para que todos se pongan a aprender por qué están mal, estadisticamente tu voto no vale nada.
            \item Rational irrationaity: aquel que no quiere ser visto como aquel que es malo, es aparentar, aquellos que no les importen que tan malos esten sus argumentos pero hace cosas en nombre de sus argumentos en son de verse mejor en la sociedad en la que vive.\textbf{\emph{Ejemplo:}} tú profesor de economía ¿hay que subir el salario mínimo? es una pregunta psicológica mas que economica \emph{\textbf{Respuesta}:no quieres que estés peor por tomar la medicina equivocada.}
            \item Generation turnover: los nuevos votantes no tienen idea de su historia.
            \item Capitalism as social classes rather than institutonal framework: en los lentes del socialismos si ``hay dueños de capital, hay clases'', esto no es cierto por el rule of law. \newline  El ``crony capitalism $\neq$ capitalism''.
        \end{itemize}
    \end{itemize}


\end{enumerate}
