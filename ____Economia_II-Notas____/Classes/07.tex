\section{Laboratorio}
\begin{itemize}
    \item Llegué tarde 20 minutos.
    \item El marginalismo, es valorar basado en la ultima unidad que tenemos
    \item El salario mínimo, en México el salario mínimo es la mitad del de Guatemala.
    \begin{itemize}
        \item Supongamos que el salario producido a precio de mercado es Q1,500, y ponemos el precio tope de Q3,000, 
        \item Escazes es que hay más empresas y pocas personas para trabajar, hay escazes para cosas muy especializadas
        \item Una sobre oferta: \textbf{\emph{Definición: hay poca demanda}} \emph{\textbf{(Paréntesis:}Sin hablar de las opotunidades de estudio de las personas de salario mínimo no se intercambian 2 millones.)}
        % \begin{figure}[htbp]
        %     \centering
        %     %\includegraphics[width=6cm,angle=90]{2019-08-09-1}
        %     \caption{}
        %     \label{}
        % \end{figure} 
    \end{itemize}
    \item Fondo mutuo: \textbf{\emph{Definición: cuando un grupo de personas se ponen de acuerdo para meter dinero en una sola cuenta para acumular mayor cantidad de intereses ya que es mayor cantidad de dinero.}}
\end{itemize}
