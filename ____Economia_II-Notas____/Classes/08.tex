\section{Resolución del corto}
\begin{enumerate}
    \item Estamos mejor, la gente piensa que estamos peor por la pobreza pero desde la revolución industrial hemos estado exponencialmente mejor
    \item Las diferencias de ingresos y de consumo, a pesar que los ingresos pueden ser abismales Bll Gates no puede comer 10 veces, consume lo mismo que yo en cierta manera.
    \item Si te equivocas especulando, caso1 te quedas igual, caso2 se registra pérdida, caso3 la pérdida es mayor ya que el precio a lo mejor subió y entonces se perdió el potencial. \emph{\textbf{(Paréntesis:}Charla de el ministro de economía, \textbf{Nos preguntamos:} ¿qué pasa si el quetzal se deprecia?)}
    \item El salario mínimo produce una sobre oferta y un excedente en la oferta de trabajo.
\end{enumerate}

\section{Noticia: El dinero falso de Facebook}
\begin{itemize}
    \item Libra: Criptomoneda
    \item Se procura pagar a través de libras no divisas comúnes, hay muchas empresas afiliadas, cuenta con una reserva física.
    \item Lenguaje de programación Mu.
    \item Problema, las transacciones son muy costosas
    \item Beneficios, facilidad de transporte, facilidad de ser moneda global, facilidad en pagos internacionales, ampliar el acceso a servicios financieros.
    \item \textbf{Nos preguntamos:} ¿Que puede hacer que no funcione la moneda? \emph{(\textbf{Respuesta}:Confianza, estereotipar con bitcoin)}.
    \item Conclusión: Se crea una moneda semi fiduciaria.
\end{itemize}

% \textbf{\emph{El problema es este: $0}}
% \textbf{\emph{Caso ``$1\'': $0}}


\section{Discusión de clase}
\begin{itemize}
    \item El problema de las monedas fiduciarias se presta a la incertidumbre de los usuarios entonces se dan fluctuaciones ya que la gente tiende a adquirir o renuncar a la confianza, pero la confianza baja porque las personas saben que conforma la demanda de libras se aumenta cada libra vale menos.
    \item \textbf{Nos preguntamos:} ¿Porque los países no les gustan la inflación pero les gusta imprimir dinero? \emph{(\textbf{Respuesta}:Para no incrementar los impuestos los países imprimen dinero para aumentar sus ingresos sin incrementar impuestos}).
    \item La diferencias entre ingresos fiscales y egreso fiscales es (33.33) REVISAR AUDIO 
    \item Último comentario: FB tiene dos formas de incrementar la confianza, 1 es decir que no lo va a hacer, es casi obligatorio respaldar la libra con una moneda que ya existe que no es fiduciaria. 
\end{itemize}

\section{Diferencias entre riesgo e incertidumbre}
\begin{itemize}
    \item Ambos conceptos refieren a eventos futuros
    \item Como el futuro no lo podemos conocer, pero podemos especular, podemos establecer diferentes probabilidades de ocurrencia
    \item Eventos parametrizable: que se le puede poner un parámetro, se le puede asignar una distribución de probabilidad. Teorema de Bayes, se pueden conocer las probabilidades de algo suceder a la proporcional a la cantidad de probables eventos.
    \item Eventos no parametrizables: no tienen ocurrencias, no se observa un patrón, \textbf{\emph{(Ejemplo: partido madrid y barcelona, ¿la probabilidad cuál?)}} intentar calcular probabilidades de eventos únicos no obtienen datos útiles.
    \item La diferencia es que si se conoce la distribución de probabilidad, parametrizabilidad o no parametrizabilidad. \textbf{\emph{(Ejemplo: Casino, uno gana en un casino evaluando probabilidades, es teoría de probabilidad.)}}.
    \item Si hay dos tipos de probabilidad, una donde se conoce el parámetro, \emph{\textbf{(Paréntesis:}El riesgo es parametrizable y la incertidumbre es no parametrizable)}, consolidar riesgos es parametrizables y distribuirlos.
    \item Consolidación y distribución:\textbf{\emph{(Ejemplo: Basado en genero y grupo étnico, da cáncer, una persona en 10,000 les dan cáncer, entonces hago un fondo común y las 10,000 personas diluyen el riesgo, y la persona que desafortunadamente le de está cubrida)}}
    \item Especialización: transmitir el riesgo a especialista (típica de incertidumbre): se puede calcular la probabilidad no como Bayes pero con especialización pero se puede, la forma de reducir la incertidumbre es el experto.
    \item Las aseguradoras son los especializados afrentan al riesgo, el empresario se afrenta la incertidumbre es de conocimiento tipo B.
    \item \emph{\textbf{(Paréntesis:}La aseguradora en si es una empresa, la incertidumbre que afronta la aseguradora son las catástrofes tipo terremotos, por eso que nosotros firmamos contratos, diseñados para reducir incertidumbre; la aseguradora debe saber qué eventos no asegurar, si la propia aseguradora asegura cosas que incentivan la actividad la cual se intenta asegurar la aseguradora pierde \textbf{\emph{(Ejemplo: cuando las personas matan a los familiares por el dinero de la aseguradora)}}.)}
    \item Los beneficios: son el único valor productivo que tiene una empresa, depende de la especulación, \textbf{\emph{Definición: los beneficios son el pago por hacer frente a la incertidumbre}}
\end{itemize}

\section{Lectura}
Leer la lectura del mercado laboral.
