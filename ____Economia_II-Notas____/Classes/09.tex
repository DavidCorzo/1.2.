\section{Desequilibrios fiscales, monetarios y cambiario en Argentina: Una historia de alta inflación - Adrián Ravier}
\emph{\textbf{(Paréntesis:}Los economistas pueden ser muchas cosas, principalmente pueden predecir macro a varibles acerca de qué va a pasar en el futuro económico)} \newline 
\textbf{La herencia en números:}
\begin{itemize}
    \item Falta de números en INDEC: 2015, INDEC mentía con todos los datos, eso implicó que se necesito hacer de nuevo esos números.
    \item Actividad económica no crecía \emph{\textbf{(Paréntesis:}Guatemala crece 3\%)}, Argentina no crecía.
    \item Desequilibrios fiscales
    \begin{itemize}
        \item Sobre empleo público de 2,3 a 4,2 millones de empleados públicos
        \item Excesivo gasto público (consolidado). De 26,2 a 43 \% del PIB.
        \item Déficit discal (con intereses): 7.66\% del PIB.
        \item Subsidios, tarifas atrasadas.
        \item Déficit en infraestructura:
    \end{itemize}
    \item Desequilibrios monetarios
    \begin{itemize}
        \item Alta inflación (Expansión de MO en 2015 superior al 40 \%)
        \item Controles de precios e inflación reimprimida.
    \end{itemize}
    \item Desequilibrios cambiario
    \begin{itemize}
        \item Cepo cambiario
        \item Apreciación de futuros 
        \item Venta de futuros
        \item Política exterior (aislamiento)
    \end{itemize}
\end{itemize}

\subsection{Krishnerismo}
El Krishnerismo hizo: el efecto serrucho, la economía no arranca ya que por los abismales números de empleados públicos, los empresarios tienen una labor muy difícil por la cantidad de empleados públicos y por la fluctuación de precios.
\begin{itemize}
    \item La gente sufrió una depreciación del peso argentino y todos se empobrecieron.
    \item Todo esto por la presión social que los gobiernos querían hacer un montón de cosas gratis, como dar la luz gratis, fútbol.
    \item Un futbolista en argentina que destaque se va porque argentina no puede pagarle como los demás países.
    \item La gran divergencia de argentina con respecto a otros países de américa latina es cada vez mas grande.
    \item Balanza de pagos, Argentina gasta más dólares que pesos argentinos.
    \item En 2018 paso una recesión por la sequía, como la cosecha no fue buena, no se pudieron adquirir dólares, fueron a el fondo monetario, subió el precio para compatibilizar la oferta con la demanda. Argentina cayó en escalón, se genero la crisis.
    \item \textbf{Nos preguntamos:} ¿Cómo paró la crisis? Se cambio a los del banco central, subió el precio de la luz, del agua, del gas.
    \item En el 2019 Argentina se empieza a recuperar, las proyecciones del gobierno eran buenas para 2020.  
    \item El gobierno pudo cortar el déficit a 0.00, los ingresos del gobierno venían con altos impuestos a la agricultura etc.
    \item Las cuentas se estaban equilibrado
    \item La inflación es la inflación monetaria, todas las demás causas no. 
    \item LEBAC la gente estaba comprando LEBACs por que tenia un interes de 40\%
    \item LELIC la gente compra LELIC; todas las personas empezaron a comprar LELICs y con eso se financió la crisis de 2018
\end{itemize}

\[
  \frac{\text{Base Monetaria}}{\text{Reservas Netas}} = \frac{2 \text{BILLONES}}{77\text{MILLONES}} \text{USD} 
\]

\begin{itemize}
    \item Se desea dolarizar pero dolarizar implicaría que todas las personas jubiladas se vuelvan pobres.
    \item El problema: no se quieren endeudar para obtener un tipo de cambio que no es del mercado, dolarizar implica que los salarios estén muy bajos y se pretendía dolarizar manteniendo los salarios mas o menos normales, entonces es o endeudarse dolarizando o dolarizar y empobrecer a la gente.
    \item La inflación iba bajando
    \item La bolsa saltó 7\% en Argentina, todos querían comprar acciones, pierde el gobierno especulado, entonces se devaluó de 45\% a 60\%.
    \item Guillermo Calvo, Sudden Stop. 
    \item Guatemala tiene un banco central independiente, el gobierno en Argentina no tiene límite el banco central. El presidente electo en argentina elige quién es el presidente de el banco central; algo que se discute en Guatemala es dependizar el banco central en guatemala con el gobierno.
    \item El economista es como un piloto con indicadores, ellos de acuerdo con qué toquen lleva al destino que se quiere. \emph{\textbf{(Paréntesis:}En realidad la economía es mucho más compleja que un avión.)} \emph{Citación: Milton Friedman ``imprimí tanto dinero conforme a la demanda de dinero, imprimir el dinero conforme la demanda de dinero que es muy difícil"}
\end{itemize}

\[
  \underbrace{M}_{3\%}*V = \bar{p} \underbrace{y}_{3\%}
\]

\begin{itemize}
    \item La política monetaria se necesita por que impacta al país, se toman malas decisiones por no saber políticas monetarias, si no se tiene un claro concepto no se pueden tener importantes decisiones.
    \item Estos temas no los pueden pasar.
\end{itemize}
