\section{Resolución del corto}
\begin{itemize}
    \item Parametrizable y no parametrizable, parametrizable es lo que se puede medir con un margen de error relativamente pequeño, no parametrizable es lo que no es medible de ninguna manera.
    \item Las aseguradoras se enfrentan a la incertidumbre de lo que no puede parametrizar.
    \item El gobierno en Argentina el gobierno en cierta manera controla la banca central. Un incentivo a gastar y un incentivo a no ingresar, se aumenta la deuda pública. Después toca ajustarse, y nadie quiere subir impuestos entonces se imprime dinero y se genera inflación. \emph{\textbf{(Paréntesis:}Prohibición constitucional que regula que se reuna el gobierno con el director de la banca central, eso es lo que está prohibido en Guatemala, en Argentina sí)}
    \item El pago por enfrentar la incertidumbre son los beneficios que derivan.
\end{itemize}

\section{Noticia de insulina genérica}
\begin{itemize}
    \item La insulina está muy cara
    \item Causas de los altos costos son por:
    \begin{itemize}
        \item Tecnología: avances tecnológicos
        \item Intermediarios: aseguradoras con precios especiales, artificialmente están causando inflación sobre su producto, pero la inflación es artificial. La demanda es inelástica.
        \item FDA: Crea un control de precios, colisionan entre empresas
    \end{itemize}
    
    \item Soluciones del FDA sugiere que se retiren barreras de entrada, reducir sus estándares y recomendar a los ciudadanos.
    \item En Canadá se usa insulina animal por ende es más barata, en EEUU se usa de la bacteria, por ende es más cara, pero es la única que acepta el FDA.
\end{itemize}

\section{Discusión de la clase}
\begin{itemize}
    \item Hay muchas críticas hacia el FDA, eleva las barreras de entrada puestas por el FDA que comprende una situación que solo las empresas más grandes pueden participar 
    \item \emph{Citación:``Ojos que no ven, corazón que no siente, las miles de personas que se mueren por drogas que no están aprobadas por el FDA, mientras si se mueren por una droga aprobada por el FDA se les viene todos encima a la FDA, entonces el incentivo es a no aprobar"}.
    \item Los países que más gastan en sanidad es primero EEUU y en cuba como segundo lugar.
    \item En Singapur se da un muy buen uso de la medicina para la sanidad del país.
\end{itemize}


\section{Salarios}
\begin{itemize}
    \item \textbf{Nos preguntamos:} ¿Cuál es la diferencia entre un no empleado y un desempleado? \textbf{Nos preguntamos:} ¿Qué es la taza de actividad? \emph{(\textbf{Respuesta}:Taza de actividad es la población que está en edad y capacidad de trabajar (15,64), \textbf{Estos PUEDEN TRABAJAR}}) \textbf{Nos preguntamos:} ¿Taza de dependencia? \emph{(\textbf{Respuesta}:Es el porcentaje de la población que son dependientes, no trabajan o son ancianos y niños que no se encuentran en edad y/o en capacidad de trabajar.})
    
    \item \textbf{Nos preguntamos:} ¿Taza de desempleo? \emph{(\textbf{Respuesta}:Personas capacitadas que pueden trabajar, que tienen el deseo de trabajar que no pueden adquirir trabajo})
    
    \item La taza de desempleo en Guatemala se considera bajo si se consideran los trabajos informales. \textbf{\emph{Definición: Trabajos informales, son aquellos trabajos que operan bajo el salario mínimo sin apoyo legal, sin bonos, que es considerado subempleo}}
    
    \item \textbf{\emph{Definición: No empleados, incluye a desempleados e incluye a niños y ancianos mas los incapacitados.}} 
    
    \item En Guatemala tiene bajo desempleo y altísima informalidad. El PIB informal y PIB formal en Guatemala son casi mitad mitad.
    
    \item \textbf{\emph{Definición: PIB: Producto Interno Bruto por el valor total de bienes y productos finales.}}
    
    \item \textbf{Nos preguntamos:} ¿Cómo se mide el PIB informal? \emph{(\textbf{Respuesta}:El PIB se mide por medio de encuestas, encuestas de gastos, encuestas de ingresos y la última es la encuesta del \textbf{el valor añadido} el PIB debería de ser las conclusiones derivadas de la encuesta del valor añadido, De esta se puede calcular el PIB informal estimado.})
    
    \item \textbf{\emph{Definición: Salario: }\textbf{Nos preguntamos:} ¿Por qué hay desempleados? \emph{(\textbf{Respuesta}:})} \emph{\textbf{(Paréntesis:}Los factores de producción son tierra y trabajo\textbf{)}} 
    \item \textbf{\emph{Definición: Salario de reserva: la persona tiene un coste de oportunidad diferente al del trabajo, el trabajo es escaso y es útil}} \textbf{Nos preguntamos:} ¿El precio que se paga por el trabajo? \emph{(\textbf{Respuesta}:Ese precio que se adquiere por trabajo se llama \textbf{Salario}, en el trabajo el oferente determina subjetivamente cuánto quiere recibir basado en qué le guste más subjetivamente, ¿qué desutilidad? })
    
    \item Para Daniel es más util dar clase que trabajar en una mina. 
    
    \item \emph{Citación:``Dedicate a lo que te gusta"} a lo que se refiere es que tu trabajo no te genere desutilidad.
    
    \item \textbf{\emph{Definición: Desutilidad del trabajo se refiere a qué tanto es el nivel de molestia que incurres al ir a trabajar.}}
    
    \item \textbf{Nos preguntamos:} ¿Se puede comprar y vender trabajo? \emph{(\textbf{Respuesta}:Se \textbf{RENTA} no se compra, comprar sería esclavitud.})
    
    \item Para el que demanda trabajo busca poder adquirir bienes y servicios mayores a la utilidad que genera no trajar, lo mismo con los empleadores dan trabajo buscando mayor utilidad dando trabajo que no dando.
    
    \item \textbf{Nos preguntamos:} ¿Hay un salario relativo o salario de equilibrio? \emph{(\textbf{Respuesta}:Depende de la categoria profesional, ya que no se puede hacer arbitraje de una manera fácil (un arquitecto no se puede convertir en dministrador sin 4 o 5 años de por medio)})
    
    \item \textbf{\emph{(Ejemplo: el título de la Marro, dan una idea al empleador de cuál es la productividad, y los empleadores adivinan tu productividad.)}}
    
    \item La clave para volverte un empleado sobre pagado es fingir más productividad que la que tu gerente piensa que tiene.
\end{itemize}

% mercado laboral
