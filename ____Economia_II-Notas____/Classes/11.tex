\section{Noticia de Dunkin Doughnuts}
\begin{itemize}
    \item Cambio de las acciones de Dunkin Brands 
    \item \textbf{Nos preguntamos:} ¿Por qué gana Dunkin por publicidad, usualmente se paga o gasta en publicidad? Prácticas contables que categorizan publicidad en la apertura de nuevas franquicias
    \item \textbf{Nos preguntamos:} ¿es bueno tener un price earnings alto? Es mejor el bajo, cuánto tiempo tardo si el PE está a 10, imaginamos que es así, ¿cuánto tiempo pasa para recuperara mi inversión? en 29 años recuperarás la inversión por tener el price earnings alto implica mayor espera. \emph{Citación:``Es mejor un Price Earning bajo"} 
    \item PD: Price Sales es la rotación, \textbf{Nos preguntamos:} ¿quién vende más al día walmart o una joyería? \emph{(\textbf{Respuesta}:El price sales es alto en walmart mientras la joyería tiene un PS mas bajo}) 
\end{itemize}

\section{Discusión de clase}
\begin{itemize}
    \item Price earnings es lo primero que ven los inversionistas, \textbf{\emph{(Ejemplo: Amazon vale como 1000, y Apple 100, ¿es mejor amason que apple? poniendolo en términos relativos del tamaño de la empresa un dividendo vale de acuerdo al número de acciones.)}}
    \item \emph{\textbf{(Paréntesis:}en términos relativos el price earnings es:\textbf{)}}
\end{itemize}
\[
    Price Earnings = \frac{\text{PRECIO ACCIÓN}}{\text{BENEFICIO POR ACCIÓN}} 
\]
\begin{itemize}
    \item  El price earnings no nos dice nada del futuro.
    \item Si uno ve el price earning como indicador puede subir pero no nos dice nada del futuro, es una medida de sobre valoración general de la bolsa, es una medida de sobre valoración. Esta es la medida que usan los inversionistas del mundo.
\end{itemize}

\begin{itemize}
    \item \textbf{\emph{Definición: el salario de reserva es el coste de oportunidad de trabajar}}
    \item \emph{\textbf{(Paréntesis:}No hay precio de equilibrio en los salarios ya que es muy subjetivo el coste de oportunidad\textbf{)}}
    \item Los inversionistas mediante el sistema de precios es el price earnings, cuando el price earnings sube todos compran y por ende todo sube
    \item \textbf{Nos preguntamos:} ¿por qué estudias en la marroquín? \emph{(\textbf{Respuesta}:por que provee al empleador una mejor estimación de tu productividad})
\end{itemize}

\begin{tabular}{ |c|c|c| } 
    Ratio de salarios = &  $\frac{\text{SALARIO}}{\text{PRODUCTIVIDAD (EN TÉRMINOS ECONÓMICOS)}}$ \\
\end{tabular}
  


\begin{itemize}
    \item Se busca gente con salario de reserva y con la productividad muy alta, es una ganga la que buscan.
    \item La competencia hace que suban los salarios ya que los salarios suben y la productividad se empieza a apreciar, por ende se sube el salario de los trabajadores.
    \item Cuando el PE es alto nadie quiere, y cuando es bajo es más atractivo.
    \item Los trabajadores están buscando salarios más altos y los inversionistas más productividad y menos salarios.
    \item Una posición de trabajo si lo vales puedes amenazar que te irás por ende poniendo en cuestión la productividad que el empleador recibe como capital.
    \item El factor trabajo es movible solo de una forma imperfecta, con movible se refiere  35:49 AUDIO
    \item La automatización produce lo que es atractivo en los ojos de empleador, un tractor sustituye a los macheteros, y la gente mayor es la más afectada, este cambio a corto plazo es jodido.
    \item En la primera revolución industrial (segunda mitad del siglo XVIII): Habían movimientos que quemaban las máquinas, sustituyeron el tipo de trabajo por uno más productivo.
    \item La automatización no genera problemas a largo plazo, pero sí genera problemas a corto plazo por el factor trabajo imperfecto.
    \item El coste de oportunidad de el cambio es este caos a corto plazo.
    \item El país que opone a estos cambios usualmente son los más pobres.
    \item En GT funciona muy bien Uber y AirB\&B.
    \item El trabajo puede ser sustituido por otros factores como capital, y vice versa, 42.18 AUDIO
    \item Cuando baja el tipo de interés se sustituye por el capital, \textbf{\emph{Caso ``Azúcar en GT": Se fueron sustituyendo a la gente que no tenía la productividad para pagarle el salario mínimo por un tractor y automatización. Se cambia el tipo de trabajo y se empieza la especialización en por ejemplo manejar tractor, componer tractores, hacer tractores. El piloto del tractor aporta más productividad que los macheteros}}
    \item \textbf{\emph{Caso ``India y sus maquilas": por más que la gente teja a mano no van a superar a las máquinas que se utilizan para hacer ropa}}
    \item \textbf{\emph{Caso ``McDonald's": sube el salario mínimo a 15\$ entonces se automatiza y se despiden a las personas que atendian por falta de productividad y salarios muy altos, ahora se produce una nueva deanda por mantenimiento de esas maquinas o por la manufactura de esas maquinas}}
    \item Desigualdad de la negociación, \textbf{Nos preguntamos:} ¿no es cierto que entonces los empresarios tienen poder monopólico con los salarios? \emph{(\textbf{Respuesta}:no, cada uno tiene sus costes de oportunidad, \textbf{\emph{(Ejemplo: si a mi me pagan 5 Q por chapear un campo, no aceptaría ese salario, si me pagan 1,000,000Q por chapear pospuesto que lo hago, esto es lo que hace el salario mínimo)}}}) La clave es, cuando hay mucha competencia el poder de negociación del empresario se disminuye y aumenta el de el trabajadors, cuando no hay competencia se aumenta el poder de negociación del empresario y disminuye el del trabajador.
    \item GEM: \textbf{\emph{Definición: Global Entreprenouship Monitor}}, mide la informalidad en GT.
    \item El salario que puede pagar un empresario varia entre dos límites:
    \begin{enumerate}
        \item El superior es el precio descontado por el tipo de interés de los bienes que el trabajador realiza en la empresa actual. (Lo que puede producir un trabajador)\emph{\textbf{(Paréntesis:}A veces los conceptos de desempleo son capciosos \textbf{\emph{(Ejemplo: si tu salario de reserva es 1,000)}}\textbf{)}}  
        \item EL límite inferior: precio descontado por el tipo de interés de los bienes que el trabajador podría realizar en otra empresa. (asumiendo que no hay oportunidades de arbitraje, hasta cuanto te puede bajar el salario el empleador recae en el límite inferior)
    \end{enumerate} 
    
    \item Estos son los límites se pueden dar la vuelta por que está subiendo por equilibrio, como el equilibrio no existe.
    \item El empresario puede pagarte en un intervalo que oscila entre el límite superior e inferior.
    \item Por eso hay que estar abierto a viajar.
\end{itemize}

\begin{itemize}
    \item El salario mínimo, provoca entonces basado en este principio nunca se obtiene trabajo por que no se llega a la productividad mínima, por eso es óptimo que te paguen de acuerdo a tu productividad aun que esté debajo de el salario mínimo.
    \item La informalidad en la capital según los últimos años es de 50\%, en las verapaces la informalidad es de 90\%
    \item El salario mínimo impacta a aquel que es menos productivo, es precisamente a los que quieren ayudar.
    \item El salario mínimo diferenciado: diferenciar por región y profesión, produce que más gente no lleguen a los requicitos mínimos. El payaso de Maldonado quitó los salarios diferenciados. 
    \item Que haya informalidad depende del poder o totalitariedad del estado. El empresario va a tender a contratar muy pocas personas por el salario mínimo.
    \item Es contradictorio por que es precisamente lo que mantiene al pobre pobre.
    \item La media jornada, estupidez
\end{itemize}
