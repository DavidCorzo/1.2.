\section{Resolución de corto}
\begin{itemize}
    \item ¿Es posible que el desempleo llegue a 0\%? No, por que algunas personas tienen salarios de reserva muy altos.
    \item ¿qué es la curva de Kuzner? Es la destrucción ambiental respecto a el desarrollo económico, ya la gente que vive en un país desarrollado tiende a demandar más verde.
    \item ¿Bien económico y un bien, diferencias? El bien es solo algo que satisface necesidades existentes, el bien económico es escaso.
\end{itemize}

\section{Noticia}
\begin{itemize}
    \item FedEx retira a Amazon.
    \item Debido a la competitividad de los mismos.
    \item Función empresarial, 
    \begin{itemize}
        \item Al servicio de los consumidores, los empresarios buscan beneficios, los empresarios deben estar buscando nuevos medios
    \end{itemize}
\end{itemize}

\section{Discusión de clase}
\textbf{Parical 11 de septiembre}

\begin{itemize}
    \item Características de los bienes:
    \begin{enumerate}
        \item Necesidad humana
        \item Que el bien tenga cualidades para satisfacer esa necesidad.
        \item Conocimiento del uso del bien, a veces se necesita cierta tecnología para utilizarla.
        \item Poder de disposición sobre la cosa, \textbf{\emph{(Ejemplo: por ejemplo el petroleo en GT, pero supongamos que está a 100,000 km de profundidad, no se puede.)}}
    \end{enumerate}

    
    \item \textbf{Nos preguntamos:} ¿que ocurre cuando algo no tiene cualidades para satisfacer una necesidad sin embargo creemos que sí la tiene? Como satisface las necesidades subjetivas las personas, es un bien. \textbf{\emph{(Ejemplo: Fidget spinners, pulsera del balance)}}; Menger les llama bienes imaginarios, son imaginarios por que no tienen la capacidad real de satisfacer las necesidades, \emph{\textbf{(Paréntesis ``Las ciudades pobres son místicas'':}si quiero que llueva me pongo a hacer un rito o alguna mierda así.\textbf{)}}; \textbf{\emph{(Ejemplo: el vegetarianismo como bien imaginario)}}; \textbf{\emph{(Ejemplo: La homneopatía)}}

    \item \textbf{Nos preguntamos:} ¿por qué hay homneopatas o chamanes? por que la gente es tonta y cree en ellos, por ende hay una demanda por ese bien. \textbf{\emph{(Ejemplo: Steve Jobs)}}

    \item \textbf{Las cualidades del bien es algo que no existe en el universo, solo existe en nuestra mente.} es un plano simbólico (algo que no es real pero no es real en nuestras mentes).
    \item Bienes de consumo y de capital:
    \begin{itemize}
        \item Bien de consumo son bienes que se consumen. Los bienes de primer orden satisfacen las necesidades de manera inmediata.
        \item Bien de capital, son por ejemplo la máquina. Los bienes capitales son de orden superior y son capaces de satisfacer las necesidades mediatamente, lleva consigo una transformación para llegar a ser un bien de primer orden.
        \item \textbf{Nos preguntamos:} ¿diferencias de panadol y quimioterapia? \emph{(\textbf{Respuesta}:tiene que ver con la marginalidad, basado en la última unidad disponible, y con subjetivismo})
        \item Ejemplo:
        \begin{etaremune}
            \item Electricidad
            \item Horno - Trigo
            \item Masa pan - Agua - Trabajadores o panaderos
            \item Pan 
        \end{etaremune}
        \emph{\textbf{(Paréntesis ``El PIB'':}mide estos bienes de orden superior y primer orden para determinar tu calidad de vida.\textbf{)}}
        
        
        \item \textbf{\emph{(Ejemplo: la parsona que le gusta aprender la educación es un bien de consumo, para otros la educación es un bien superior para satisfacer otros bienes subjetivamente más importantes para ellos)}}

        
        \item \textbf{\emph{(Ejemplo: La guerra de sessesión: problemática con los esclavos, los estados del norte hicieron una intervención con el sur e inglaterra tampoco mediante tratados diplomáticos podían comercializar. Lo que ocrurrió entonces es que todos los textiles y la especialización derivada de la producción de textiles no vale nada a partir de eso.)}}
        
        
        \item \emph{\textbf{Definición de ``Bienes específicos":} son bienes que sirven para procesos específicos de acuerdo a los posibles usos que se le pueden dar.}
    \end{itemize}
\end{itemize}
