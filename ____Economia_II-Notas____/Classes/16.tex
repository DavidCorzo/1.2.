\section{Noticia de aguas y proyecto}
\begin{itemize}
    \item Proyecto Bill Gates 
    \item Proyecto de purificación de agua 
\end{itemize}


\section{Discusión de clase}
\begin{itemize}
    \item \emph{\textbf{(Paréntesis ``La idea de una máquina que produzca energía infinita'':}Hasta ahora no existe.\textbf{)}}
\end{itemize}

\begin{tabular}{ | p{5cm} | p{5cm} | }
    \hline
     Compradores & Vendedores      \\
     100 Bienes presentes & 100 Bienes presentes \\ 
    \hline
       A1-300 & B1-99 \\ 
       A2-200 & B2-100 \\ 
       A3-150 & B3-101 \\ 
       A4-120 & B4-102 \\ 
       A5-110 & B5-103 \\ 
       A6-108 & B6-105 \\ 
       A7-107 & B7-106 \\ 
       A8-106 & B8-107 \\ 
       A9-104 & B9-108 \\ 
       A10-102 & B10-110 \\ 
   \hline
       Demandante de tiempo & Oferentes de tiempo \\ 
   \hline
\end{tabular}


\begin{itemize}
    \item En la tabla anteriormente presentadas:
    \begin{itemize}
        \item \textbf{\emph{(Ejemplo:}}\emph{ Ejemplo alternativo: Estas en un bar, estas tomando algo y un amigo está hablando con una amiga muy guapa, cuando estoy \textbf{picado} tengo preferencia temporal \textbf{alta}, mi amigo tiene que pagar su cerveza y la de la amiga, el incentivo personal es 2 - 1 \& 4 - 1, mi amigo me pide pagar dos en el futuro por el dinero para pagar una cerveza hoy; el amigo se endeuda en cervezas, el tipo de 4 - 1 es muchísimo más impaciente porque pagará  tres cervezas más en el futuro por una hoy. Dependiendo de la circunstancia las personas van a demandan más o menos bienes presentes por bienes futuros.)}
        \item \textbf{Nos preguntamos:} ¿qué pasa cuando una economía tiene gente muy impaciente? El interés sube ya que más demanda por bienes presentes y mayor disposición de pago en bienes futuros. Los países pobres tienden a ser muy impacientes, ¿por qué? \textbf{no pueden ahorrar}. En el primer mundo es fácil ahorrar, en el tercer mundo implica pasar hambre.
        
        \item \textbf{Nos preguntamos:} ¿Una economía de mucho emprendimiento tiene un interés alto o bajo? El interés tiende a ser elevado ya que mayor gente demanda bienes presentes para invertir hoy y pagar después. En las recesiónes económicas el interés tiende a caer.

        
        \item Mientras más arriba estoy más ahorro
        \item El banco central tiende a tener mucho miedo a que suba el interés y mucho incentivo a querer bajarlo. El banco es solo un intermediario entre los oferentes o demandantes. ¿Por qué? por que estimula la economía para que personas lleven a cabo sus proyectos productivos, por consiquientes se hacen rentables, por consiguiente se provoca un boom económico, los proyectos de bajo valor agregado empiezan a surgir y entonces todo esto produce.
        \item En los interéses bajos se produce más demanda y menos oferta, al claramente estar en \textbf{desquilibrio}, el banco compensa este diferencial con la creación de dinero, por ende el que gana es el banco por la inflación producida retardada y el interés ``bajo'' se queda igual. 
        \item En cuanto se baja el interés se estimula la economía y después para compensar inflan la moneda.
        \item Estimula la economía a coste de la ilusión de interéses bajos, el retardo fluctúa entre 18-24 meses.
        \item \emph{Citación:``La inflación es un impuesto oculto (por esto, creyendo que el interés es más bajo invierten pero tras tomar en cuenta de la inflación es un engaño)"} 
    \end{itemize}
\end{itemize}
