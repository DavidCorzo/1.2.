\section{Resolución del corto}
\begin{enumerate}
    \item Solo se toma en cuenta los bienes finales para no contabilizar dos veces el mismo bien.
    \item Se calcula la infomalidad con las diferentes fórmulas de calcular el PIB.
    \item Problemas del uso del PIB: no toma en cuenta el auto-consumo, auto-producción, actividades domésticas.
    \item Por qué el PIB es engañoso en las guerras:
    \begin{itemize}
        \item Aumenta la producción de bienes destinados a destruir riquezas.
        \item \emph{Citación:``En una guerra gana el que más capacidad tenga de producir"}
        \item El PIB aumenta pero no genera riqueza, usualmente se usa el PIB para medir riqueza.
    \end{itemize}
\end{enumerate}

\section{Noticia}
\textbf{Nos preguntamos:} ¿Los robots remplazarán a las personas trabajando?
\begin{enumerate}
\item \emph{\textbf{Definición de ``Producción":} Cantidad de output respecto a la cantidad de input}
\item Se ven ventajas al trabajar con robots, la automatización aumenta la producción.
\item Premisa a que el mercado va a hacer albitraje al estudiar cómo reparar, hacer, etcétera los robots de automatización, se va a especializar en diferentes capacidades.
\item \textbf{Nos preguntamos:} ¿La riqueza solo se concentrará en una pequeña parte de la población capacitada en automatización? A corto plazo sí, pero a largo plazo no.
\end{enumerate}

\section{Discusión de clase}
\begin{enumerate}
    \item Relacionando a el tema de la noticia: \textbf{Nos preguntamos:} ¿dónde se analfebetizó de primero? Francia e Inglaterra, por que su mano de obra era especializada en educación privada. Cuando llegó la necesidad de más mano de obra especializada lo primero que se invirtió en fue en educación.
    \item \emph{\textbf{(Paréntesis ``Censo'':}en GT se pretendía que tenía 17M pero son 15M\textbf{)}}
    \item Ludismo, analfabetismo.
    \item La automatización es una problemática similar a la que pasó con la revolución industrial por ende la automatización no pronostica que la automatización vaya a devastar todo.
    \item La revolución industrial mejoró la calidad de vida, \emph{\textbf{Ejemplo:}las condiciones de el campo eran de lunes a domingo de sol a sol, te enfermabas te morías}, los que oponen el trabajo infantil son los \textbf{terratenientes}, relacionar con régimen feudal.
    \item Los terratenientes celan a los industriales por que la gente que trabajaba en el terreno del terrateniente ve mejores condiciones trabajando para los industriales, por ende los terratenientes oponen las leyes en contra de el trabajo infantil.
\end{enumerate}
\section{\textbf{Nos preguntamos:} ¿Cómo crece el PIB?}
\begin{enumerate}
    \item Recursos naturales: 
    \begin{itemize}
        \item \emph{Citación:``con los recursos que tiene GT cómo hay todavía pobreza"}.
        \item Los recursos naturales en GT son del gobierno, los \textbf{recursos} del subsuelo no son tuyos, los ríos tampoco, los lagos tampoco, ninguno.
        \item Si se quiere tener una hidroeléctrica el recurso del flujo de agua es gobierno.
        \item \emph{\textbf{Ejemplo:}Venezuela, mayores reservas de petroleo del mundo}, \emph{\textbf{Ejemplo:}África, más pobre de del mundo y es uno de los países más ricos del mundo según sus recursos naturales.}, \emph{\textbf{Ejemplo:}A lado de Congo está Vusbana, de los países más ricos a pesar de sus vecinos ser un desastre.}., \emph{\textbf{Ejemplo:}Chile tiene cobre, a partir del año 1980 le va bien.}, \textbf{los recursos naturales no importan tanto}.
        \item La enfermedad holandésa: tener recursos naturales es perjudicial para la economía.
        \begin{enumerate}
            \item Tener recursos naturales destruye las instituciones de un país. \emph{\textbf{(Paréntesis ``instituciones'':}son cosas abstractas, son normas costumbre, normas de comportamiento pautados, reginenes normativos formales e informales.\textbf{)}}. Por que los recursos naturales generan \textbf{renta sin esfuerzo}, media vez es encontramos un recursos naturales ya lo demás es fácil. \emph{\textbf{Caso} ``petroleo en Petén": curiosamente el ministerio de ambiente no permite explotar el recurso natural del petroleo que sabemos que hay en Petén}. Como la renta es casi gratuita en gobierno se empieza a preguntar cómo se pueda beneficiar de ello. 
            \begin{itemize}
                \item \emph{\textbf{Ejemplo:}Venezuela y su petroleo, cuando se encontró se empezó a regular y se nacionaliza em recurso de petroleo, la carga fiscal de hace años en Venezuela el ingreso es de 2.5\% y gasta 40\% del PIB, el ingreso de GT es de 2.5\% el gasto en GT es 13\% }, esto es por la renta que les provee esto.
                \item Destruye el incentivo privado.
                \item \emph{\textbf{(Paréntesis ``nacionalizar el petróleo'':}es expropiar el petroleo\textbf{)}}
            \end{itemize}
            
            \item Los recursos naturales producen un efecto similar al de la ayuda social.
        \end{enumerate}
    \end{itemize}
    
    \item La destrucción del tipo de cambio:
    \begin{itemize}
        \item Se dice que GT tiene una enfermedad Holandesa. Por que entran muchos dólares \textbf{por remesas}, esto devalúa la moneda, se dice que hay una enfermedad holandesa por el tipo de cambio. Ejemplo:
        \begin{align*}
            \text{EXPORTACIÓN} & \cong \$ 10,000 \\ 
            \text{IMPORTACION} & \cong \$ 18,000 \\ 
        \end{align*}
        Cómo se satisface la diferencia de impoteciones y exportaciones se pagan por importaciones con \textbf{remesas}. \emph{\textbf{Observación: }muchas de las remesas son lavado de dinero}.
        
        \item  La enfermedad holandesa: por el recurso en Holanda en los años 30 encontró gas, al haber un ingreso la moneda del dólar se devalúa la moneda local respecto al dolar. Encuentras el recurso natural $\Rightarrow $ se exporta $\Rightarrow $ entra dólares $\Rightarrow $ cuando entra dólares se daña la competitividad de las empresas.
    \end{itemize} 
\end{enumerate}


\section{Sobrepoblación}
\begin{itemize}
    \item La sobre población: 
    \begin{itemize}
        \item mayor población permite una mayor especialización del trabajo y enfoque en diferentes tareas.
        \item \emph{\textbf{Observación: }imposible producir comida para el mundo con métodos de agricultura primitivo.}
        \item El progreso tecnológico que conlleva mas personas es mayor y permite darle de comer a más personas.
        \item \emph{Citación:``El recurso más escaso que existe son las ideas"}, por ende más personas $\Rightarrow$ más ideas. \emph{\textbf{Observación: }si abortaste a una persona con la capacidad de revolucionar el mundo con una idea}, es mejor tener más personas para el desarrollo. \emph{Citación:``La basura es basura por que no sabemos qué hacer con ella"}.
        \item \emph{\textbf{Observación: }\textbf{Nos preguntamos:} ¿por que hay un estigma con la energía nuclear?, los reactores nuevos en China usa la basura del uráneo de los reactores nucleares.}
        \item El desarrollo económico aumenta la urbanización.
        \item Cuando un país se industrializa $\Rightarrow$ La taza de natalidad se mantiene constante o se disminuye.
    \end{itemize}
\end{itemize}

\section{Explotción}
\textbf{Nos preguntamos:} ¿Explotan los países ricos a los países pobres?
\begin{enumerate}
    \item La izquierda dice que sí, el ejemplo favorito es el colonial.
    \begin{itemize}
        \item La colonia ha sido más costosa para la metrópolis.
        \item \emph{\textbf{Ejemplo:}India, a Inglaterra le costaba muchísimo mantener a sus colonias en India.}
        \item Noción del colegio: Periódo precolonial, periódo colonial explotativo; en la historia económica es al revés.
        \item \textbf{Nos preguntamos:} ¿Cómo 200 tipos de españa semi analfabétas conquistaron 2 imperios bien establecidos?
        \item Pudieron hacerlo por la teocrácia de los incas que igualaban a los reyes a los dioses.
        \item \textbf{La diferencia entre las civilizaciones ricas y pobres es el sistema de instituciones.}
    \end{itemize}
\end{enumerate}

\section{Recomendación de lectura complementaria}
\begin{itemize}
    \item 
\end{itemize}
