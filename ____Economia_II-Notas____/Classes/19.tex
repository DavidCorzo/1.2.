\section{Resolución de corto}
\begin{enumerate}
    \item Sobrepoblación es mejor ya que permite la especialización.
    \item Se rompe la institucionalidad al querer nacionalizar los recursos por ser de mucho ingreso y de poco costo de producción.
    \item El tipo de cambio se destruye al importar más de lo que exportamos, entonces destruya la competitividad de las empresas.
\end{enumerate}

\section{Presentación de noticia}
\begin{enumerate}
    \item Space X 
\end{enumerate}


\section{Sistema de precios vs. Racionalización de dictador benevolente; el mercado laboral y peculiaridades del mismo}
\begin{enumerate}
    \item \textbf{Nos preguntamos:} ¿Opera el sistema de mercado en las empresas? No, dentro de una empresa no se da por incentivo, se dan ordenes, esa es la diferencia entre el sistema político y el sistema una empresa. Una empresa es una institución \textbf{jerárquica}.
    \item \emph{Citación:``Hayek decía que es un ordenamiento organizacional"}
    \item \emph{\textbf{Ejemplo:}un contrato laboral, es ceder parte del tiempo para los fines de la persona que tiene que pagar.} \emph{\textbf{Observación: }las empresas tienen picos de trabajo, los contadores tienden a tener picos de trabajo cuando tienen que generar los estados de resultados}, \emph{\textbf{Definición de ``pico de trabajo":} que hay mucho que hacer algunos días y algunos otros no hay nada.} 
    \item \textbf{Nos preguntamos:} ¿Si hay picos de trabajo, por qué las empresas simplemente no contratan a la gente en los picos de trabajo? \emph{\textbf{La respuesta a esta pregunta es: } es por que el coste de transacción es muy grande}
    \item \emph{\textbf{Observación: }Uber elimina intermedirios, reduce costes de transacción y solo emplea al empleado cuando hay picos de trabajo.}
    \item \emph{Citación:``Cada vez las empresas van a ser más pequeñas, por que las empresas tienen un costo de transacción muy alto"}.
    \item Básicamente es un contrato de trabajo más flexible y eso logra diluir los trabajos por que solo opera en picos, cuando hay más tiempo que se puede adueñar el empleado puede incurrir en más contratos. Es \textbf{pura flexibilización laboral}.
    \item \textbf{Nos preguntamos:} ¿Las criptomonedas eliminan intermediarios? \emph{\textbf{La respuesta a esta pregunta es: }Sí, totalmente reduce los costos de transacción.}
    \item \emph{\textbf{Ejemplo:}IPO, Initial Public Ofering, es cuando inicialmente se ofrece el capital al público.}
\end{enumerate}


\section{PIB, hemos visto cómo se calcula, cómo se saca, qué comprende, sobrepoblación, ahora veremos: si podemos aumentar la riqueza aumentando los factores productivos}
\begin{enumerate}
    \item \emph{\textbf{Ejemplo:}La economía de GT crece $\cong$1\% al año muchos países tienen similar \textbf{Nos preguntamos:} ¿estamos creciendo más rapido que los países desarrollados?} \emph{\textbf{La respuesta a esta pregunta es: }depende de la población per cápita}
    \item \textbf{Nos preguntamos:} ¿Podemos incrementar el PIB incrementando factores productivos? \emph{\textbf{La respuesta a esta pregunta es: }la clave es la eficiencia en el uso de los factores productivos, la pregunta es cómo incremento la eficiencia.}
    \item Un Guatemalteco en EEUU es más productivo que en GT. Por la eficiencia de los factores productivos.
    \item Política de sustitución de importaciones, es una política que busca aislar cada país que sea independiente, fue un desastre para la economía de américa latina y otras. \emph{\textbf{Ejemplo:}Chato, es un carro muy malo y gastón.}
    \item \textbf{Nos preguntamos:} ¿La clave entonces es etender la propia ventaja comparativa del país y enfocarse en producir eso? 
    \item Factores de incremento del PIB, cómo aumentar la eficiencia de los factores productivos:
    \begin{enumerate}
        \item Mayor \textbf{especialización}, economías de aprndizaje, desigual distribución y capacidad. \emph{\textbf{(Paréntesis ``interesante'':}autanquía es desear ser independiente al 100\% producir todo lo nuestro, es común es países\textbf{)}} también se necesita \textbf{Intercambio}, sin intercambio no hay especialización es necesario que las políticas protejan este intercambio se necesita ``rule of law fuerte''. 
        \begin{itemize}
            \item ``rule of law'' (el estado está por encima de la ley), imperio de la ley, es la ley la que gobierna.
            \item Derecho de estado, el gobernante está por encima de la ley.
        \end{itemize}
        
        \item \textbf{Acumulación de capital}: más máquinas es más producción, máquinas que ahorran trabajo implica que todos somos más productivos. En GT se van a EEUU por que allá está el capital. Para aumentar el PIB per capital se necesita acumular de capital, de nuevo para acumular capital se necesita un ``Rule of law'' fuerte, \emph{si la ley no me asegura que me puedo apropiar de lo que produce lo que mi capital produce menor va a ser el incentivo para invertir y acumular capital}, \emph{tieme que ser una \textbf{inteligente inversión de los bienes de capital}}. 
        \begin{itemize}
            \item Interesante: anterior a las máquinas sólo se podría incrementar el capital con animales.
            \item En GT no vienen las mineras grandes por que el coste de reputación es muy grande por que para operar tiene que hacer corrupción jurídica, por eso hay un incentivo a no venir.
        \end{itemize}
        
        \item Acumulación de ahorro: es imposible acumular capital sin ahorrar, es agarrar parte de la producción y guardarla para re-invertir en la producción. El trabajador está consumiendo lo que aún no se ha consumido, consumen los bienes de los empresarios. \textbf{Nos preguntamos:} ¿Quién va a ahorrar si me lo van a quitar después? si no hay un rule of law fuerte por que no hay ahorro por que \emph{Citación:``¡me lo van a quitar mejor me lo como yo!"}. 
    \end{enumerate}
    
    \item Interesante: el problema de el plástico, tomó como orden ministerial prohibir el plástico, es el derecho del estado.
\end{enumerate}
