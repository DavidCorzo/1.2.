\section{Resolución de corto}
\textbf{Omitida}
\section{Noticia}
\textbf{Omitida}

\section{Discusión de clase - Dinero}
\begin{enumerate}
    \item \emph{\textbf{(Paréntesis: es de los temas más importantes de esta clase.}\textbf{)}}
    
    \item Orden:
    \begin{enumerate}
        \item Trueque: 
            \begin{itemize}
                \item coordinación cualitativa:
                    \begin{itemize}
                        \item Tiende a funcionar en pequeños establecimiento por el número de damba.
                        \item El trueque funciona bien en estas civilizaciones, donde no hay casi nada de innovación, limitada de especialización.
                        \item Casi todos se conocen en estas pequeñas.
                        \item El problema es \textbf{doble coincidencia de necesidades en tiempo y lugar}, es decir que tengo que encontrar a alguien que quiera lo que tienes y que tenga lo quieres. 
                        \item Es complicado encontrar a alguien que quiera algo que yo tengo.
                        \item Es limitada la capacidad, tengo que limitarme a que \textbf{coicidentemente} algunos quieran intercambiar por algo que tengo.
                        \item No se presta a coordinación cuantitativa: \emph{en breve que no hay divisibilidad.}
                    \end{itemize}
            
                \item Coordinación cuantitativa:
                    \begin{itemize}
                        \item El permite únicamente intercambio de enteros, no puedo cambiar media vaca.
                        \item Esto con dinero se soluciona, mis capacidades de llegar al bien que quiero se disparan, es un bien común que todos tendrán.
                        \item Ya con dinero la doble coincidencia no se tiene que dar.
                        \item Historia de plata en India: una persona tenía que pagar con un fragmento de plata martillado.
                        \item El trueque por ende tenía un gran problema en el ámbito cuantitativo.
                        \item \emph{\textbf{Ejemplo:}Los aztecas tenían cacao pero tenía problemas también (no es muy duradero, el dinero que no es duradero no es bueno para ahorrar), en roma tenía dos monedas, las vacas y la sal (para transacciones grandes se cambiaban las vacas, para pequeñas se usaba la sal)}.
                    \end{itemize}

                \item Problemas de el trueque: 
                    \begin{itemize}
                        \item Limita la especialización, intercambiar con alguien fuera de la civilización es casi imposible.
                        \item Otro problema es que no hay unidad de cuenta, \emph{es decir} la unidad de cuenta es \textbf{una medida standard}, \emph{\textbf{Definición de ``unidad de cuenta":} es la definición estandarizada del valor de un bien}.
                    \end{itemize}
            \end{itemize} 

        \item Dinero:
            \begin{itemize}
                \item Permite hacer un intercambio triangular.
                \item \emph{\textbf{Definición de ``Dinero":} es un medio de cambio generalmente aceptado, es un bien o mercancía que no satisface una necesidad si no que por que de él deriva bienes de consumo, es un bien \textbf{proxy}.}
                \item Divisibiliad, reciprosidad, que sea duradero.
                \item Función del dinero, cuando se desarrolla el dinero permite separar la venta y compra de un bien, se ven como dos transacciones.
                \item Se empieza a pensar todo en términos de la unidad de medida, por \emph{\textbf{Ejemplo:} cuando un grupo de amigos tiende a salir a beber cervezas las cervezas pueden llegarse a convertir en una unidad de medida}, \emph{\textbf{Ejemplo:} las estampas de fútbol, se empieza a dar una unidad de medida con las estampas de fútbol, estafas dadas por falsificar las cartas pasa también por que es posible que no se vuelvan a intercambiar con la misma persona}.
                \item En los museos de antigua Roma las monedas tenían una vaca, o el Cesar, la vaca está en la moneda porque una moneda valía una cabeza de ganado; el problema de esto es que los precios de la moneda y de las vacas fluctúan.
                \item \textbf{Nos preguntamos:} ¿qué pasa cuando cambian de precio, que ocurre si fluctuá el precio? una autoridad tiene que poner un cambio fijo entre esas monedas, si no el mercado empieza a considerar la nueva fluctuación de precios.
                \begin{center}
                \begin{tabular}{ | p{2cm} | p{2cm} | p{2cm} | p{3cm} | }
                 \hline
                 1 vaca & $\Rightarrow$ & 1 vaca & $\underbrace{\Rightarrow}_{\text{Tiende a Atesorarse}}$ \\ %\multirow{1}{|c|}{Ley de Gregsham} \\
                 1 oro & $\Rightarrow$ & 2 oro & $\underbrace{\Rightarrow}_{\text{Tiende a Circular}}$ \\ 
                \hline
                 \multicolumn{4}{|c|}{Esto se le conoce como ley de Gresham}\\ 
                 \hline
                \end{tabular}
                \end{center}
                
                \item \textbf{Características de un buen dinero}: es un buen dinero el que coordina temporalmente, es decir que el dinero puede ser utilizado hoy para ser intercambiado, puede ser utilizado en dos años para utilizar para intercambio.
                \begin{enumerate}
                    \item Que el dinero sea vendible; se tiende a hablar de vendibilidad de tipos de dinero. \emph{\textbf{Observación: }Hayek y su dinerabilidad, la dinerabilidad es la vendibilidad, un bien más vendible tiene mejores características y por ende un buen candidato a ser dinero} \emph{\textbf{Observación: }\textbf{Nos preguntamos:} ¿por qué crees que en Roma se necesitaba tanto la sal y las vacas?} \emph{\textbf{Observación: }observar cómo bienes se convierten en dinero}
                    \item Fácilmente transportable: un problema que ilustra esto es cuando se utilizaba la tierra como un bien, las tierras no se pueden transportar, por eso la vaca.
                    \item Tiene que ser escaso: para que que sea un bien económico.
                    \item Tiene que ser imperecedero, tiene que ser duradero y no se tiene que estropear pronto, en este aspecto lo mejores son los metales preciosos. \emph{\textbf{Ejemplo:}es posible que todo el oro que tenemos al rededor del mundo pueda ser el que minaron los griegos.}
                \end{enumerate}
            \end{itemize}
        
        \item Crédito:
            \begin{itemize}
                \item En el trueque a veces habían crédito, se dice que en las primeras escrituras era para registrar deudas entre humanos.
                \item El crédito es una de las soluciones a los problemas del trueque. 
                \item El dinero bancario es una promesa a entregar dinero
            \end{itemize}

    \end{enumerate}
\end{enumerate}
