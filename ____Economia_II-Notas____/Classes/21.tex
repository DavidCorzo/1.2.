\section{Resolución de quiz}
\begin{itemize}
    \item Limita la especialización por el hecho que limita todos a una civilización pequeña.
    \item La unidad de cuenta -  es una medida estándar
    \item Características del buen dinero,
        \begin{enumerate}
            \item Vendibilidad
            \item Transportabilidad
            \item Imperecedero
            \item Escaso
        \end{enumerate}
\end{itemize}

%%%%%%%%%%%%%%%%%%%%%%%%%%%%%%%%%%%%%%%%%%%%%%%%%%%%%%%%%%%%%%%%%%%%%%%%%%%%%%%%%%%%%%%%%%%%%%%%
\section{Noticia - El mito del buen gobierno}
\begin{enumerate}
    \item El gobierno no sabe las necesidades de las personas
    \item El gobierno y su eficiencia en brindar la satisfacción a las necesidades de las personas nunca superará la eficiencia del mercado de satisfacer las necesidades de las descoordinaciones en el mercado por la función empresarial
    \item \emph{\textbf{Ejemplo:}China, pretendió ser el gran planeador}
    \item \textbf{Nos preguntamos:} ¿Por qué un estado es ineficiente a conocer las necesidades de las personas y el mercado es mejor?, \emph{\textbf{La respuesta a esta esta pregunta es: }un fenómeno muy peculiar, el sistema de precios comunica información y es directamente el mercado, para que algo pase a nivel legislativo en el gobierno se debe conllevar el proceso de legislación y de ley, la ventaja del mercado es que si se descoordina algo en el mercado inmediatamente se intenta coordinar con la función empresarial.}
\end{enumerate}

%%%%%%%%%%%%%%%%%%%%%%%%%%%%%%%%%%%%%%%%%%%%%%%%%%%%%%%%%%%%%%%%%%%%%%%%%%%%%%%%%%%%%%%%%%%%%%%%
\section{Discusión de clase}
\begin{enumerate}
    \item \emph{\textbf{Observación: }\textbf{Nos preguntamos:} ¿Cuánto pagarías por una carretera que pasas gratis?}, \emph{\textbf{La respuesta a esta pregunta es: }el empresario está evaluando mediante el sistema de precios a querer hacer la carretera, depende de qué tanto coordine hacer una carretera el empresario la hará, pero el estado no evalúa esto, si el mercado hiciera computadoras no podría saber por que no hay un precio final, luego se añade el problema de la captura del regulador}
    \item \emph{\textbf{Observación: }El problema de las empresas que se cartelizan entre ellas tienden a hacer trampa, hay un incentivo a salirse del cartel, el problema es: cuando una persona ajena al cartel y empieza a vender mejor y a precio más bajo, por ende el cartel se disuelve.}
    \item \emph{\textbf{Ejemplo:}La cartelización de la azucar en GT está llena de problemas por la regulación, pueden hacer esta cartelización respaldada del gobierno, este ente gubernamental pone barreras de entrada como la de la \underline{vitamina A}, por ende el cartel no se disuelve con una nueva persona fuera del cartel}.
    \item Dos problemas:
        \begin{enumerate}
            \item El gobierno no tiene la información por que no utiliza precios de mercado. Si el estado hace carreteras lo hace con impuestos que no son voluntarios, si hace mal la carretera no pasa nada.
            \item Captura del regulador, las empresas privadas no pueden limitar la competencia, solo un monopolio con respaldo del gobierno puede prevalecer.
        \end{enumerate}
\end{enumerate}

\section{Continuación, características de un bien dinero}
\begin{enumerate}
    \item (Preliminarmente) Fácilmente transportable
    \item (Preliminarmente) Vendible - hay un rango de dinerabilidad, una de característica de la dinerabilidad es la vendibilidad, de hecho es una de las más importantes. 
    \item (Preliminarmente) Imperecedero -  
    \item Fácil de almacenar - si el bien es imperecedero se puede almacenar, si es perecedero conlleva la implicación que tengo que deshacerme de él por que no va a durar.
        \begin{itemize}
            \item \emph{\textbf{(Paréntesis ``la peste negra'':}producida por las malas condiciones sanitarias, el alcantarillado es algo que impacta la esperanza de vida, curiosamente ni siquiera es la sanidad, es más importante el alcantarillado\textbf{)}}.
            \item Los metales preciosos, \textbf{Nos preguntamos:} ¿Son almacenables? \emph{\textbf{La respuesta a esta pregunta es: }Sí, \emph{\textbf{Observación: }en Venezuela por ejemplo cuando me pagan tengo hasta medio día para comprar algo antes que se deprecie o se pudra, los dólares es prohíbido, entonces las personas tienen su dinero en dólares en su casa por que el banco se lo quita, entonces se dan asaltos a casas por el ``botesito'' de dinero}}
        \end{itemize}
    
    \item Fácilmente divisible - \emph{\textbf{Recordar lo siguiente:}la coordinación cuantitativa, solo los medios de intercambio enteros}, las vacas no son un buen medio de dinero por esta características, por eso en Roma existían dos monedas, las vacas y el oro. \textbf{Nos preguntamos:} ¿Los metales preciosos son fácilmente divisible? \emph{\textbf{La respuesta a esta pregunta es: }depende de la tecnología, en la época de los aztecas o incas no se usaban estos metales por la razón que no eran divisibles por la falta de tecnología.} \emph{\textbf{(Paréntesis ``la emisión de los billetes de \$10,000'':} eran billetes que se cambiaban entre bancos, ¿por qué la gente no quiere un billete que sea más grande? por que hay un sustituto muy bueno que es la tarjeta de crédito/débito \textbf{)}} \newline \emph{\textbf{(Paréntesis ``se inclina al oro'':}en el siglo XIX se inclinan todos por el oro.\textbf{)}}, \textbf{Nos preguntamos:} ¿Las criptomonedas son un buen dinero? \emph{\textbf{La respuesta a esta pregunta es: }Sí, uno puede hacer 0.0000000001 Cryptomonedas, la ventaja de las Cryptomonedas es que no es un pasivo, cuando tenemos dinero en el banco es un pasivo y si quiebra puede incumplir su promesa de pasivo (\underline{riesgo de contraparte}), el bitcoin no tiene uso fuera de cualidad monetaria, el oro por ejemplo tiene características no monetarias como la joyería por ejemplo, si el oro pierde su valor se puede recurrir a su uso no monetario.}
    
    \item Homogéneo - es que dos unidades valen lo mismo una respecto de otra, \emph{\textbf{Ejemplo:} si uno vende una vaca y uno tiene tres, uno va a tender a querer vender la más viejita y el comprador va a querer la más joven}, \emph{\textbf{Ejemplo:}en Ecuador los billetes eran asquerosos}, tienen la \underline{unidad de cuenta} es la misma, pero uno tiene más incentivo a querer no tener el que es más asqueroso a pesar que valen lo mismo.
        \begin{itemize}
            \item Sudar la moneda, pretendía no distinguir cosas que no eran homogénias como homogéneas.
            \item En India por ejemplo el intercambio tenía que comprobar 
        \end{itemize}
\end{enumerate}

%%%%%%%%%%%%%%%%%%%%%%%%%%%%%%%%%%%%%%%%%%%%%%%%%%%%%%%%%%%%%%%%%%%%%%%%%%%%%%%%%%%%%%%%%%%%%%%%
\section{Paréntesis - Las encuestas}
\begin{itemize}
    \item Se habla de las noticias por la razón que hay temas que no están en el programa pero sí es parte de la clase.
    \item Le da más riqueza a la clase.
\end{itemize}
