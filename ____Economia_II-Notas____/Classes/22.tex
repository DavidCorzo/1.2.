\section{Noticia, desarrollo de la noticia de Mises institute}
\begin{itemize}
    \item \emph{\textbf{Definición de ``cratos":} poder políticas}
    \item \emph{\textbf{Definición de ``anarco":} falta de jerarquía}
    \item La educación:
        \begin{enumerate}
            \item Inversión:
                \begin{itemize}
                    \item Importante para demostrar el costo de la señalización para ser más escaso tus habilidades.
                \end{itemize}
            \item Señalización: cuanto es el nivel de educación. \emph{\textbf{Ejemplo:}Doctor aquí en GT, es señalización}.
                \begin{itemize}
                    \item Usualmente la calidad de la educación va para abajo.
                    \item La señalización de ser licenciado.
                    \item \emph{\textbf{Ejemplo:}En Alemania no se pone mucho enfoque en la señalización.}
                \end{itemize}
            \item Ideología: 
                \begin{itemize}
                    \item Cultura, herencia por imitación, psicología inversa.
                    \item Pensar en Hakey, \emph{\textbf{Ejemplo:}¿Comer en un baño?}  
                    \item Se dice que los países nórdicos tienen muy pocos policía.
                    \item La sociedad homogénea cuando la educación es muy alta se disminuye el 
                    \item \emph{\textbf{Observación: }\emph{Citación:``Querer cambiar la sociedad"}, hacer que la educación tenga una agenda ideológica.}, en la UFM se aclara la ideología.
                    \item \emph{Citación:``Las gafas de colores te hacen ver el mundo según las gafas que tengas"}, confirmation bias.
                    \item \textbf{Socialización primaria}, la familia,
                    \item \textbf{Socialización secundaria}, la iglesia, el colegio, 
                    \item ver: Douglas North, \emph{Citación:``Mis cosas son mejor ¿por? mis ideas son mejor ¿por?"}
                    \item La educación es una ideología de adoctrinamiento para disminuir la necesidad de la coacción.
                    \item \emph{\textbf{Observación: }Vinculación educación $\Leftarrow\Rightarrow$ Política}
                    \item \emph{Citación:``Gramsey: la gente ya es muy rica la gente no quiere el comunismo, por eso la izquierda intenta por medio de ideologías hacer popular el comunismo"}.
                    \item Adoctrinamiento por la izquierda a la educación, por eso tienden a ser socialistas y de izquierda.
                \end{itemize}
            
            \item \emph{\textbf{Observación: }\textbf{Nos preguntamos:} ¿Cómo conquistaban los Romanos?}, tocaban la puerta y decían que si se dejaran conquistar no los esclavizaban, de lo contrario los esclavizaban.
            \item \emph{\textbf{(Paréntesis ``Islam, cristianismo'':}Cómo se da la ideología\textbf{)}}
        \end{enumerate}
\end{itemize}

%%%%%%%%%%%%%%%%%%%%%%%%%%%%%%%%%%%%%%%%%%%%%%%%%%%%%%%%%%%%%%%%%%%%%%%%%%%%%%%%%%%%%%%%%%%%%%%%
\section{De qué depende el valor del dinero}
\begin{itemize}
    \item Depende de la habilidad para hacer transacciónes, basado en las características descritas anteriormente.
    \item \textbf{Nos preguntamos:} ¿De qué es el precio del dinero? \emph{\textbf{La respuesta a esta pregunta es: }es el poder adquisitivo}.
    \item IPC, índice de precios al consumo, mide el precio del dinero respecto a varias variables, se mide cuánto cuesta la cesta de canasta básica respecto a las variables de poder adquisitivo de agrupaciones de bienes, la mejor es la que contiene todos lo bienes.
    \item Considerar, el deflactor del PIB considera a todos los bienes de toda una economía, esta es la mejor medida para medir el precio del dinero.
    \item \textbf{Nos preguntamos:} Sabemos que es, cómo se mide pero ¿de que depende?, \emph{\textbf{La respuesta a esta pregunta es: }la demanda y la oferta, estas tienen particularidades.}
    \item Demanda de saldos de caja: 
        \begin{itemize}
            \item \emph{\textbf{Definición:} cuánto tengo en el bolsillo}, la suma razón es por la incertidumbre, considerar: si tengo una emergencia pago en efectivo.
            \item En las crisis $\Rightarrow$ incrementa la incertidumbre $\Rightarrow$ sube la demanda del dinero.
            \item En una crisis deflacionaria se tiende a querer más efectivo esto $\Rightarrow$ que almenos hay mucho incentivo a querer su dinero en cash, cuando no hay crisis uno disminuye su reserva de efectivo. 
            \begin{itemize}

                \item \emph{\textbf{Observación: }Se tiende a observar en depresiones económicas}, \newline  Dudas de el estado de la economía $\Rightarrow$  $\uparrow$ incertidumbre  $\Rightarrow$  $\uparrow$ Demanda de dinero $\Rightarrow$ $\uparrow$ Poder adquisitivo $\Rightarrow$  $\downarrow$ Precios (deflación) 
                
                \item \emph{\textbf{Observación: }Se tiende a observar en expansiones económicas}, \newline Expansión económica $\Rightarrow$ $\downarrow$ Incertidumbre $\Rightarrow$ $\downarrow$ Demanda de dinero $\Rightarrow$ $\downarrow$ Poder adquisitivo moneda $\Rightarrow$ $\uparrow$ Precios (inflación)
            \end{itemize}
        \end{itemize}
\end{itemize}






