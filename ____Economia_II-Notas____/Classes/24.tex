\section{Discución de clase}
\begin{enumerate}
    \item El valor del dinero depende de las características del dinero.
    \item El precio del dinero es el poder adquisitivo.
    \item Calculo lo que pude comprar en un año para esa canasta si me sobró dinero hubo deflación y si no alcanzó hubo inflación.  
\end{enumerate}

%%%%%%%%%%%%%%%%%%%%%%%%%%%%%%%%%%%%%%%%%%%%%%%%%%%%%%%%%%%%%%%%%%%%%%%%%%%%%%%%%%%%%%%%%%%%%%%%
%%%%%%%%%%%%%%%%%%%%%%%%%%%%%%%%%%%%%%%%%%%%%%%%%%%%%%%%%%%%%%%%%%%%%%%%%%%%%%%%%%%%%%%%%%%%%%%%

\section{Noticia, ¿Los empresarios nacen siendo empresarios?}
\begin{itemize}
    \item Mises dice que la gente nace como un canvas en blanco.
    \item No importa las habilidades que un humano tenga, ``por nacimiento pueden haber muchos emprendedores pero no te garantiza encontrar una oportunidad para ejercer la función empresarial'', tienen que tener las habilidades pero tienen que tener una oportunidad para coordinar el mercado.
    \item \textbf{Nos preguntamos:} ¿Cual es la deferencia entre un empresario y un emprendedor? \emph{\textbf{La respuesta a esta pregunta es: }Desde un punto de vista de económico, son lo mismo, en empresiarialidad se hace una diferencia pero esto es una taxonomía.}
\end{itemize}

%%%%%%%%%%%%%%%%%%%%%%%%%%%%%%%%%%%%%%%%%%%%%%%%%%%%%%%%%%%%%%%%%%%%%%%%%%%%%%%%%%%%%%%%%%%%%%%%
%%%%%%%%%%%%%%%%%%%%%%%%%%%%%%%%%%%%%%%%%%%%%%%%%%%%%%%%%%%%%%%%%%%%%%%%%%%%%%%%%%%%%%%%%%%%%%%%

\section{Discusión de clase}
\begin{itemize}
    \item La empresarialidad resuelve descoordinaciones en el mercado.
    \item Versión Kirsner de la función económica del emprendedor:
        \begin{enumerate}
            \item Kirsner: el empresario es casi pasivo pero se asume que las oportunidades ya están ahí. \emph{\textbf{Ejemplo:}Los economiastas y el billete del piso.}
            \item Lachmann/P.Klein: el empresario crea las oportunidades, la crean y la gente lo cree como una necesidad es coordinación. El empresario se está imaginando un futuro lo que quieren coordinar y la crean.
            \item Schumpeter: \emph{Citación:``destrucción creativa"}: nuevos modelos de negocio que destruyen antiguos modelos de negocios. \emph{\textbf{Ejemplo:}Los que producían hielo los dejó en bancarota el invento de la refri}, \emph{\textbf{Definición de ``destrucción cretiva":} los empresarios hacen nuevas cosas o inventan nuevas cosas que ``destruyen'' o sustituyen a las antiguas.} Ver: Neoschupeterianos
        \end{enumerate}
\end{itemize}

%%%%%%%%%%%%%%%%%%%%%%%%%%%%%%%%%%%%%%%%%%%%%%%%%%%%%%%%%%%%%%%%%%%%%%%%%%%%%%%%%%%%%%%%%%%%%%%%
%%%%%%%%%%%%%%%%%%%%%%%%%%%%%%%%%%%%%%%%%%%%%%%%%%%%%%%%%%%%%%%%%%%%%%%%%%%%%%%%%%%%%%%%%%%%%%%%

\section{Continuación de dinero}
\begin{itemize}
    \item \emph{\textbf{Recordar lo siguiente:}El dinero es un medio de intercambio por que no es un fin}.
    \item La demanda de transacciónes no es una demanda de dinero, la demanda de saldos de caja, \textbf{Nos preguntamos:} ¿Hay una demanda de transacción? \emph{\textbf{La respuesta a esta pregunta es: }Si quitaramos toda la incertidumbre no habría demanda en efectivo}.
    \item \emph{\textbf{Observación: }Los pobres tienen más dinero en líquido, un rico tiene el dinero bien invertido por eso es rico. \emph{\textbf{Recordar lo siguiente:} la persona que quiere que se devalúe el quetzal los pobres se hacen más pobres.}}
    \item \textbf{Nos preguntamos:} ¿En una economía donde no hay incertidumbre por que tener un saldo de cuenta? \emph{\textbf{La respuesta a esta pregunta es: }\textbf{\underline{Implícito es la inflación, el explícito es el interés que no gano por no invertir}}}, sin incertidumbre se cancela la demanda y la oferta de dinero, la demanda de transacción es por la demanda de efectivo.
    \item \textbf{Nos preguntamos:} ¿Qué es la oferta de dinero? \emph{\textbf{La respuesta a esta pregunta es: }\emph{\textbf{Recordar lo siguiente:}El trigo es perecedero relación flujo stock, hay bienes que tienen una relación flujo stock alta y otros muy bajos}, el dinero es homogeneo, el dinero es uno de los bienes mas imperecedero, \emph{\textbf{(Paréntesis:}El oro es uno de los bienes con menos flujo stock\textbf{)}}}
        \begin{itemize}
            \item La oferta de dinero es igual al total de dinero en circulación.
        \end{itemize}
    
    \item Leyes:
        \begin{enumerate}
            \item Ley de Gresham:
                \begin{enumerate}
                    \item Gresham es alguien que vivía en el siglo XIV y era banquero, el momento donde más metales preciosos estában llegando a Cevilla.
                    \item El banco donde trabajaba Gresham tenía que cobrarle a Cevilla, entonces Gresham dijo que ``si le cobro lo que tengo que cobrar a Cevilla quiebro a toda la ciudad de Cevilla'', la gente tendía a sudar la moneda.
                    \item Siempre que se establezca un tipo de cambio fijo entre monedas y el tipo de cambio no refleja el valor nominal, \textbf{tiende a circular la mala moneda} ó \textbf{La mala dezplaza a la buena}.
                    \item Interesantemente: cuando un bien es malo tiende a no circular, pero en el dinero es especial, ya que la buena moneda es expulsada y circula la mala.
                    \item \emph{\textbf{(Paréntesis:} el bimetalismo, circulaban las dos, plata y oro\textbf{)}}.
                    \item Considerar lo siguiente:
                        \begin{center}
                        \begin{tabular}{ | p{5cm} | p{5cm} | p{5cm} | }
                         \hline
                         Tipo de cambio [$\underbrace{1.15}_{\text{T.C. fijo}}$] & $\overbrace{\underbrace{\Rightarrow}_{Mina de plata}}^{\text{Oro}} $ & [$\underbrace{1.30}_{\text{T.C. de mercado}}$] $\Rightarrow$ El oro $\Uparrow$ \& Plata $\downarrow$\\
                         \hline
                        \end{tabular}
                        \end{center}
                    
                    \item \emph{\textbf{Observación: }Isaac Newton puso el precio de la libra, puso un precio de la plata respecto al oro porque Newton puso una política que sobre valoraba la plata y infravaloraba el oro.}
                    \item En Venezuela se fija el tipo de cambio pero el precio del mercado disminuyó. 
                \end{enumerate}
            \item Ley de Thiers:
        \end{enumerate}
\end{itemize}
