\section{Resolución de corto}   
\begin{enumerate}
    \item Para qué sirve tener saldos de caja: para la incertidumbre
    \item Por que la inflación hace que los pobres sean más pobres:  que ellos tienen más saldos de caja.
    \item La ley de Gresham tiene un tipo de cambio establecido, y hace circular la mala moneda.
    \item La ley de Thiers aplica con la hiperinflación, los tres pasos y el punto de quiebre.
\end{enumerate}

\section{Noticia \textbf{Nos preguntamos:} ¿Why socialism stinks?}
\begin{enumerate}
    \item Robert Lawson, 
    \item Argumento de Swesia, por los tipos de pseudo-socialismo.
    \item Los empresarios empezaron a huir de las pol+iticas socialistas y Swesia bajó los impuestos.
    \item Cuba: 
    \begin{itemize}
        \item Opresión, persecución, espionaje 
        \item Las famosas etapas del socialismo ``real'':      
        \item Hoteles: inicialmente los turistas se quedarían en hoteles y propagar las ideas socialistas, pero ahora los hoteles están mal.
    \end{itemize}

    \item En las drogas illegales uno no puede quejarse por la calidad del producto, las drogas legales sí se puede alegar la calidad.
    \item Muertes atribuidas al socialismo:
        \begin{itemize}
            \item Mao Zedong
            \item Stalin 
            \item Holocausto
        \end{itemize}
    \item Ver: Holodomor, URRS Rusia con mano de hierro con un montón de repúblicas satélites, el telón de acero (división de europa en la que una mitad es socialista y otra es capitalista). \emph{\textbf{Observación: }Ukrania recibió con los brazos abiertos a los nazis}, \emph{\textbf{Recordar lo siguiente: }Cuando se intentó industrializar la repúblicas agrarias y nadie se queda produciendo la comida y todos se mueren de hambre.}
\end{enumerate}

%%%%%%%%%%%%%%%%%%%%%%%%%%%%%%%%%%%%%%%%%%%%%%%%%%%%%%%%%%%%%%%%%%%%%%%%%%%%%%%%%%%%%%%%%%%%%%%%
\section{Discusión de noticia \& de clase}
\begin{enumerate}
    \item Ben Powel, su investigación de ir a regímenes socialistas y conversar con las personas en los bares.
    \item \textbf{Nos preguntamos:} ¿Por qué la taza de mortalidad de los infantes es tan buena en los países socialistas? \emph{\textbf{La respuesta a esta pregunta es: }Cómo se modifica las cifras del estado, \emph{\textbf{Recordar lo siguiente: }el Correismo, es el siglo del sigo XXI, sostiene el objetivo de nacionalizar a los medios de producción, es la versión latinoamericana del socialismo, se tiende a preferir la adquisición del poder por medios democráticos.}}, lo que hicieron en Ecuador, querían aumentar la alfabetización entonces pusieron a los que estaban saliendo del colegio en lugar de hacer seminario se le ponía a alfabetizar a todos los demás, pero esto solo eran cifras no se alfabetizo casi nada. 
    \begin{itemize}
        \item Las autoridades cubanas se manipulan mucho las cifras ya que los índices de mortalidad infantil de Cuba se entendió que tiraban todas las muertes infantiles las catalogan como muertes en el vientre, y no como muertes infantiles por ende las cifras de mortalidad infantil subió exponencialmente aparentando que el socialismo de cuba esté funcionando.
    \end{itemize}
    Tres pilares de los indices de desarrollo humano: 
    \begin{enumerate}
        \item Renta
        \item Educación 
        \item Sanidad
    \end{enumerate}
\end{enumerate}

\subsection{Menger y su evolución de dinero}
\begin{enumerate}
    \item Menger dice que nadie se inventó el dinero ha surgido espontáneamente, este es fruto de el humano pero no es diseñado por un humano, \emph{\textbf{Ejemplo: }el español surgió inicialmente por personas que hablaban mal el latín, los lenguajes se desarrollan con las sociedades de personas}.
    \item \textbf{Nos preguntamos:} ¿Lenguajes diseñados? \emph{\textbf{La respuesta a esta pregunta es: }existió uno llamado ``esperanto'' que se intentó universalizar el lenguaje con este nuevo lenguaje, pero fracasó por que nadie lo hablaba, se intento para evitar guerras, el esperanto está muerto.} \emph{\textbf{Ejemplo: }Otro ejemplo es el hebreo}.
    \item Utilizar la analogía del lenguaje para entender el surgimiento del dinero, surge \textbf{espontáneamente}.
    \item \textbf{Nos preguntamos:} ¿Proceso de creación del dinero? \emph{\textbf{La respuesta a esta pregunta es: }El proceso de economía trival es increiblemente costoso, el \emph{Citación:``el trueque directo es mucho más indirecto que el trueque indirecto"} }, \emph{\textbf{Observación: }es mucho más probable ver más intercambios en una economía de trueque que en una economía con dinero, \emph{\textbf{Ejemplo: }quiero un reloj $\Rightarrow$ cambio algo $\Rightarrow$ cambio algo $\Rightarrow$ cambio algo $\Rightarrow$ llego a tener un reloj}}.
    \item \emph{\textbf{Observación: }Economías trivales: el trueque}
    \item Hay bienes que se venden fácil y otros que no, hay bienes que por alguna razón tiene una demanda mayor a la otros.
    \item \textbf{Nos preguntamos:} ¿Qué hago para hacer trueque en una economía de dinero? ver el bien que es más demandado y y buscar usarlo como unidad de cuenta, entonces el trueque se disminuye ya que puedo poner lo que quiero intercambiar por el equivalente en ese bien vendible, surge una \emph{una demanda monetaria.}, esto \textbf{incrementa la capacidad de intercambio.} Por eso en diferentes civilizaciones usaban diferentes dineros \emph{\textbf{Ejemplo: }en centroamérica la gente usaba cacao por su vendibilidad pero lo terminaron ya no utilizando por que es perecedero.}
    \item Cuando surge un dinero que sobre sale sobre el resto, es cuando hablamos de dinero, se desarrolla una demanda monetaria y todo se empieza a medir por ese bien altamente vendible.
    \item \textbf{Nos preguntamos:} ¿El estado y la legislación en relación al dinero? \emph{\textbf{La respuesta a esta pregunta es: }El estado no crea el dinero, solo pone un sello que garantiza calidad y cantidad, pero el dinero es selecciónado espontáneamente.}, \emph{\textbf{Observación: }Lydia es una de las civilizaciones que en el siglo VII antes de Cristo en emitir una moneda, \textbf{Nos preguntamos:} ¿por que se pone un borde y un sello en la moneda? \emph{\textbf{La respuesta a esta pregunta es: }asegura la calidad de la moneda, Es un desarrollo reciente que garantiza cantidad y calidad del dinero, pero no es dinero por el sello.}}
    \item \emph{\textbf{(Paréntesis ``dinero fiat'':}era un dinero que era básicamente pasivos, es dinero que se basa en la confianza fiat viene del latín que significa confianza, es dinero fiduciario.\textbf{)}}
    \item \emph{\textbf{Observación: }El amazon giftcard es dinero en la comunidad de amazon, el store credit es un tipo de dinero (generalemente aceptado) en la comunidad de walmart}
    \item \emph{\textbf{Ejemplo: }Bitcoin, los gobiernos más bestias los quieren prohibir}
\end{enumerate}

\subsection{Parcial el lunes 28}


