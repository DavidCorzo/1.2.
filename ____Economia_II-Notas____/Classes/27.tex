\section{Noticia de el crecimiento económico de un país en relación con el banco central}
\begin{enumerate}
    \item La economía en aumento no signifique que la banca deba meterse para asegurarse que no se inflen los precios.  
\end{enumerate} %pib irreal o real

%%%%%%%%%%%%%%%%%%%%%%%%%%%%%%%%%%%%%%%%%%%%%%%%%%%%%%%%%%%%%%%%%%%%%%%%%%%%%%%%%%%%%%%%%%%%%%%%

\section{Análisis de noticia}
\begin{enumerate}
    \item Análisis:
        \begin{itemize}
            \item $M_{\text{Cantidad de dinero}} \leftarrow Q_{\text{Cantidad de bienes}}$ 
        \end{itemize}
\end{enumerate}

\section{Discusión de clase}
\begin{enumerate}
    \item John locke: \emph{Citación:``Como tienen un interés bajo son ricos y por que tenemos interés alto somos pobres"}, un principio al pensamiento económico, el tipo de interés bajo es un efecto del desarrollo económico no es la causa del pensamiento económico.
    \item \emph{\textbf{Recordar lo siguiente: }La base del desarrollo económico es la inversión y para invertir necesito ahorrar, Holanda tenía un tipo de interés bajo porque tenía ahorros, inversión, etc.}
        \begin{center}
        \begin{tabular}{ | p{5cm} | p{5cm} | }
         \hline
        \multicolumn{2}{|c|}{Banco X} \\
        \hline  
        $+\underbrace{\text{Préstamo}}_{\text{Activo del banco}}$ & $+\underbrace{\text{Depósitos}}^{\text{Pasivos del banco}}$ \\ 
         \hline
        \end{tabular}
        \end{center}
    
    \item El banco es básicamente un comerciante de crédito. Intereseante: Con los préstamos incrementa los depósitos. 
        \begin{itemize}
            \item Los cheques son equivalentes a los pagos por tarjeta de débito.
            \item Cuando tu le pagas al banco mediante un depósito se va cancelando el préstamo que adquiriste.
            \item \textbf{La cantidad de bienes} incrementa la cantidad de dinero.  
        \end{itemize}
\end{enumerate}

%%%%%%%%%%%%%%%%%%%%%%%%%%%%%%%%%%%%%%%%%%%%%%%%%%%%%%%%%%%%%%%%%%%%%%%%%%%%%%%%%%%%%%%%%%%%%%%%
\subsection{Cálculo del PIB nominal \& PIB Real}

\[
  \Delta PIB_{\text{Nominal}} = \frac{P_{1}*P_{1}}{\underbrace{P_{0}}_{\text{¿$\Delta p$?}}* \underbrace{q_{0}}_{\text{¿$\Delta q$?}}} \approx 1'08 = 8\% \\ 
  \text{Donde P es precios y q es cantidad}
\]

\[
    \Delta PIB_{\text{Real}} = \frac{P_{0}*q_{1}}{P_{0}*q_{0}} \approx 1'03 = 3\% \\ 
    \text{Donde P es precios y q es cantidad}
\]

Los Datos deben de ser el PIB real no el nominal, en el nominal tiene el efecto precio  




%%%%%%%%%%%%%%%%%%%%%%%%%%%%%%%%%%%%%%%%%%%%%%%%%%%%%%%%%%%%%%%%%%%%%%%%%%%%%%%%%%%%%%%%%%%%%%%%
\subsection{Historia, cómo llegamos del oro a la moneda moderna}
\begin{enumerate}
    \item \emph{\textbf{Recordar lo siguiente: }Cómo se seleccionan los dineros en el bimetalismo, se establece un ratio o tipo de cambio fijo y se genera la ley de Gresham, el camino al infierno está lleno de buenas intensiones}.
    \item Cuando se establecían esos tipos de cambio fijo se tendía a infravalorar y sobrevalorar el otro dinero, es decir que el tipo de cambio no reflejaba el valor real de la moneda.
    \item Se dice que el mundo se pasa casi completamente en el siglo 19 al patrón oro, Inglaterra ya estaba en el patrón oro.
    \item Interesante: Newton, \emph{\textbf{(Paréntesis:} La física de newton\textbf{)}}, \emph{\textbf{(Paréntesis ``uno de los ultimos hombres del renacimiento'':}significa que sabía un poco de todo, por eso se la da el nombre de universidad.\textbf{)}}, Newton ejercía el rol de la ceca \emph{\textbf{Definición de ``ceca":} es un nombre común denominado a un objeto}, Newton establece un precio del oro y de la plata para intentar dejar a Inglaterra en el patrón oro, pero no circuló el oro si otra moneda \textbf{los bancos}. 
    \item Los bancos ya existían en el siglo XVIII.
    \item Entonces tenían el patrón oro es peligroso por el oro, guardarlo en la casa era peligroso también, entonces la gente le empieza a dejar su dinero a especialistas a estos denominados primeros banqueros, estos primeros banqueros eran los goldsmiths ellos tenían ya metales preciosos y ya tenían la especialidad de cuidar esos metales preciosos.
    \item Los goldsmiths les daban a los depositantes papel moneda estableciendo un pagaré.
    \item Entonces empiezan a circular \textbf{Los pagarés del orfebre o los pagarés}, entonces tenemos un patrón oro donde no circula el oro si no billetes de pagarés del orfebre ó papel moneda.
    \item Empieza a surgir el banco central de Inglaterra, empieza intercambiar un papel, entonces le dio \textbf{la característica de transportabilidad al oro.}
\end{enumerate}
