\section{Recomendaciones}
\begin{enumerate}
    \item   
\end{enumerate}
%%%%%%%%%%%%%%%%%%%%%%%%%%%%%%%%%%%%%%%%%%%%%%%%%%%%%%%%%%%%%%%%%%%%%%%%%%%%%%%%%%%%%%%%%%%%%%%%

\section{Video de VisualPolitik}
México y su crecimiento económico:
\begin{enumerate}
    \item López Obrador, referéndum \% de aprobación por causas muy atractivas, cancela la construcción de el aeropuerto internacional de México.
    \item La inversión política en México está en picada por la razón que López Obrador cancela grandes proyectos como el de AIM.
    \item Después de una enfermedad holandesa es decir: recursos nacionalizados $\Rightarrow$ privatizar los recursos, es muy difícil.
    \item PeMEX, un recurso nacionalizado, es difícil privatizar algo que no funciona en el sector pública.
    \item Problema de el sector de educación con el sindicato; un problema de privilegios entre los sindicatos educativos.
    \item Medida de ley que el ejercito puede combatir el narco-tráfico, el problema es que todo cambio para seguir exactamente igual.
    \item El problema del tren Maya, por problemas económicos.
    \item Problemas económicos:
        \begin{itemize}
            \item El problema es que el rule of law es muy cambiante.
            \item También que se paró la construcción y limitó la inversión extranjera.
        \end{itemize}
\end{enumerate}
%%%%%%%%%%%%%%%%%%%%%%%%%%%%%%%%%%%%%%%%%%%%%%%%%%%%%%%%%%%%%%%%%%%%%%%%%%%%%%%%%%%%%%%%%%%%%%%%
\section{Video de Oscar Alejandro El malecón, 12 cosas impresionante de Cuba}
\begin{enumerate}
    \item Puesto \# 12: 
        \begin{itemize}
            \item No hay publicidad.
            \item No hay marcas.
            \item No existen las marcas ni los tipos de productos, lo únicos anuncios son propaganda del dinero.
        \end{itemize}

    \item Los hombres se besan en el cachete: 
        \begin{itemize}
            \item Asunto de cultura.
        \end{itemize}
    
    \item Existen dos monedas nacionales:
        \begin{itemize}
            \item El peso convertible (supuestamente, un peso convertible = un dollar).
            \item Existe el billete de tres pesos convertible.
        \end{itemize}
    
    \item No existen los puestos de venta:
        \begin{itemize}
            \item Ninguno de estos sistema existe en Cuba. 
            \item Hay que traer dinero en efectivo.
            \item No funcionan las tarjetas MasterCard ni Visa.
        \end{itemize}
    
    \item Tratan mejor al turista que al cubano: 
        \begin{itemize}
            \item No se le dan buen trato al cubano.
            \item Está prohibido andar sin camisas.
            \item Si sos turista no importa.
        \end{itemize}
    
    \item Casas y carros viejos:
        \begin{itemize}
            \item Son antigüedades que se mantienen solas.
        \end{itemize}
    
    \item Tarjetas de internet: 
        \begin{itemize}
            \item Se necesita comprar tarjetas para acceder al internet, son básicamente datos, para los cubanos esto insostenible.
            \item nauta, tipo de tarjeta.
        \end{itemize}
    
    \item El paquete semanal: 
        \begin{itemize}
            \item Sistema para poder estar conectado sin internet, es un discoduro que tiene todas las películas de YouTube, netflix etc.
            \item Los videos que se encuentran ahí las escogen los que distribuyen.
        \end{itemize}
    
    \item Salario mínimo: 
        \begin{itemize}
            \item De 20\$ al mes, no alcanza para nada, esta es la fuente de la pobreza.
        \end{itemize}
    
    \item La libreta de racionamiento: 
        \begin{itemize}
            \item Son subsidios por los productos.
            \item Es un instrumento de control político, por que se puede controlar.
            \item La gente piensa que es mejor pero no.
        \end{itemize}
    
    \item La seguridad: 
        \begin{itemize}
            \item No existe la inseguridad.
            \item Cometer un crimen o robo es anti-revolucionario.
            \item Esto es algo bueno.
        \end{itemize}
    
    \item El cubano siempre está feliz: 
        \begin{itemize}
            \item La actitud del cubano es muy buena.
            \item Pero la mayoría de gente está mal y ha cedido a las cosa estar así de mal en Cuba.
            \item Ya perdieron la esperanza del cambio.
            \item \textbf{Nos preguntamos:} ¿Por qué ir a cuba solo para ayudar el régimen? \emph{\textbf{La respuesta a esta pregunta es: }Ayudar a la gente pero está el trade-of de ayudar por medio de impuestos.}
        \end{itemize}
\end{enumerate}

%%%%%%%%%%%%%%%%%%%%%%%%%%%%%%%%%%%%%%%%%%%%%%%%%%%%%%%%%%%%%%%%%%%%%%%%%%%%%%%%%%%%%%%%%%%%%%%%
\section{Video de VisualPolik, Silicon Valley es tan rico}
\begin{enumerate}
    \item Es en California donde está la tercera zona más grande del mundo: 
        \begin{itemize}
            \item No tienen minas de silicio ni ningún recurso natural.
            \item Tiene la tasa de impuestos más alta de EEUU.
            \item \textbf{Nos preguntamos:} ¿Con estas tasas de impuestos cómo es uno de los estados más ricos?
            \item Historia:
                \begin{itemize}
                    \item Shockley, una fábrica de semiconductores basados en silicio.
                    \item Los ocho traidores, se revelan e intentan hacer su propia empresa.
                    \item En 1959, nació la empresa de fondo capital riesgo (ahorros de personas que están dispuestas a arriesgar su dinero).
                    \item El valor de una idea: después de la II guerra mundial, todos regresan a casa y querían emprender, entonces se empieza a demandar el fondo de capital riesgo.
                    \item Rockfeller y sus amigos millonarios financian la idea de los ocho traidores.
                    \item En este proceso nacen el primer fondo de capital riesgo.
                    \item Surge Fairchild semiconductors.
                    \item La clave de esta riqueza es porque lograron por medio de fondos de capital riesgo atraer los ahorros de todo el mundo.
                \end{itemize}
            
            \item El método Silicon Valley: 
                \begin{itemize}
                    \item IKEA, sólo crecía con sus propios fondos.
                    \item Facebook, es un negocio escalable, no importa la cantidad de clientes sus gastos son básicamente los mismos.
                    \item Esto es la clave del funcionamiento de Silicon Valley.
                \end{itemize}
            
            \item Clave: 
                \begin{enumerate}
                    \item El valle de la muerte.
                    \item Convertirse en una empresa.
                    \item Buscar un business angel, una persona dispuesto a invertir en una empresa en el valle de la muerte. $\equiv$ Peter Teal.
                        \begin{itemize}
                            \item El business angel compite con otros business angels.
                            \item Se tratan de competir por la inversión en empresas en el valle de la muerte.
                        \end{itemize} 
                    \item Start-up $\Rightarrow$ crecer, por ende se acude a un fondo de capital riesgo.
                        \begin{itemize}
                            \item En 2005, se hace un fondo de capital de Palmers. 
                            \item Entonces Facebook's Peter Teal, su medio millón de dolares se convirtieron en 15 millones.
                            \item Apple compra Shazam por ejemplo. 
                        \end{itemize}
                \end{enumerate}
            
            \item \textbf{Nos preguntamos:} ¿Se puede replicar el método silicon valley en otros países?
        \end{itemize}
\end{enumerate}
