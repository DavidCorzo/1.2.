\section{Resolución de corto}
\begin{enumerate}
    \item \textbf{Nos preguntamos:} ¿Por qué en holanda había un tipo de interés bajo que en Inglaterra? \emph{\textbf{La respuesta a esta pregunta es: }porque había más ahorro, más inversión.}
    \item \textbf{Nos preguntamos:} ¿Cuál es la base de todo crecimiento económico? \emph{\textbf{La respuesta a esta pregunta es: }ahorro} 
    \item \textbf{Nos preguntamos:} ¿Por qué el orfebre se convirtió en el primer banquero? \emph{\textbf{La respuesta a esta pregunta es: }por que era el especialista, era peligroso y costoso guardar el oro.}
\end{enumerate}

%%%%%%%%%%%%%%%%%%%%%%%%%%%%%%%%%%%%%%%%%%%%%%%%%%%%%%%%%%%%%%%%%%%%%%%%%%%%%%%%%%%%%%%%%%%%%%%%
\section{Noticia Health care insurance, discusión análisis}
\begin{enumerate}
    \item Single payer tax health care, sostiene la propuestas que todos los gastos en salud el gobierno iba a asegurar por medio de impuestos, totalmente financiado por el gobierno.
    \item Esto elimina la competencia y fomenta un ambiente atractivo para un monopolio.
    \item Definen la ``salud esencial'' como la salud mental no es esencial y no lo cubre el gobierno.
    \item Single payer system vs multi-payer system.
    \item Elimina los incentivos para hacer las cosas mejor ya que el gobierno siempre te pagará aun que hagas un producto basura.
\end{enumerate}
%%%%%%%%%%%%%%%%%%%%%%%%%%%%%%%%%%%%%%%%%%%%%%%%%%%%%%%%%%%%%%%%%%%%%%%%%%%%%%%%%%%%%%%%%%%%%%%%
\subsection{Análisis de noticia}
\begin{enumerate}
    \item Sistema Bismac, Bebeich; es el primero que instituye el sistema de salud de Alemania.
    \item Curiosamente, en EEUU McDonald's tenía un mejor plan de sanidad que ObamaCare.
    \item \emph{\textbf{Interesante:} El asunto en EEUU los empresarios deciden proveer a sus empleados por que es un deducible de los impuestos.}
    \item \emph{\textbf{Interesante:} EEUU es uno de los países que hay un exceso en los costos de prevención de enfermedades, todos los que ObamaCare .}
\end{enumerate}

%%%%%%%%%%%%%%%%%%%%%%%%%%%%%%%%%%%%%%%%%%%%%%%%%%%%%%%%%%%%%%%%%%%%%%%%%%%%%%%%%%%%%%%%%%%%%%%%
\section{Discusión de clase}
\begin{enumerate}
    \item El sistema americano, no es plenamente privado, subcidia la demanda, la oferta es libre, subsidia la demanda mediante impuestos, \textbf{\emph{El problema es este: que la oferta no está restringido entonces es gratis ir hasta de broma al médico.}} Por eso EEUU gasta hasta el 18\% del PIB en sanidad, el que más gasta en sanidad en el mundo, el siguiente es Cuba.
    \item $\Rightarrow$ Subcidio de demanda $\Rightarrow$ La oferta puede 
    \item Beverich: 
        \begin{itemize}
            \item Single payer, en Europa lo que ocurre es que la demanda se \textbf{queda fija}, entonces se forman listas de espera.
            \item Con un tope hacen que si tu queres ir al médico tenes que esperar.
            \item En España Daniel se había estropeado la rodilla y se logró curar antes de ver un médico.
            \item El estado define cuánto podes gastar.
        \end{itemize}
    
    \item Bismarck: 
        \begin{itemize}
            \item Te obliga a tener seguro, 
            \item \emph{\textbf{Interesante:} En Singapur te dicen, ``el 15\% de tu sueldo va a ahorrar'', esto funciona por que a pesar que es un poco paternalita, esto soluciona por que hay menos incentivos a no ir al médico solo por la gana de ir. este es el sistema más privado que existe.}
            \item \textbf{Nos preguntamos:} ¿Tiene sentido asegurarme en contra de la gripe? No, uno se asegura en contra de eventos de riesgo parametrizable.
            \item En conclusión el sistema de salud Bismack es más bueno en términos de sanidad.
        \end{itemize}
\end{enumerate}

%%%%%%%%%%%%%%%%%%%%%%%%%%%%%%%%%%%%%%%%%%%%%%%%%%%%%%%%%%%%%%%%%%%%%%%%%%%%%%%%%%%%%%%%%%%%%%%%
\section{Dinero y el patron oro que resulta en la circulación de pagarés}
\begin{enumerate}
    \item Fujo especie de Hume: \emph{\textbf{Definición de ``flujo especie":} es una explicación acerca del funcionamiento del patrón oro, analogía de Hume con vasos comunicantes}
        \begin{itemize}
            \item Estos vasos todos están unidos en una misma base, el asunto es que si le inyectas dinero ( agua ) al vaso el agua sube ( el precio sube ), esto tiene repercusiones en términos de importaciones y exportaciones, \emph{\textbf{Ejemplo: }el dólar no es el mejor dinero internacional por ejemplo.}
            \item \emph{\textbf{Ejemplo: }por que las compus son más caras en GT que en EEUU, por que en la aduana te sampan un montón de impuestos}
            \item $\uparrow$ En la masa monetaria $ \underbrace{\text{(teoría cuantitativa del dinero)}}_{\Rightarrow} \uparrow$ incremento de precios $\Rightarrow$ $\downarrow $ exportaciones $\Rightarrow$ $\uparrow $ importaciones $\Rightarrow$  $\downarrow $ Precios.
            \item Verlo así: por ejemplo cuando se descubrió el nuevo continente de américa provocó un aumento en la cantidad de dinero, después un aumento en los precios, produjo una disminución en la cantidad de exportaciones y aumento en las importaciones produciendo una disminución en la masa monetaria y disminución en la cantidad de precios.
            \item Balanza comercial es = exportaciones - importaciones 
            \item Más crecimiento económico depósitos más inversión por más préstamos, más depósitos.
            \item \emph{\textbf{Interesante:} La guerra mundial hace que se eliminó el patrón oro, el tipo de cabio fijo de 4'86$\frac{\$}{\pounds} $, elimina el patrón oro por la guerra por haber cambiado el cambio fijo del patrón oro.}
            \item \textbf{Nos preguntamos:} ¿Quién paga la guerra? La gente con inflación.
        \end{itemize}
\end{enumerate}

%%%%%%%%%%%%%%%%%%%%%%%%%%%%%%%%%%%%%%%%%%%%%%%%%%%%%%%%%%%%%%%%%%%%%%%%%%%%%%%%%%%%%%%%%%%%%%%%
\section{Parcial}
\begin{enumerate}
    \item Biskmarck: multiplepayer.
    \item Beveridge: es un sistema single payer.
\end{enumerate}
