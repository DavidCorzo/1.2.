\section{Resolución de corto}
\begin{enumerate}
    \item 
    \item Se demandaba poco dinero por la poca actividad, en la expansión se demanda dinero.
    \item 
\end{enumerate}

%%%%%%%%%%%%%%%%%%%%%%%%%%%%%%%%%%%%%%%%%%%%%%%%%%%%%%%%%%%%%%%%%%%%%%%%%%%%%%%%%%%%%%%%%%%%%%%%

\section{Noticia}
\begin{enumerate}
    \item 
\end{enumerate}

%%%%%%%%%%%%%%%%%%%%%%%%%%%%%%%%%%%%%%%%%%%%%%%%%%%%%%%%%%%%%%%%%%%%%%%%%%%%%%%%%%%%%%%%%%%%%%%%
\section{Discusión de clase - temas de crisis económicas}
\begin{enumerate}
    \item AUDIO 24.14
    \item Recesión económica: 
        \begin{itemize}
            \item Las recesiones económicas están 
            \item Mueren empresas que no generan valor suficiente, en un punto de vista a largo plazo esto es algo bueno por que realloca los recursos humanos en las empresas que sí aportan el valor a la sociedad. 
            \item En Japón el gobierno no deja que la economía caiga, esto se llama una recesion en u-v doble.
            \item En EEUU se deja que quiebren las empresas que no aportan valor.
            \item \textbf{Nos preguntamos:} ¿por qué es difícil anticipar una recesión? \emph{\textbf{La respuesta a esta pregunta es: }es muy complicado distinguir una demanda genuina que una demanda artificial, \emph{\textbf{Ejemplo: }La burbuja mobiliaria en GT, cuando uno va a hablar con los empresarios y le dicen que si hay demanda pero en sí no.}} 
            \item \emph{\textbf{Interesante:} Es más probable que la gente invierta en cosas que la gente haya tenido experiencia con.}
        \end{itemize}
    
    \item En una recesión económica se tiende a tomar una de dos decisiones, se deja que pase o el gobierno lo impide a toda costa.
        \begin{itemize}
            \item 
        \end{itemize}
\end{enumerate}

%%%%%%%%%%%%%%%%%%%%%%%%%%%%%%%%%%%%%%%%%%%%%%%%%%%%%%%%%%%%%%%%%%%%%%%%%%%%%%%%%%%%%%%%%%%%%%%%    
\section{Situación monetaria mundial hasta el día de hoy}
\begin{enumerate}
    \item La promesa a entregar oro era muy explícita, entonces se intentó intercambiar billetes sin reserva física de oro y lo que ocurrió fue hiperinflación.
    \item En los años 20 los países vuelven al patrón oro, EEUU vuelve en 1919 y Inglaterra en 1926.
    \item El patrón oro clásico (1872-1914) lo mata la guerra mundial. 
    \item El patrón cambio-oro: 
        \begin{itemize}
            \item Es un cambio casi piramidal, en un patrón oro hay usualmente dos denominadas \textbf{monedas de reserva}, \emph{\textbf{Recordar lo siguiente: }} 
            \begin{center}
            \begin{tabular}{ | p{5cm} | p{5cm} | }
             \hline
             \multicolumn{2}{|c|}{Bancos centrales} \\
             \hline
             ORO & Moneda \\  
            \end{tabular}
            \end{center}
            La clave es que se puede emitir todas las monedas y no tener oro.
            
            \item El banco central decidió mejor mandar dolar \& libra en lugar de enviar oro ya que es muy costoso transportar el oro.
                \begin{enumerate}
                    \item Oro 
                    \item Dolar - 
                    \item Moneda - Nacional
                \end{enumerate}
        \end{itemize}

        
        
    \item Eventos en 1931-1945:  
        \begin{itemize}
            \item Se hace una especie de nacionalismo monetario.
            \item En una conferencia se crea el patrón cambio-dólar:
                \begin{itemize}
                    \item 
                \end{itemize}
            
            \item Este sistema se basa en la confianza de EEUU.
            \item 
        \end{itemize}
    
    \item 1944-1971: 
        \begin{itemize}
            \item En EEUU tiene 
        \end{itemize}
\end{enumerate}
    