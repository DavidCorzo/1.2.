\section{Resolución del corto}
\begin{enumerate}
    \item C.R. es lo que el banco retiene, el cálculo de veces que se pueden multiplicar con 15\% es $\frac{1}{0.15} \approx 6.667 $
    \item Cuando el C.R. sube las ganancias del banco bajan por el multiplicador monetario ser menor.
    \item Dealer: no utiliza el banco, el broker: sí.
\end{enumerate}

%%%%%%%%%%%%%%%%%%%%%%%%%%%%%%%%%%%%%%%%%%%%%%%%%%%%%%%%%%%%%%%%%%%%%%%%%%%%%%%%%%%%%%%%%%%%%%%%
\section{Noticia - Mercados bursátiles más atractivos del momento EEUU no es uno de ellos}
\begin{enumerate}
    \item \textbf{Nos preguntamos:} ¿Capitalización burcente? $\Rightarrow$ precio de acción * candidad de acciónes.
    \item La relación precio beneficio: métrica con márgen de error ya que no se pueden anticipar algunos eventos.
    \item Se busca el actual alto futuro bajo.
    \item Price Earning Ratio $\Rightarrow$ PER 
    \item \emph{\textbf{Interesante:} Sudafrica impone políticas racistas en contra de los blancos, esto incrementa el incentivo a no ahorrar por que en cualquier momento te lo pueden venis a quitar.}
    \item Mar-to-market: valor de mercado.
    \item \emph{\textbf{Consultar el siguiente recurso:} Burbuja inmobiliaria.}
    \item Enrun - pagó más dividendos ue beneficios. 
\end{enumerate}

%%%%%%%%%%%%%%%%%%%%%%%%%%%%%%%%%%%%%%%%%%%%%%%%%%%%%%%%%%%%%%%%%%%%%%%%%%%%%%%%%%%%%%%%%%%%%%%%
\section{Discusión de clase}
\begin{enumerate}
    \item Importante: saber qué mide cada métrica.
    \item \textbf{Nos preguntamos:} ¿por qué amortizar un activo en lo menor? Por fisco.
        \begin{itemize}
            \item Comparar África vs. EEUU, 
            \item \emph{\textbf{Interesante:} \textbf{Nos preguntamos:} ¿cuánto vale una patente?}
            \item Es necesario ser escéptico, muy pocos informes financieros van a decir que están en la mierda. 
        \end{itemize}
\end{enumerate}

%%%%%%%%%%%%%%%%%%%%%%%%%%%%%%%%%%%%%%%%%%%%%%%%%%%%%%%%%%%%%%%%%%%%%%%%%%%%%%%%%%%%%%%%%%%%%%%%
\section{Banca central}
\begin{enumerate}
    \item Primer banco central del mundo: Suecia.
    \item Nacimiento de banco centrales, dos oleadas: 
        \begin{itemize}
            \item Primera oleada: siglo XIII \& siglo XIX
            \item Nace para financiar al gobierno, la banca comercial nace primero que la banca central.
            \item En sus inicios, es una forma de financiar al gobierno en momentos de crisis como guerra.
            \item El estado le llega un banco comercial, la gente acude a los orfebres y se crean los bancos comerciales, ya existe a priori entre el estado, los reyes solían pedir préstamos a estos bancos comerciales, si una persona tiene un monton de dinero perteneciente al público, el estado se acerca y le dice ``te voy a dar una concesión'', un ``caramelito''; llega y le ofrece \textbf{el monopolio de hacer billetes} le entrega el monopolio de billetes, a una empresa privada le parace muy atractivo esto por que puede ejercer sin competencia. Prestamos al estado por un monopolio, los únicos que pierden es el público.
            \item Así nace toda la banca central antes al siglo XX.
            \item El estado tiende a no pagar prestamos, para el estado no se le pueden embargar bienes.
            \item \emph{\textbf{Ejemplo: }Otros estados sí pueden joder a otros más débiles, México fue invadido por Francia; Aerolíneas Argentinas, el avión presidencial de Argentina no puede salir de Argentina ya que la soberanía de los otros estados se lo quitan por tener una deuda.}
            \item El estado se monopoliza en todo, en la coacción, en el billete.
            \item El banco central en GT lo funda un estadounidense, Edwin Walter, llamado ``doctor dinero'', se fundan para \textbf{dar uniformidad monetaria}.
            \item El público no sabe distinguir un buen banco de un mal banco.
            \item Esto es \textbf{una justificación para fundar un monopolio}.
            \item \emph{Citación:``Napoleon decía que para conocer al mundo se llega a los 20 años, los cambios y lo que está mal se toman por dados"}.
        \end{itemize}
    
    \item Funciones especificas del banco central: 
        \begin{itemize}
            \item Justificación: nacen para tener un monopolio de emisión monetaria, es al final un soporte para el estado.
            \item Banquero del gobierno:
                \begin{itemize}
                    \item Agente de pagos: guarda la tesorería de dinero, el agente de pagos de gobierno.
                    \item Solamente el estado y lso bancos comerciales pueden tener cuenta en el banco central, no se puede abrir una cuenta en el banco central, solo le permite cuentas a pocos agentes en la economía.
                    \item Otorgar préstamos y hacer de agente financiero, es el banquero de inversión.
                \end{itemize}
            
            \item Cuando el banco central le da muchos préstamos al gobierno produce inflación, como al prestamista le beneficia la inflación el gobierno se causa una auto-ganga.

            \item El desarrollo monetario ha sido retroactivo debido a que el público ha estado mucho menor atento al banco.

            \item Legislación de ``anti-panico-financiero'' un analista inteligente no puede operar y causa que los desastres financieros tengan un factor de exponenciación en su caos, esto contribuye al público ser menos precavido, el balance puede ser fácilmente maquillado. 
        \end{itemize}
\end{enumerate}
