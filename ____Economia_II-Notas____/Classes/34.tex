\section{Resolución de corto}
\begin{enumerate}
    \item El moral hazard:
        \begin{itemize}
            \item Es tener un ``castigo'' a que la banca central incentive a mejorar la calidad de decisiones de lso bancos comerciales por que lo hace muy costoso que la banca central salve a los bancos.
        \end{itemize}
    
    \item La cámara de compensación: 
    \item Estabilizador de ciclos económicos: 
\end{enumerate}

%%%%%%%%%%%%%%%%%%%%%%%%%%%%%%%%%%%%%%%%%%%%%%%%%%%%%%%%%%%%%%%%%%%%%%%%%%%%%%%%%%%%%%%%%%%%%%%%
\section{Balanza de pagos}
\begin{itemize}
    \item La balanza de pagos registra todas las transacciones que entran y salen del país.
    \item La balanza de pagos debe de dar cero por que entre ellas deberían de compensarse.
    \item Cada balanza tiene una sub-balanza.
\end{itemize}

%%%%%%%%%%%%%%%%%%%%%%%%%%%%%%%%%%%%%%%%%%%%%%%%%%%%%%%%%%%%%%%%%%%%%%%%%%%%%%%%%%%%%%%%%%%%%%%%

\subsection{Balanza de Pagos:}    
\subsection{Balanza de cuenta corriente:}
\begin{itemize}
    \item Registran transacciones que termina y no generan más actividad económica.
    \item \emph{\textbf{Ejemplo: }Exporto bananos y me dan crédito, el banano va en la balanza corriente, el crédito está en capitales / financiera}
    \item Son transacciones corrientes, que se cierran ahora.
    \item Al no genera actividad económica futura.
\end{itemize}

\begin{enumerate}
    \item Balanza comercial:
        \begin{itemize}
            \item Mide las exportaciones + importaciones, de bienes y servicios.
            \item Hay más sub-balanzas dentro de esta.
            \item Esta es la más común.
        \end{itemize}
    
    \item Balanza ingreso primario:
        \begin{itemize}
            \item Recoge los beneficios, rendimientos, utilidades; del extranjero. 
            \item Está relacionada con la balanza financiera, en GT esta es negativa por que hay más gente invirtiendo en GT que GTecos en el exterior.
            \item Si un país le meten inversión baja esta, si de ese país se invierte en el extranjero esto sube.
        \end{itemize}
    
    \item Balanza ingreso secundario:
        \begin{itemize}
            \item Es el $\equiv$ a remesas, no solo remesas.
            \item Es unilateral, es ingreso de dólares gratis, no por transacciones de productos, literalmente a GT le entran dólares en remesas así.
        \end{itemize}
\end{enumerate}

%%%%%%%%%%%%%%%%%%%%%%%%%%%%%%%%%%%%%%%%%%%%%%%%%%%%%%%%%%%%%%%%%%%%%%%%%%%%%%%%%%%%%%%%%%%%%%%%
\subsection{Balanza de capitales/financiera:}
\begin{itemize}
    \item Registran transacciones que provocan actividad económica en el futuro.
    \item \emph{\textbf{Ejemplo: }Un GTeco en el extranjero se registra aquí.}
    \item 
\end{itemize}

\begin{enumerate}
    \item Inversiones extrangera directa:
        \begin{itemize}
            \item Toda inversión (no es corriente) que retornará los beneficios de la inversión.
            \item Es inversión de un residente fuera que busca o controlar una empresa que ya existe.
            \item \emph{\textbf{Ejemplo: }La inversión extranjera directa quiere control, por ejemplo pone una empresa que el activamente busca una oportunidad de negocio y la está explotando.} \textbf{No sería} ser accionista de los miles de accionista de apple, por ejemplo.
            \item Comprende cuando se controla más del 10\% de una empresa, con intensión de quedarse a largo plazo y poder controlar la empresa.
            \item Es el empresario que invierte activamente.
        \end{itemize}
    
    \item Inversiones portafolio:
        \begin{itemize}
            \item Inversiones de acciones o bonos que no pretende controlarla y tiene intención de quedarse temporalmente corto.
            \item \emph{\textbf{Ejemplo: }Comprar acciones de Apple.}
            \item Empresario que invierte pasivamente.
        \end{itemize}
    
    \item Otras Inversiones:
        \begin{itemize}
            \item Es un cajón desastre, de estas resaltan dos:
                \begin{itemize}
                    \item Crédito comercial: Exportar algo y me lo pagarás. Se usan pagarés, todo esto es otras inversiones.
                    \item Depósitos: \emph{\textbf{Ejemplo: }Un americano que tiene una cuenta en GT}; es tener dinero en otros países.
                \end{itemize}
        \end{itemize}
\end{enumerate}
%%%%%%%%%%%%%%%%%%%%%%%%%%%%%%%%%%%%%%%%%%%%%%%%%%%%%%%%%%%%%%%%%%%%%%%%%%%%%%%%%%%%%%%%%%%%%%%%
\subsubsection{Usualmente...}
\begin{itemize}
    \item La balanza cuenta corriente se compensa con la de financiera, pero evidentemente no siempre pasa.
    \item Cuando la balanza cuenta corriente genera muchos dólares afecta otras inversiones.
    \item Cuando se aprecia la moneda nacional aumentan las exportaciones.
    \item Los exportadores quieren (eufemismos) devaluar la moneda para combatir las remesas.
\end{itemize}
%%%%%%%%%%%%%%%%%%%%%%%%%%%%%%%%%%%%%%%%%%%%%%%%%%%%%%%%%%%%%%%%%%%%%%%%%%%%%%%%%%%%%%%%%%%%%%%%
\subsection{Reservas internacionales}
\begin{itemize}
    \item Acumulación de dólares, oro, divisas.
    \item Principalmente las reservas que tiene el BanGuat.
    \item El BanGuat logró duplicar sus reservas.
\end{itemize}

Resumen...
\[
  \text{Ingreso} - \text{Gasto} = \text{Cambios en deuda} \pm \text{Camvios en activos} \pm \text{Cambios en dinero}
\]

\begin{itemize}
    \item \emph{\textbf{Observación: }Crisis en la balanza de pagos, se tiene que ajustar de una manera abrupta, \emph{\textbf{Ejemplo: }tener déficit del 10\% de PIB}}
    \item En GT está negativa, se perdía dólares, en GT las balanzas están cada vez menos negativa.
    \item Los exportadores tienen un incentivo a querer que el BanGuat  compre más dólares para apreciar su valor, esto por querer vender en dólares.
\end{itemize}

\subsection{De qué depende el precio de la moneda}
\begin{itemize}
    \item Depende de la oferta y de la demanda (no decir eso) viene determinada de la balanza comercial y por la financiera.
    \item Determina el poder adquisitivo de las monedas. 
\end{itemize}

%%%%%%%%%%%%%%%%%%%%%%%%%%%%%%%%%%%%%%%%%%%%%%%%%%%%%%%%%%%%%%%%%%%%%%%%%%%%%%%%%%%%%%%%%%%%%%%%
\subsection{Determinación de poder adquisitivo de paridad de los precios}
\begin{itemize}
    \item Un bien debe de costar lo mismo en todas partes del mundo. Por que si no costara lo mismo habría una oportunidad de arbitraje.
    \item PPA, los bienes tienen tienen que tener el mismo precio en todas partes del mundo, a excepción:
        \begin{itemize}
            \item Costes de transportación.
            \item Bienes no transables: los bienes inmuebles, por que la vivienda está más barata en el Salvador pero no me puedo traer las viviendas para aquí.
            \item Los impuestos y los aranceles: cuando son suficientemente altos provocan contrabando, \emph{\textbf{Ejemplo: }el tabaco}.
        \end{itemize}
\end{itemize}
