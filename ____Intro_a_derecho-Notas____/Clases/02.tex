\section{Introducción al derecho}
\begin{itemize}
    \item La clase es socrática
    \item No se termino la anterior clase
\end{itemize}

\section{Elementos}
El estado está integral por diversos elementos:  \newline 
\begin{itemize}
    \item Elementos previo a la existencia del estado 
    
    
    \begin{enumerate}
        \item Población
        \begin{itemize}
            \item constituye el elemento humano
            \item Aristóteles: el hombre es un ser sociable por naturaleza.
            \item Habrá un número de hombre requerido para formar un estado \emph{\textbf{Respuesta}:no, no hay un número}
            \item En que consiste la nacionalidad? \emph{\textbf{Respuesta}:un vínculo a la patria}
            \item quienes son nacionales de un estado 
            
            \item Clases de nacionalidad: en Guatemala se permite la doble ciudadanía, la nacionalidad es un derecho humano. \textbf{Ejemplo: } amigos de China tuvieron un bebé en China (jus sanguini) se tuvieron que hacer tramites de para establecer ciudadanía.
            \begin{itemize}
                \item Originaria o de origen: la de sangre.
                \item Adquirida: se adquiere por que un quiere.
            \end{itemize}
        \end{itemize}
        \begin{itemize}
            \item \textbf{Nos preguntamos:} ¿puede un estado remover la nacionalidad a una persona? Los convicted-felons no pueden votar, pero es un derecho humano reconocido internacionalmente.
            \item \textbf{Nos preguntamos:} ¿qué pasaría si cataluña se separa de españa? \emph{\textbf{Respuesta}:se crearían nacionalidad de esa región en particular para aislar y reconocer el nuevo estado}
            \item Guatemala Constitución : 144;145;146 nacionalidad de origen; \emph{\textbf{Paréntesis:}se hace la excepción con los funcionarios políticos} \emph{\textbf{Paréntesis:}la nacionalidad es básicamente cómo se va a tratar el estado a las personas, la constitución tratará como nacional a las personas de centro américa.} \emph{\textbf{Paréntesis:}no hay reciprosidad en otros paises nos considerarían naturalizados no nacionales, solo en Guatemala se trata a los extranjeros convertidos en nacionales como nacionalidades de origen}
            \item \textbf{Nos preguntamos:} ¿cuál es la diferencia entre nacionalidad y ciudadanía? ciudadanía es cuando una persona es nacional y mayor de edad, y cuando una persona es menor de edad es considerada por el estado como solamente nacional.
        \end{itemize}
        
        
        \item Territorio
        \begin{itemize}
            \item Porqué se dice que el territorio previo y necesario para la existencia del estado?
            \item su extensión es variable 
            \begin{itemize}
                \item Estado con mayor extensión? \emph{\textbf{Respuesta}:Russia}
                \item Estado con menor extensión? \emph{\textbf{Respuesta}:El vaticano}
                \item Extensión de Guatemala? \emph{\textbf{Respuesta}:131,894}
            \end{itemize}

            \item por qué se dice qie la unidad territorial del estado es una unidad jurídica y no geográfica? \emph{\textbf{Respuesta}:no se necesita que este unido el territorio para que el estado ejerza su función, ya que en el caso de EEUU con Alaska queda a miles de km de la masa central de EEUU, por eso no es geográfica es un \textbf{unidad juridica}}.
            \item \textbf{\emph{Ejemplo:}} Alaska (EEUU), islas de Japón
        \end{itemize}
        \begin{itemize}
            \item \textbf{Nos preguntamos:} ¿Cómo puede definirse el territorio? \emph{\textbf{Respuesta}: una característica fundamental de }
            \item \textbf{Nos preguntamos:} ¿Cuando pierde su territorio desaparee. (nación) \emph{\textbf{Respuesta}:desaparece el territorio de un estado entonces se convierte en nación}
            \item Caracteres generales del territorio:
            \begin{itemize}
                \item estabilidad
                \item Circunscripción territorial definida reconocida por otros estados
            \end{itemize}
            \item El territorio incluye el espacio, espacio marítimo y aéreo; Art. 142 constitución de Guatemala
        \end{itemize}

        \begin{center}
        \begin{tabular}{| p{5cm} | p{5cm} |}
         \hline
        \textbf{Postivas} & \textbf{Negativas} \\
        \hline
        a) constituye el asiento físico de la población del estado. & Circunscribe los límites de la actividad estatal (fronteras). \\
        b) Constituye la fuente fundamenta de sus recursos.& Frena la actividad de estados extranjeros. \\
        c) Contituye la fuente fundamental de sus recursos. & \\
         \hline
        \end{tabular}
        \end{center}

        \begin{itemize}
            \item Espacio terrestre:
            \begin{itemize}
                \item 121 Const.
                \item 122 Const.
                \item 123 Const.
                \item 124 Const.
                \item 456 C.Civil: bienes públicos y bienes privados, bienes de uso común y bienes de uso no común; bienes públicos de uso común como un parque o bienes publicos de uso no comun por ejemplo tratar de entrar al palacio nacional.
            \end{itemize}
            \item \textbf{\emph{Ejemplo:}} si se encuentra petroleo en tu tierra es propiedad del estado.
            \item Hay leyes de animales, \emph{\textbf{Respuesta}:hay leyes que hace el CONAP protegen las reservas naturales y animales, el estado está limitado en cuanto a qué quiere hacer.}
        \end{itemize}
    \end{enumerate} 
    
\end{itemize}


\textbf{El estado}: es una sociedad humana regida a un orden político-jurídico, con la finalidad de alcanzar el bien público temporal.


\begin{itemize}
    \item Espacios
    \begin{itemize}
        \item Espacios
    \end{itemize}

    \item Elementos derivados de el ordenamiento jurídico
    
    
    \begin{enumerate}\setcounter{enumi}{2}
        \item Poder
        \item Fin
        \item Orden jurídico
    \end{enumerate}


\end{itemize}
