\section{Espacio marítimo y espacio aéreo}
El estado comprende los espacios marítimos siguientes:
\begin{enumerate}
    \item Mar territorial: En Guatemala son 12 millas náuticas, \emph{\textbf{(Paréntesis:}La milla terrestre no es igual a la milla nautica aprx. 0.8km = 1 milla náutica)}
    \item Aguas interiores
    \item Zona económica exclusiva: 200 millas náuticas desde la costa
    \item Zona contigua: 24 desde la costa
    \item Plataforma continental \newline 
    --------------------------------------------
    \item Alta Mar: No forma parte del territorio de un estado. Comprende las partes del mar no incluidas en la zona económica exclusiva. Hay plena libertad de todo.
    \item La zona: No forma parte del territorio de un estado. 
\end{enumerate}

\section{Espacio marítimo}
\textbf{Convención sobre el derecho del mar}: es una convención sobre el derecho del mar, se establece todo como qué medidas cuánto, las medidas del mar territorial. \newline 

\subsection{Derechos el estado Ribeño}
\begin{center}
\begin{tabular}{ | c | c | }
\hline
 Todos los que implique el ejercicio de su soberanía & Por ejemplo: Regular navegación, pesca, investigación  \\
\hline
 Derechos de otros estados & NINGUNO \\  
 \hline
\end{tabular}
\end{center}

\subsection{Zona contigua}
\begin{itemize}
    \item Zona de mar adyacente al mar territorial hasta una distancia de 24 millas marinas contadas desde la líneas de base a partir de las cuales de mide el mar territorial.
    \item \textbf{Derechos del estado}: tomar medidas de fiscalización para prevenir las infracciones aduaneras, fiscales, de inmigración, o cuestiones sanitarias:
    \item \textbf{\emph{Ejemplo:}} Si un barco no identificado se va acercando a la costa y es un potencial peligro al estado no es considerado un paso inocente y se necesita solicitar permiso al estado.
\end{itemize}

\subsection{Derechos de los otros estados}
Derechos de paso inocente:
\begin{itemize}
    \item Se permite pasar simplemente, el barco del aborto por ejemplo no era un paso inocente.
    \item Los cruceros estarían cruzando en la zona contigua, si quiere entrar al mar territorial se necesita pedir permiso al estado.
    \item \textbf{\emph{Definición} :Son pasos que alteran la seguridad de alguna manera en el estado.}
\end{itemize}

\subsection{Zona Económica exclusiva}
\begin{itemize}
    \item 
\end{itemize}

\subsection{Derechos del estado ribereño}
\begin{itemize}
    \item Tiene derechos de soberanía para fines de exploración y explotación, conservación y administración de los recursos naturales vivos y no vivos del lecho y el subsuelo del mar.
    \item Se puede hacer islas en la zona económica exclusiva.
\end{itemize}

\subsection{Borde exteriores de la plataforma continental}
\textbf{\emph{Definición} :Es el lecho de el subsuelo que se encuentra de bajo del suelo del mar, se pueden tener cables, se pueden hacer perforaciones siempre con el consentimiento del estado, fondos marínos y oceanos y su subsuelo fuera de los límites de la jurisdicción nacional.}\newline 
\textbf{\emph{Ejemplo:}} Brooklin 99


\section{Espacio aéreo}
\textbf{\emph{Definición} : el espacio que existe entre la atmósfera y el territorio de un estado} \newline 
Es esencial por:
\begin{itemize}
    \item Mantenimiento de seguridad y defensa del estado
    \item Transporte aéreo
\end{itemize}

\subsection{Tipos de espacio aéreo}
\begin{itemize}
    \item Controlado: aquel espacio donde una torre de control por herramientas como radar se establece un espacio aéreo Controlado
    \item No controlado: \textbf{\emph{Ejemplo:}} torre de control de la aurora y torre de control en peten, pero cuando se fumiga con aviones en fincas solo se avisa cuando se usa el avion para comunicar que uno va a estar en el aire.
    \item Espacio aéreo de uso especial: Con propósitos mas que todo militares, es como un territorio que usa el militar para prácticas de soldados por ejemplo.
    \item Derechos sobre el espacio aéreo: qué se puede hacer.
    \item En cuanto a una línea recta no hay, simplemente se establecieron ciertos criterios.
    \item Libertades del aire: 8 libertades principales, los derechos que tiene un estado.
    \item Se clasifican en libertades técnicas y libertades comerciales.
\end{itemize}

\subsection{Derechos de espacio aereo}
\begin{enumerate}
    \item Libertad de pasar sin aterrizar.
    \item Falla técnica, tiene derecho a aterrizar por emergencias. \textbf{\emph{(Ejemplo:una señora se sentía mal en un vuelo de EEUU a Guatemala, se tuvo que seguir un protocolo que atrasó 1:30h y media.)}} Referente a vuelos internos.
    \item Libertades comerciales: desembarque \textbf{\emph{(Ejemplo: pasajeros, correo y carga tomados en el territorio del país cuya nacionalidad posee la aeronave.)}} Involucra lucro. Referente a vuelos internos.
    \item Embarcar, es una libertad comercial de embarque. Involucra lucro.
    \item Poder embarcar y desembarcar en cuanto matrículas de estados diferentes.
\end{enumerate}

\subsection{Espacios ultra-terrestres}
\begin{itemize}
    \item Son espacios como la luna y otros cuerpos celestes
    \item Regula en cuanto a lo que el ser humano pueda mandar a la orbita, ya que despues de cierto tiempo se convierte en desechos
    \item Los cuerpos celestes son patrimonio común.
    \item Regulación constitucional:
    \begin{itemize}
        \item Art. 121 literal b
        \item Art 142 literal a 
    \end{itemize} 
    \item Dirección general de aeronáutica civil
    \begin{itemize}
        \item Regulada en el decreto número 93-2000
        \item Art 3
        \item Art 7 literal a
        \item Art 66
    \end{itemize}
    \item Los derechos de la política de cielos abiertos: reconocer el embarque y desembarque de naves de diferente nacionalidad.
\end{itemize}

\subsection{En Guatemala}
\begin{itemize}
    \item Regulación de drones:
    \begin{itemize}
        \item Aeronaves no tripuladas
        \item Necesidad de registro
    \end{itemize}
    \item Globos aerostáticas:
    \begin{itemize}
        \item RAC 31
    \end{itemize}
    \item Vuelos en parapente y ala delta:
    \begin{itemize}
        \item RAC 103
    \end{itemize}
\end{itemize}

\subsection{Tarea}
\textbf{}   
