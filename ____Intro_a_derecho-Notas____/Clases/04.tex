\section{Elementos posteriores del estado}

\subsection{El poder}
\begin{itemize}
    \item El poder: En la antigüedad, se necesitaba el estado, la gente buscaba el poder en gobernantes que consideraban dignos de gobernar. 
    \begin{itemize}
        \item El estado \textbf{\emph{Caso ``Juan Sitierra": Era un gobernante }}, se buscaba orden. no puede existir sin poder, no puede existir sin llegar a su fin tempora. y no se puede llegar al fin sin poder. El estado impone el orden a través del poder.
        \item Orden:Se considera mucho el orden, tiene que haber orden.
        \item Poder (autoridad): poder no es absoluto
    \end{itemize}


    \item Características del poder:
    \begin{enumerate}
        \item No es absoluto, \textbf{\emph{Definición: Absoluto es un poder sin límites, como Luis XIV}}, lo que más regula el comportamiento de los gobernantes es la \textbf{Constitución}, hoy en día se tiene límites.
        
        \item Supremo o soberano; es supremo pero no absoluto, el poder es el máximo poder de un territorio y todo lo que conlleva su territorio, es supremo y soberano, el poder esta por encima de todos los demás poderes.
        
        \item Poder de derecho = ordenamiento jurídico; \textbf{\emph{Caso ``Quiché": El mismo pueblo indígena en Nebaj se da el caso de derecho indígena. Es más una ausencia de la autoridad.}}, significa que está determinado por un ordenamiento jurídico.
        
        \item Poder de dominación = Coacción; \textbf{\emph{Definición:  El estado puede ejercer su poder por coacción, o por fuerza.}}
    \end{enumerate}

    
    \item Tareas del poder
    \begin{center}
    \begin{tabular}{ | p{6cm} | p{8cm} | } 
     \hline
    Gobernar & Administrar \\
    \hline
    Dirección de los ciudadanos & Organización de la función administrativa \\
    \hline
    Se gobierna a persona & Se administran cosas \\ 
    \hline
    Mandatos & Satisfacción de necesidades COLECTIVAS \textbf{\emph{(Ejemplo: Carreteras, servicios, públicos)}} \\
     \hline
    \end{tabular}
    \end{center}

    Administrar \textbf{\emph{Definición: se refiere mas que todo a prestar servicios públicos}} \newline 
    Gobierno \textbf{\emph{Definición: Formular mandatos para la conservación dek estado y para el logro de sus fines.}}

    
    \item Soberanía:
    \begin{itemize}
        \item \textbf{Nos preguntamos:} ¿Cómo definirían la soberanía? \emph{(\textbf{Respuesta}:Es el poder supremo. y soberano es })
        \item Art. 141 Constitución; La subordinación entre las ramas del estado es prohibida \textbf{\emph{Caso ``Baldeti y la corrupción": Daba dinero a diputados para aprobar ciertas leyes.}}
        \item \emph{\textbf{(Paréntesis:}Las ramas del estado, ejecutivo, legislativo y judicial. El Art. 141 frena a los tres poderes entre sí)}
        \item El sujeto de la soberanía es el estado: 
        \begin{itemize}
            \item Doctrinas de la soberanía Absolutista, popular y soberanía nacional.
        \end{itemize}
    \end{itemize}

    
    \item ¿Cómo se debe ejercer el poder público en Guatemala?
    \begin{itemize}
        \item Art. 152 Poder público: El poder que ejercen las autoridades públicos del estado. Como se transmite la soberanía a alguien para poder ejercer el obierno y administración.
        \item Art. 153 Imperio de la ley: la ley se extiende a todos sean extranjeros o nacionales se aplica la ley Guatemalteca.
        \item Art. 154  Función pública: \textbf{\emph{(Ejemplo: Los juramentos de jurar defender la constitución)}}
        \item Art. 155 Responsabilidad por infracción a la ley: \textbf{\emph{(Ejemplo: si un funcionario comete un delito, por el hecho de ser funcionario se duplica la pena por el delito.)}}
        \item Art. 156 No obligatoriedad de órdenes ilegales: Ningún funcionario es obligado a obedecer algo que no es legal.
    \end{itemize}

    
    \item El fin es el Bien común: 
    \begin{itemize}
        \item Fin de un estado definido en la Constitución: Art. 1 de la constitución. 
        \item Art.1 Constitución de Guatemala
        \item ¿Cómo definirían el Bien Común? \textbf{\emph{Definición: Bien común, es el conjunto de condiciones sociales, económicas, culturales, morales y espirituales para desarrollarnos}}
        \item \emph{\textbf{(Paréntesis:}En el preámbulo de la constitución de Guatemala, se establece que se va a preferir por default el bien común sobre el bien individual.)}
    \end{itemize}

    \item \emph{\textbf{(Paréntesis:}Razón del estado, teoría que sostiene que el individuo está a la disposición del estado. La teoría de derecho natural, Guatemala sostiene esta teoría ya que el estado esta al servicio de el individuo, es al revéz que la razón del estado.)} \emph{\textbf{(Paréntesis:}Libertad de credo la asegura la constitución pero así como no hay idioma oficial la constitución está escrita en español, asi mismo pasa con la religión Art. 36 Libertad de religión)}
    
    \item Orden jurídico
    \begin{itemize}
        \item \textbf{Nos preguntamos:} ¿Qué entienden por orden jurídico? Ultimo elemento posterior es el orden jurídico, el estado esta sujeto a normas internas y externas.
        \item Estado con otros Estados = Normas de Derechos Internacionales
        \item Estado con Gobernados = Normas Jurídicas Internas
    \end{itemize}

    
    \item La necesidad del derecho en un estado.
    \begin{itemize}
        \item No se concibe al Estado sin Derecho ni al  Derecho sin Estado. 
        \item “Un Estado sin poder soberano es inconcebible, y un Estado con poder que no esté limitado por el Derecho, no es Estado sino fenómeno de fuerza”
    \end{itemize} 

    \item \emph{\textbf{(Paréntesis:}El derecho y el estado según mucho sautores son lo mismo pero en este curso se tratarán como diferentes por la razón que es un ordenamiento jurídico.)}
\end{itemize}

\subsection{Noticia: Conflicto Mexico-Guatemala}
\begin{itemize}
    \item Conflictos acerca de tala de arboles en Petén.
    \item El bombardeo de los barcos.
    \item México corta lazos diplomáticos.
    \item Se resuelve el conflicto.
    \item \emph{\textbf{(Paréntesis:}Relacionar con la soberanía Guatemalteca.)}
\end{itemize}

\begin{itemize}
    \item Sealand
\end{itemize}


