\section{¿Qué es el derecho?}
\begin{itemize}
    \item Producto de una manifestación social
    \item \textbf{Nos preguntamos:} ¿Cómo fue la agrupación del ser humano?
    \begin{itemize}
        \item Por sociedad, \emph{la división del trabajo} 
        \item En grecia era la Pólis y en Roma era la Res Pública
    \end{itemize}


    \item \textbf{Nos preguntamos:} ¿Cómo evolucionaron las normas que establecían límites?
    \begin{itemize}
        \item Primero era por temor a los Dioses
        \item Después la gente empezó a ya no creer entonces surgen las normas jurídicas
        \item No bastaron las normas religiosas entonces por eso surgen las normas jurídicas
        \item Viene del derecho natural, en algún momento bastaron las religiosas pero despues al no abstecerse surgen las normas jurídicas.
    \end{itemize} 
\end{itemize}

\section{Elementos del derecho:}
\begin{enumerate}
    \item Existencia de las normas o reglas preestablecidas:
    \begin{itemize}
        \item No se puede acusar a las personas por un delito que no está preestablecida, por razones de justicia por que no sería practico andar acusando a la gente por leyes que no existen aún.
        \item \emph{\textbf{(Paréntesis:}Femicidio es matar por el simple hecho de ser mujer.\textbf{)}}, \emph{\textbf{(Paréntesis:}Parrecidio, se matan a las personas parientes\textbf{)}}, \emph{\textbf{(Paréntesis:}infanticidio, es de 4 días o menos se mata a un infante.\textbf{)}}
        \item Art. 5 Const. GT todas las personas pueden hacer todo lo que no está prohibido.
    \end{itemize}

    \item Las normas son de aplicación obligatoria.
    \begin{itemize}
        \item ¿Cómo se diferencias de las morales? A diferencia de las religiosas no son por temor o por que subjetivamente se teme a lo que pueda pasar, las normas jurídicas son de aplicación obligatoria.
        \item Si no se cumple una norma el estado entra con su coacción.
        \item \textbf{\emph{Caso ``Adultero \& el conyugato": solía ser un delito, después se prohibió, y el conyugato se prohibió.}}
    \end{itemize}
    
    \item Respaldo del poder coactivo del estado.
    \begin{itemize}
        \item Se aplican las leyes y de no cumplirse se aplica la coacción.
        \item En GT la pena máxima es de 50 años.
    \end{itemize}
\end{enumerate}

\section{}

\section{La omnipresencia del Derecho}
