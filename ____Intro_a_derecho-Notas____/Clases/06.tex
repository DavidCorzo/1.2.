\section{Caracteres del derecho}
\begin{itemize}
    \item \textbf{Nos preguntamos:} ¿por qué el derecho es un fenómeno social?
    \begin{itemize}
        \item por que es una obra humana,  es creado por el hombre, dicta el comportamiento para poder vivir en sociedad.
        \item El hombre es el que lo crea, para funcionar tiene que crear normas.
    \end{itemize}
    
    \item \textbf{Nos preguntamos:} ¿por qué el derecho es un fenómeno cultural?
    \begin{itemize}
        \item Si se impusiera una norma que funciona en otro país no implica que vaya funcionar por la razón cultural
    \end{itemize}

    \item \textbf{Nos preguntamos:} ¿por qué el derecho es un fenómeno Histórico?
    \begin{itemize}
        \item Por que el derecho cambia a través del tiempo, cosas que funcionaban en un tiempo y ahora ya no o vise versa.
        \item Se cambia por que en algún momento se consideró importante.
        \item \textbf{Nos preguntamos:} ¿qué evoluciona más rápido la sociedad o el derecho? \emph{(\textbf{Respuesta}:es la sociedad, lo ideal es que el derecho evolucione a la par de la evolución de la sociedad })
        \item Puede cambiar retrocediendo o progresando. \textbf{\emph{(Ejemplo: El derecho involucionó en el caso de Hitler, el ejemplo del voto por ejemplo)}}
        \item \emph{\textbf{(Paréntesis ``Hábeas corpus exhibición personal'':} se puede emitir una solicitud al juez para determinar la legalidad del derecho \textbf{)}}
    \end{itemize}

    \item \textbf{Nos preguntamos:} ¿por qué el derecho es un fenómeno político? Sí por que expresa relaciones con el poder, por ejemplo la coacción.
    
\end{itemize}

\section{Acepciones del derecho}
\begin{itemize}
    \item Problemas de ambigüedad:
    \begin{itemize}
        \item Es confuso, significados diferentes en los que la terminología se refiere a muchas cosas y depende del contexto, \textbf{\emph{(Ejemplo: mora, fruta o multa por retraso a obligaciones, ejemplo la alimentación y la ambigüedad de eso, otro ejemplo ``competente''.)}}
    \end{itemize}

    
    \item Vaguedad:
    \begin{itemize}
        \item Se desconoce el alcance, falta claridad.
    \end{itemize}

    
    \item Emotividad: 
    \begin{itemize}
        \item Tiene una carga emotiva, relacionado con las emociones, \textbf{\emph{(Ejemplo: la justicia)}}
    \end{itemize}

    
    \item Distintas acepciones del ``derecho'':
    \begin{itemize}
        \item Tiene varios significados:
        \begin{enumerate}
            \item Derecho como facultad o derecho subjetivo:
            \begin{itemize}
                \item ``Tengo \underline{\textbf{derecho}} a algo''
            \end{itemize}
            
            \item Derecho como ciencia:
            \begin{itemize}
                \item Se usa científicamente en el área académica, es estudio como una ciencia .
            \end{itemize}
            
            \item El derecho como ideal ético o moral de \underline{\textbf{Justicia}}: 
            \begin{itemize}
                \item Relacionados con lo justo según lo moral o ética, se usa al derecho como lo que ``es justo''.
            \end{itemize}
            
            \item El derecho como la norma:
            \begin{itemize}
                \item Derecho objetivo, es las referencias a las normas escritas.
            \end{itemize}
        \end{enumerate} 
        
        De dónde se deriva:
        \begin{enumerate}
            \item De todo derecho objetivo se derivan los subjetivos.
        \end{enumerate}
        
        En otros lenguajes como el inglés:
        \begin{itemize}
            \item Derecho subjetivo = right 
            \item Derecho objetivo = law
        \end{itemize}

        Tesis de la indefinición: es imposible de definirlo en su totalidad, solo parcialmente. 

        
        \item Derecho subjetivo:
        \begin{itemize}
            \item El derecho subjetivo público:
            \begin{itemize}
                \item Cuando el estado se mete en los derechos. \textbf{\emph{(Ejemplo: subsidios)}}
            \end{itemize}
            
            \item El derecho subjetivo privado:
            \begin{itemize}
                \item Cuando el estado no se mete, por ejemplo cuando se da una cobra venta. Ojo, el estado puede tener un carácter de individuo particular.
            \end{itemize}
            
            
        \end{itemize}
    \end{itemize}
        

    Clases de derecho objetivo:
        \begin{itemize}
            \item El derecho objetivo orientado al derecho natural:
            \begin{itemize}
                \item Orientado a aquello que el humano tiene como intuición de qué es lo bueno, el derecho natural no cambia y permanece constante.
                \item Evolución de el derecho natural:
                \begin{itemize}
                    \item Época antigua $\rightarrow$ Naturaleza del hombre
                    \item Época cristiana $\rightarrow$ Dios (autoridad suprema)
                    \item Época moderna $\rightarrow$ El D. natural se origina en la razón.
                    \item Renacimiento del derecho natural $\rightarrow$ Reconocimiento de derechos humanos
                \end{itemize}

                
                \item Diferencias: \newline 
                \begin{tabular}{ | p{5cm} | p{5cm} | } 
                 \hline
                \textbf{El derecho natural} & \textbf{El derecho positivo} \\
                Inherente al hombre & Se crea autoridad competente \\ 
                Inmutable & Mutable \\ 
                Universal & Es reflejo de la cultura \\ 
                Justo & No siempre es justo \\ 
                Incoersible & Coercible \\ 
                 \hline
                \end{tabular}
                
            \end{itemize}

            \item El derecho objetivo orientado al derecho positivo:
            \begin{itemize}
                \item No es creado por el hombre, el derecho positivo no es igual en todos los tiempos.
            \end{itemize}

            
            \item Iuspositivismo:
            \begin{itemize}
                \item Admite la distinción del derecho natural y el positivismo.
            \end{itemize}
        \end{itemize}
        
        \item Clasificación del derecho positivo:
        \begin{itemize}
            \item Por su grado de efectividad:
            \begin{enumerate}
                \item Vigente
                \item No vigente:
                \begin{itemize}
                    \item Actual, no efectiva ahorita, no van a limitar nada innecesario (ley de orden público, \textbf{\emph{(Ejemplo: el asesinato de militares causó una limitación en los derechos de la constitución, estado de sitio)}})
                    \item Histórico: derogado 
                \end{itemize}
            \end{enumerate}
            
            \item Por su forma de manifestarse:
            \begin{itemize}
                \item Escrito: plasmado en documentos.
                \item No escrito: costumbre (derecho consuetudinario) \textbf{\emph{(Ejemplo: derecho indígena)}}
            \end{itemize}

            
            \item Por materia que regula:
            \begin{itemize}
                \item Derecho público: interviene el estado con poder soberano o como institución pública. 
                \item Derecho privado: entre el particulares. 
                \item \textbf{Nos preguntamos:} ¿puede intervenir el estado en una relación de derecho privado? \emph{\textbf{La respuesta a esta esta pregunta es: }...}
            \end{itemize}
        \end{itemize} 







    \end{itemize}
    

\section{Teorías que explican la división del derecho en público y privado}
\begin{itemize}
\item Teoría del interés:
\begin{itemize}
    \item Público: interes colectivo
    \item Privado: interes particular 
\end{itemize}

\item Teoría del organo:
\begin{itemize}
    \item Público: estado interviene 
    \item Privado: estado no interviene
\end{itemize}

\item Teoría del interés:
\begin{itemize}
    \item Privado: relaciones de coordinación, es de peers.
    \item Público: relaciones de subordinación o supraordinación
\end{itemize}
\end{itemize}

\section{Derecho público}
\textbf{Irrenunciable e innomidicable:} No nos queda otra que obedecer, no es renunciable ni modificable.
\begin{itemize}
    \item Derecho Constitucional (Derechos fundamentales): constitución 
    \item Derecho Fiscal (SAT): impuestos
    \item Derecho Administrativo (Municipalidades): sacar pasaporte, licencia, permisos de construcción etc.
    \item Derecho Penal (Delitos): sansiones y coacción.
    \item Derecho Procesal (Procesos judiciales): normas establecidas para procesos legales.
    \item Derecho Internacional Público (Tratados y convenios internacionales TLC): CONVEMAR, TLC's, etc.
    \item Derecho laboral (relación con empleados): \textbf{es un área gris} puede pactarse con el patrono y el empleado qué suma de dinero hay que pagarle siempre y cuando no sea menos del salario mínimo, tengo libertad de pagarle cualquier suma arriba del mínimo. Es entre particulares pero por la constante intervención del estado es un derecho público. Proteccionismo de parte del estado al trabajador. \newline \textbf{\emph{(Ejemplo: fallecimiento de trabajador con la esposa conflicto con los Q1,000)}}, \textbf{\emph{(Ejemplo: premisa que todos los trabajadores pueden joder a los patronos sin evidencia.)}}, \textbf{\emph{(Ejemplo: Ventana económica, todo lo adicional que le da el patrono al trabajador hace mas caro cualquier cosa, puede exigir 30\% más en su indemnización.)}}, \textbf{en GT se considera derecho público}. \textbf{Es frecuente el abuso del trabajador al patrono}.
\end{itemize}

\section{Derecho privado}
\textbf{Renunciable y modificable:} sí se puede renunciar.
\begin{itemize}
    \item Derecho Civil (Contratos): contratos. Vendo la casa por que no la uso no por fin de lucro. \textbf{\emph{(Ejemplo: viaje con cosas que parecen ser de comercio pero son civil.)}}
    \item Derecho Mercantil (Fin de lucro):tienen fin de lucro, ejemplo: títulos de crédito, sociedades. 
    \item Derecho Internacional Privado (Legislación aplicable): casarse en otro país, con una persona extranjera, cambio de nombre, cambio de nacionalidad.
\end{itemize}

