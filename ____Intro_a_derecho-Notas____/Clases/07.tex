\section{``ORDENES NORMATIVAS''}
\begin{itemize}
    \item ¿A qué se encuentra sujeta la conducta del ser humano? lo social, cultural, costumbre.

    \item ¿Qué regía en el orígen de las civilizaciones? al principio por la religión pero después por normas de carácter religioso al caer la religiosidad.
    
    \item Distinción por dos factores: 
    \begin{itemize}
        \item La pérdida del sentimiento religioso;
        \item La complejidad de la vida social
    \end{itemize}
    
    \item ¿Existe relación entre la religión, moral y el Derecho?
   Ejemplo: 1866 C. Civil (Causas de ingratitud): \textbf{\emph{(Ejemplo: si dono y un día necesita la donación o la fundación me hace daño, se puede revocar la donación)}} \textbf{\emph{(Ejemplo: Honrar a los padres, parece un mandato religioso.)}}\emph{\textbf{(Paréntesis ``adopciones'':}los hijos adoptivos rompen relación con padres biológicos\textbf{)}}
    \item La conducta del ser humano se encuentra regulada por diferentes órdenes normativos:  
    \begin{enumerate}
        \item Derecho: 
        \item Moral: pegarle a los hijos 
        \item Religión: ejemplo de mandamiento
        \item Convencionalismos Sociales: costumbres
    \end{enumerate}
\end{itemize}

\subsection{Diferentes tipos de normas}
El 8 de Noviembre del 2007 Dennis Lindberg, un joven de 14 años fue diagnosticado con Leucemia en Estados Unidos. Los médicos recomendaron hacerle varias transfusiones de sangre para que el menor pudiera sobrevivir tratamientos de cáncer que podían salvarle la vida.
Siendo Dennis un joven practicante de sus creencias se negó a que le hicieran las transfusiones, tenía la convicción de que las mismas violaban la ley de Dios. 
Sus padres no sólo trataron de convencerlo para que aceptara las transfusiones sino además trataron de convencer a un juez de la Corte Suprema a que obligara a Dennis por mandato legal a realizarse dichos procedimientos.
Sin embargo hubo quienes apoyaron su decisión; en principio su tía, quien era su tutora legal y por otro quienes tenían las mismas convicciones calificaron de admirable su decisión estableciendo: “Está expresando su propia fe.” 
Este caso obligó al juez a un difícil acto de equilibrio: Sopesar el derecho de Lindberg a tomar sus propias decisiones y la necesidad de proteger al menor. Después de analizar el caso el  juez del Tribunal Superior negó la moción presentada por el gobierno del estado para obligar a Dennis a tener una transfusión de sangre, estableciendo que a pesar de su edad, se le consideraba como “menor maduro” que significa en Estados Unidos que era lo suficientemente sensato como para tomar las decisiones relacionadas con su tratamiento. 
Después de varios días de luchar, Dennis Lindberg murió al comienzo de la tarde  del miércoles  28 de Noviembre del 2007 en  el Centro Médico Regional.
\newline 
\textbf{Nos preguntamos:} ¿Qué tipos de normas  se relacionan con el caso? \emph{\textbf{La respuesta a esta esta pregunta es: }derecho, moral, religioso; ponerse en la perspectiva de los médicos, el \textbf{juramento} hipocrático, todas las variantes del caso \textbf{la religión, la cultura, so médicos y sus juramentos; en este caso no prevaleció el derecho}.}


\subsection{Diferentes tipos de normas…..}  
En París, Francia, en septiembre del 2004 entró en vigor una ley que prohíbe los signos religiosos ostensibles en las escuelas públicas. La ley generó una enorme polémica antes de ser adoptada. El responsable de Educación, Francois Fillon subrayó que el laicismo es un principio moderno de tolerancia y que la ley fomenta la fraternidad entre los niños sin atentar contra la libertad religiosa. 
\newline 
\textbf{Nos preguntamos:} ¿Qué tipos de normas  se relacionan con el caso? \emph{\textbf{La respuesta a esta esta pregunta es: }religioso, moral, derecho, las decisiones relacionados con derecho se consideran todos los factores como el religioso o la moral etc.}


\section{DISTINTOS ÓRDENES NORMATIVOS:}
\begin{itemize}
\item MORAL:  Emana de la conciencia, persiquen el bien individual mediante la práctica de las virtudes. Le interesa el individuo con relación así mismo, la intención.\textbf{Emana de la consciencia de nosotros.}
   
\item RELIGIÓN: Conjunto de creencias reveladas por un ser supremo. Le interesa la actuación interna del individuo, la Salvación. \textbf{La impone un ser supremo.}
 
\item USO O CONVENCIONALISMO  SOCIAL: Son exigencias colectivas que rigen la moda, la cortesía, la buena educación y las buenas costumbres. Le interesa el comportamiento  externo. \textbf{La impone la sociedad,} estos casos pesan al derecho \textbf{\emph{(Ejemplo: alguien de homicidio culposo, hay circunstancias atenuantes si asiste a la iglesia por ejemplo.)}}

\item DERECHO (Norma jurídica):  Le interesa la conducta externa, tiene como propósito fundamental lograr la convivencia pacífica y justa. \textbf{La impone el estado.}
\end{itemize}


\section{CARACTERISTICAS DE LAS NORMAS DE CONSUCTA:}
BILATERALIDAD  y  UNILATERALIDAD:  
\begin{itemize}
\item Bilateral $\implies $ impone deberes y concede facultades a la vez.
\item Unilateral $\implies $  sólo impone deberes.
\end{itemize}

EXTERIORIDAD e INTERIORIDAD: 
\begin{itemize} 
\item Exterior $\implies $ se toma en cuenta el hecho de que la conducta externa se   adecúe al deber establecido por la norma.  
\item Interior $\implies $ cuando lo que se valora es la intención.

\item Análisis: el derecho es una mezcla de las dos.
\end{itemize}


COERCIBILIDAD e INCOERCIBILIDAD:  
\begin{itemize}
    \item Coercible$\implies $ cuando exista la posibilidad de poder exigir el cumplimiento forzado por parte del Estado.   
    \item Análisis: el derecho es cohersible
\end{itemize}


AUTONOMÍA y HETERONOMÍA:  
\begin{itemize}
    \item Autónoma$\implies $ cuando el sujeto que debe cumplirla lo reconoce voluntariamente como válida. (autolesgislación).  
    \item Heterónoma $\implies $ cuando es creada por un sujeto distinto del destinatario.
    \item Análisis: el derecho es heterónomo.
\end{itemize}


\section{Características de las normas de conducta}
\begin{center}
\begin{tabular}{ | p{5cm} | p{5cm} | p{5cm} | }
 \hline
 \\
 \hline
\end{tabular}
\end{center}
