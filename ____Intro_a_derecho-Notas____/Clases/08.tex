\section{Fuentes del derecho}
\begin{itemize}
    \item Las fuentes son las razones que se necesitan un norma por ende se crea.
    \item Causas del derecho que dan origen al derecho
    \item 
\end{itemize}

\section{Trabajo en grupo}
\begin{itemize}
    \item Fuentes históricas:
    \begin{itemize}
        \item \emph{\textbf{Definición de ``fuentes históricas de derecho":} ddel derehco comprende todos los documentos, monumentos, escultura, etc. en cuyo a quedado plazmado el derecho a través del tiempo.}
        \item \emph{\textbf{Importancia de ``fuentes históricas del dereho":} Sirve para evolucionar y comparar la legislación actual, al igual que derrogar leyes cuya necesidad no continúe presente.}
        \item \textbf{\emph{(Ejemplo: Constitución federeal de 1824, Constitución de GT 1825)}}
    \end{itemize}
    
    \item Fuentes reales o materiales:
    \begin{itemize}
        \item \emph{\textbf{Definición de ``fuentes del derecho":} Factores o circunstancias que provocan la aparición y determinación del contenido de las normas jurídicas.}
        \begin{enumerate}
            \item Factores reales: Las normas obedecen a múltiples causas. Ejemplo: políticas , económicas, sociales  y biológicas .
                \begin{itemize}
                    \item Datos biológicos: toma en cuenta los hechos básicos de la vida: \emph{nacimiento, merte, diferencia de sexo ,edad, etc.}
                    \item Datos Económicos: considera la realidad económica que influye en la norma que produce.
                    \item Datos Sociales: Realidad social que se considera en la norma que produce este factor.
                    \item Datos Políticos: observa el fenómeno del poder, la diferencia entre gobernantes y gobernados \textbf{\emph{Ejemplo: el antejuicio que contempla el juicio de la alegación de un crimen que produzca que no esté en su cargo por un tiempo por ende es un factor que produce la norma del antejuicio. }}
                \end{itemize}
            \item Factores ideales: Respetados en todo tiempo y lugar (Respeto a la vida)
                \begin{itemize}
                    \item Directas:
                    \item Indirectas:
                \end{itemize}

            \item Factores racionales: Forman parte del Derecho Natural, exigencias universales que el jurista no puede ignorar en la formulación del Derecho.

        \end{enumerate}
        
        \item Ojo que el \textbf{derecho mercantil} no es fuente del derecho real o  material, la realidad económica es la fuente pero el derecho mercantil es el producto o la respuesta a la necesidad de considerar de la realidad económica.
    \end{itemize}

\end{itemize}


\section{TERMINAR POR PRESENTACIÓN}
