\section{Elementos de costumbre}
\begin{itemize}
    \item \emph{\textbf{Definición de ``Elemento objetivo":} Duración y repetición de una conducta.}
    \item \emph{\textbf{Definición de ``Elemnto subjetivo":} Opinión generalizada respecto a la obligatoriedad de esa forma de conducta.}
    \item Diferencia entre el hábito y la costumbre:
        \begin{itemize}
            \item comportamiento repetido $\neq$ norma
            \item Un hábito \textbf{no tiene }el elemento subjetivo.
            \item No toda la regla de conducta llega a convertirse en costumbre jurídica.
        \end{itemize}
    
    \item Requisitos de costumbre:
        \begin{enumerate}
            \item Generalidad: conducta común en una sociedad.
            \item Largo uso: conducta constante a través del tiempo.
            \item Notoriedad: tiene aceptación esta costumbre ante las autoridades y la sociedad.
        \end{enumerate}
    
    \item \textbf{Derecho consuetudinario:} es el derecho de la costumbre. hay clases:
        \begin{enumerate}
            \item Clase de la costumbre interpretativa: es de \textbf{interpretación}.
                \begin{itemize}
                    \item Si algo en un contrato no está del todo claro se interpretará con tendencia concluir con forme las costumbres de la sociedad en cuestión.
                    \item Hay normas pero no están del todo claro.
                \end{itemize}
            
            \item Costumbres supletoria: Surge por ausencia de ley y no opera en el ámbito penal.
                \begin{itemize}
                    \item Ayuda a llenar un vacío que las normas no abarca.
                    \item En este caso no hay norma entonces se usa la costumbre para \textbf{llenar lagunas legales}.
                    \item Ejemplo de la cuerda en Petén y en Esquintla. Relacionar esto con economía la falta de empatía de la ayuda social.
                \end{itemize} %pero todo lo que no es prohibido es legal, derecho indigena
            
            \item Costumbre derogatoria: se opone a las normas legales y no es aceptada en la sociedad.
                \begin{itemize}
                    \item un ejemplo es del derecho indígena 
                    \item Costumbre que viola una norma.
                \end{itemize}
        \end{enumerate}
    La ley va por encima de la costumbre, en GT solo se acepta la interpretativa y la supletoria.
    
    \item Ventajas y desventajas de la costumbre frente la legislación:
        \begin{itemize}
            \item Ventajas:
                \begin{enumerate}
                    \item Está sincronizada con el ritmo de la evolución de la sociedad.
                    \item Son relgas y prácticas y eficaces.
                    \item Es más democrática, por que la comunidad participa en su elaboración.
                \end{enumerate}
            
            \item Desventaja:
                \begin{enumerate}
                    \item Es mas fácil de aprobar.
                    \item Se necesita estudiar la costumbre de la sociedad para legislar coherentemente.
                \end{enumerate} % una norma puede adquirir validez jurídica?
        \end{itemize}
\end{itemize}

\section{Legislación}
\begin{itemize}
    \item Concepto de legislación:  CONTINUAR LEGISLACIÓ!!!!!!
\end{itemize}
