\section{El orden jurídico}
\begin{enumerate}
    \item \textbf{Nos preguntamos:} ¿qué es? \emph{\textbf{La respuesta a esta esta pregunta es: }es el conjunto de normas y leyes de un estado}
    \item Caracteres del ordenamiento jurídico
    \begin{itemize}
        \item Unidad - 
            \begin{itemize}
                \item Las normas jurídicas están relacionadas entre sí en una estructura jerárquica, las inferiores deben de acoplarse a las superiores.
                \item Deben de tener validez formal y materialidad.
                \item Kelsen (iusnaturalista) - la pirámide invertida - noción que todas las normas se deben de regirse a las superiores hasta llegar a la constitución, a Kelsen utiliza el proceso de la creación de normas como única justificación, \textbf{Nos preguntamos:} ¿de donde se originan las primeras leyes?. 
            \end{itemize}
        \item Plenitud - 
            \begin{itemize}
                \item 
            \end{itemize}
        \item Ceoherencia:
            \begin{itemize}
                \item Compatibilidad de las normas. 
                \item En un ordenamiento jurísdico no debe de permitir antinomias ni contradicciónes.
                \item \textbf{Nos preguntamos:} ¿qué pasa cuando una norma inferior contradice a una norma superior? La inferior se deroga.
                \item La doctrina de lps tres criterios para la solución de antinomias.
                \begin{itemize}
                    \item Jerarquico: se aplica la norma superior.
                    \item Cronológico: seapliza la norma posterior.
                    \item Especialidad: se aplica la norma especial.
                \end{itemize}
            \end{itemize}
        
        \item Plenitud:
            \begin{itemize}
                \item Se sostiene que todo conflicto tiene solución jurídica.
                \item Laplenitud absoluta: Existen \textbf{normas} para solucionar todos los problemas que se presentan.
                \item Plenitud relativa: existen \textbf{soluciones} a todos los problemas que se presentan.
                \item \textbf{Nos preguntamos:} ¿Existen las lagunas en la ley o del derecho? Es en la ley, por que el derecho es más amplio que la ley, si en la ley hay un espacio vacío acudo a otros lugares para compensar ese espacio vacío.
            \end{itemize}
    \end{itemize}
\end{enumerate}

\section{Las jerarquías normativas}
\subsection{Características}
\begin{itemize}
    \item Tiene que actuar sistemáticamente, coordinadamente, todas tienen que pertenecer a un orden sistemático sin importar qué tipo de norma sea, uno debe interpretar de acuerdo a todas las normas.
\end{itemize}
%%%%%%%%%%%%%%%%%%%%%%%%%%%%%%%%%%%%%%%%%%%%%%%%%%%%%%%%%%%%%%%%%%%%%%%%%%%%%%%%%%%%%%%%%%%%%%%%
\subsection{Se logra solamente si: }
\begin{itemize}
    \item Por el principio de jerarquía.
    \item y ademas cuando se cumple la validez tanto a nivel \textbf{formal} como \textbf{material}. Uno de los métodos que hay para neutralizar a las normas que contradicen a normas superiores es la inconstitucionalidad.
\end{itemize}
%%%%%%%%%%%%%%%%%%%%%%%%%%%%%%%%%%%%%%%%%%%%%%%%%%%%%%%%%%%%%%%%%%%%%%%%%%%%%%%%%%%%%%%%%%%%%%%%
\subsection{Jerarquía normativa: Pirámide invertida de Kelsen:}
    \begin{etaremune}
        \item Constitución: es la más pesada, es un conjunto de normas.
        \item Leyes ordinarias
        \item Normas reglamentarias
        \item Normas individualizadas
    \end{etaremune}
%%%%%%%%%%%%%%%%%%%%%%%%%%%%%%%%%%%%%%%%%%%%%%%%%%%%%%%%%%%%%%%%%%%%%%%%%%%%%%%%%%%%%%%%%%%%%%%%
%%%%%%%%%%%%%%%%%%%%%%%%%%%%%%%%%%%%%%%%%%%%%%%%%%%%%%%%%%%%%%%%%%%%%%%%%%%%%%%%%%%%%%%%%%%%%%%%

\section{La pirámide de GT:}
\begin{etaremune}
    \item Constitución 175, 204 const. Tratados internacionales en materia de derechos humanos: La constitución la hace la asamblea nacional constituyente,\textbf{es diferente que el congreso}, fue armada por diputados pero no son lo mismo que los diputados de la asamblea nacional constituyente. 
    \begin{itemize}
        \item La constitución, norma suprema que emana poder del pueblo.
        \item Finalidad: crear la parte orgánica (órganos del estado) esta es la primera parte, la segunda es la parte dogmática.
        \item Diferencias... 
            \begin{center}
            \begin{tabular}{ | p{5cm} | p{5cm} | }
                \hline
                Asamblea Nacional Constituyente & Asamlea legislativa \\ 
                \hline
                Se conoce como asamble nacional constituyente (parlamento) & Se conoce tambien como congreso \\
                Es temporal & No es temporal pero no dura para siempre \\ 
                Crean la constitución y leyes superiores (normas superiores) & Crea o reforma (también deroga), normas inferiores a la constitución, inferiores a las leyes constitucionales. \\ 
                \hline
            \end{tabular}
            \end{center}
        
        \item Hay artículos no reformables en la constitución,hay gente que quiere derogar el artículo que no permite derogar, esto no se puede hacer ya que se conoce como el fraude de ley.
        \item Los tratados internacionales, \emph{\textbf{Ejemplo:}cuando GT firmó y se comprometió a no aplicar la pena de muerte, y posteriormente a la muerte de la sra. Botrán se consideró aplicar la pena de muerte}
            \begin{itemize}
                \item Tratados de derecho humanos
                \item Tratados internacionales (\emph{\textbf{Observación: }si un tratado internacional contradice la constitución prevalece la constitución.})
                \item  \textbf{Nos preguntamos:} ¿qué es el derecho interno? se considera de la segunda en adelante. 
                \item \emph{\textbf{Definición de ``bloque constitucional":} es una permisión a hacer una excepción, por una parte se puede ver como una extensión.\emph{\textbf{Ejemplo:}convenio 169}}
                \item Leyes constitucionales: son leyes que regulan materia constitucionalidad, creadas por la asamblea nacional constituyente. 
            \end{itemize}
    \end{itemize}
\end{etaremune}



