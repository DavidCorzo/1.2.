\section{Fuentes del derecho}
\begin{enumerate}
    \item \emph{\textbf{Definición de ``Doctrina":} estudios de carácter científico que los juristas realizan acerca del derecho con la intención de interpretar normas o con el propósito puramente teórico}.
        \begin{itemize}
            \item Una doctrina nos fuente directa del derecho es una fuente formal indirecta
            \item La doctrina se usa mayoritariamente para estudiar el derecho y las normas derivadas de él. 
        \end{itemize}
    
    \item Principios generales del derecho:
        \begin{itemize}
            \item Postulados del derecho natural que comprenden la columna vertebral de la legislación en ina comunidad.
            \item Funciones: sirven para complementar al derecho aun que el derecho positivo no absorban todo lo del derecho natural.
        \end{itemize}
    
    \item \emph{\textbf{Definición de ``Considerandos":} consideraciones preliminares para justificar la existencia de una ley.}
        \begin{itemize}
            \item El derecho consuetudinario se utiliza en el \emph{\textbf{Ejemplo: }de Elmer Palmer}.
            \item En GT, ART. 924, legisla herencia cuando se asesina al heredero ``ley de ingratitud''.
            \item \emph{\textbf{Recordar lo siguiente: }Amparo, ley de exhibición personal.}
        \end{itemize}
    
    \item Declaración unilateral de voluntad:
        \begin{itemize}
            \item \emph{\textbf{Definición de ``declaración unilateral de voluntad":} actuación jurídica por la cual un sujeto de derecho establece una regla, crea una nueva norma.}
            \item \emph{\textbf{Ejemplo: }Cuando se ofrece recompensa por encontrar un perrito.}, \emph{\textbf{Ejemplo: }La recompensa de oferta al público de llamar a la policía por información en cambio de la recompensa}, \emph{\textbf{Ejemplo: }El testimonio que deja de una persona.}
            \item Puede ser de carácter público o privado:
                \begin{enumerate}
                    \item Público:Estado manifiesta su voluntad.
                    \item Privado: Particular que manifiesta su voluntad.
                \end{enumerate}
        \end{itemize}
    
    \item Ejemplos de declaraciones unilaterales de voluntad de carácter público: 
        \begin{itemize}
            \item Reglamentos 
            \item Reglamentos de necesidad 
            \item Decreto gubernativo 
            \item Decreto ley 
            \item Acuerdos gubernativos 
        \end{itemize}
    
    \item Ejemplos de declaraciones unilaterales de voluntad de carácter privado: 
        \begin{itemize}
            \item Disposiciones del testador 
            \item Reconocimiento de hijos 
            \item Aceptación de la herencia, rechazo de la herencia
            \item Renuncia de un derecho 
            \item Promesa de recompensa 
            \item Oferta al público.
        \end{itemize}
        \begin{itemize}
            \item \emph{\textbf{Ejemplo: }hijos que habían sido intercambiados y mediante pruebas ADN se establece que los niños se quedaran donde están en las familias donde están y no las biológicas.}
            \item \emph{\textbf{Ejemplo: }La deuda que quedó incobrable del abuelito de la catedrática}.
        \end{itemize}
    
    
    \item Fuente directa:
        \begin{itemize}
            \item \emph{\textbf{Definición de ``Declaración bilateral de voluntad":} doctrina le llama al contrato negocio jurídico bilateral y tiene como fin construir, modificar o extingir una relación de naturaleza patrimonial 1517 C. Civil}.
            \item Requisitos del negocio jurídico:
                \begin{enumerate}
                    \item Capacidad legal: básicamente ser mayor de edad, o ser representado por un mayor de edad.
                    \item Consentimiento que no adolezca de vicio: Visio del consentimiento, error, dolo, violencia, intimidación, coacción.
                    \item Objeto lícito: Por mas que yo sea un mayor de edad no puedo venderle a alguien la luna, o una estrella, o el palacio nacional, o venta de marihuana, el objeto en cuestión debe de ser lícito.
                \end{enumerate}
        \end{itemize}
        
        
        
\end{enumerate}
