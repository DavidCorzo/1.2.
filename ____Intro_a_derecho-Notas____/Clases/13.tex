\section{Presentaciones apriori}
None.

\section{La constitución}
\begin{enumerate}
    \item Derechos individuales:  
        \begin{itemize}
            \item Derechos que gozan todos los individuos como particulares, no pueden ser restringidos por los gobernantes, salvo casos muy especiales como los de estado de sitio o emergencias nacionales.
            \item Título I, capítulo I, Constitución política de Guatemala.
            \item \emph{\textbf{Observación: }Estos derechos no son absolutos}.
        \end{itemize}
    
    \item Derechos sociales: 
        \begin{itemize}
            \item Tienen el objetivo primordial de garantizar el bienestar económico, el acceso al trabajo, la educación, la cultura; esto de tal manera que segure el desarrollo de los seres humanos y las comunidades para una vida digna.
            \item Título I, Capítulo II, Constitución política de Guatemala
            \item \emph{\textbf{Observación: }Uno puede poner un amparo (es como una intervención por ilegalidad de ley) a cualquier hora a cualquier día, esto es bastante amplio.}
            \item \emph{\textbf{Observación: }Diferencias entre individuales y sociales (parte dogmática), los individuales son derechos de particulares, los derechos sociales son derechos en grupo, se les delega la característica de derechos programáticos por que se le van atribuyendo en función de la capacidad del estado que el estado los cumpla.} 
        \end{itemize}
    
    \item Derecho cívico y políticos: 
        \begin{itemize}
            \item Garantizan la capacidad del ciudadano para particupar en la vida cívil y política del estado, en condiciones de igualdad y sin discriminación.
            \item Derechos cívicos vs. derechos individuales:  los derechos individuales los tienen todas las personas, los derechos cívicos los tienen solo los mayores de edad.
            \item Artículos: 
                \begin{itemize}
                    \item Art. 135 Const. 
                    \item Art. 136 Const.
                    \item Los tienen todos los ciudadanos Art. 147.
                \end{itemize}
            \item \textbf{Nos preguntamos:} ¿Qué pasaría si un presidente no quiere ceder el cargo cuando ya caduque su tiempo? \emph{\textbf{La respuesta a esta pregunta es: }Dado que esto es un delito, lo primero que pasa es que el congreso lo desconoce.} \emph{\textbf{Ejemplo: }Cuando Maduro decide no ceder el cargo a Juan Guiado es asunto inconstitucional por que Maduro está ejerciendo un cargo que viola la constitución.}
        \end{itemize}
    
    \item Limitación de los derechos constitucionales:
        \begin{itemize}
            \item Art. 138 Const.
            \item Art. 139 Const.
            \item Podrá cesar la plena vigilancia de los derechos en caso de:
                \begin{itemize}
                    \item Invasión del territorio
                    \item Perturbación grave de la paz
                    \item Actividades contra la seguridad del estado
                    \item Calamidad pública 
                \end{itemize}
                
            
            \item Derechos que se pueden limitar:
                \begin{itemize}
                    \item Art. 5: Libertad de acción
                    \item Art. 6: Detención legal
                    \item Art. 9: Interrogatorio a detenidos o presos 
                    \item Art. 26:Libertad de locomoción
                    \item Art. 33: Derecho de reunión y manifestación 
                    \item Art. 35: 1er párrafo: Libertad de emisión del pensamiento
                    \item Art. 38: 2do párrafo: Tenencia y portación de armas 
                    \item Art. 116: 2do párrafo: Huelga 
                    \item \emph{\textbf{Observación: }El asunto de la USAC y sus desordenes, \textbf{Nos preguntamos:} ¿podría entrar la ley a poner orden?, podría si se da el caso justificado sí, pero por razones sociales no se hace.} 
                \end{itemize}
        \end{itemize}


        
        \item   Procedimiento: 
            \begin{itemize}
                \item +++
            \end{itemize}

        
        
        \item Ley de orden público: Se justifica la limitación de derechos en las leyes de orden público.
            \begin{enumerate}
                \item Estado de prevención 
                \item Estado de alarma 
                \item Estado de calamidad pública 
                \item Estado de sitio 
                \item Estado de guerra
            \end{enumerate}
            \emph{\textbf{Observación: }Supongamos que tenemos un negocio en Izabal, y hay un estado de sitio si tengo un guardia con arma mejor que no la porte dado a que el estado de sitio se limitan derechos.s }
            
        
\end{enumerate}

