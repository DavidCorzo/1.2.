\section{Derecho sociales}
\begin{enumerate}
    \item Buscan proteger los intereses de colectividades en casos como la discriminación, etc.
    \item Estos son derechos programáticos.
    \item Las personas van a tener estos derechos por el simple hecho de ser parte de una sociedad.
    \item Busca ( y esto es controversial a la ideología de la UFM ) proteger ante la desigualdad ante la economía, salud, etc. por ejemplo por medio de programas sociales.
    \item \emph{\textbf{Ejemplo: }La bolsa segura: consistía en proporcionar alimentos por vivir en condiciones escasas, y Q300 por cada hijo si vivis en condiciones escasas.}
\end{enumerate}
%%%%%%%%%%%%%%%%%%%%%%%%%%%%%%%%%%%%%%%%%%%%%%%%%%%%%%%%%%%%%%%%%%%%%%%%%%%%%%%%%%%%%%%%%%%%%%%%
\section{Ejercicio de clase - Sección UNIVERSIDADES - Art.82 - Art.90}
\begin{enumerate}
    \item Art. 82 autonomía de la USAC:  
        \begin{itemize}
            \item Una institución autónoma, con personalidad jurídica.
            \item Se le delega la labor de dirigir, organizar y desarrollar la educación superior como única universidad estatal.
            \item Cooperar par resolver problemas nacionales.
        \end{itemize}
    
    \item Art. 83: gobierno de la USAC: 
        \begin{itemize}
            \item Se dirige por el rector, decanos, un representante del colegio profesional egresado de la USAC, catedrático titular.
        \end{itemize}
    
    \item Art. 84: Asignaciones presupuestarias para la USAC: 
        \begin{itemize}
            \item Asegura una cantidad no \textbf{menor} de 5\% del presupuesto nacional.
        \end{itemize}
    
    \item Art. 85: Universidades privadas: 
        \begin{itemize}
            \item Las universidades privadas tiene media vez se le autorice a operar, tiene personalidad jurídica y libertad para operar como universidad a la que se le delega la responsabilidad de la educación superior.  
        \end{itemize}
    
    \item Art. 86: Consejo de la enseñanza privada superior: 
        \begin{itemize}
            \item La USAC autoriza las universidades privadas.
            \item El consejo de la enseñanza privada superior tendrá que velar por la educación superior y el nivel académico correspondiente.
        \end{itemize}
    
    \item Art. 87: Reconocimiento de grados, títulos, diplomas e incorporaciones:  
        \begin{itemize}
            \item La USAC es la única que puede reconocer os títulos, grados, diplomas, etc.
        \end{itemize}
    
    \item Art. 88: Exenciones y deducciones de los impuestos: 
        \begin{itemize}
            \item Las universidades no pagan impuestos.
            \item Serán deducibles las donaciones que se den a una universidad.
            \item El estado puede subsidiar a las universidades.
        \end{itemize}
    
    \item Art. 89: Otorgamiento de grados, títulos y diplomas:
        \begin{itemize}
            \item Solamente las universidades autorizadas podrán otorgar títulos, diplomas, etc.
        \end{itemize}
    
    \item Art. 90: Colegiación profesional:
        \begin{itemize}
            \item La colegiación es obligatoria.
            \item Tiene fin de superación moral, científica, técnica y material.
            \item Son asociaciones gremiales con personalidad jurídica.
            \item La colegiación profesional se aprobarán con independencia de las universidades.
            \item Contribuirán al fortalecimiento de la autonomía, de la USAC.
            \item En todo asunto de interés nacional científico, técnico, etc. podrán requerir la participación de los colegios profesionales.
        \end{itemize}
\end{enumerate}

%%%%%%%%%%%%%%%%%%%%%%%%%%%%%%%%%%%%%%%%%%%%%%%%%%%%%%%%%%%%%%%%%%%%%%%%%%%%%%%%%%%%%%%%%%%%%%%%
\section{Constitución: Familia Capítulo II derechos sociales, sección primera, Familia}
Artículo 47 - 56
\begin{enumerate}
    \item El estado busca proteger a la familiar y su integración.
\end{enumerate}

%%%%%%%%%%%%%%%%%%%%%%%%%%%%%%%%%%%%%%%%%%%%%%%%%%%%%%%%%%%%%%%%%%%%%%%%%%%%%%%%%%%%%%%%%%%%%%%%
\section{Constitución: Cultura, Sección segunda, Art. 57 - 65}
\begin{enumerate}
    \item Busca promover las artes de la cultura.
    \item \emph{\textbf{Recordar lo siguiente: }GT es un país pluricultural tiene 23 etnias.}
    \item Eunesco, declaración de patrimonios de la humanidad.
\end{enumerate}

%%%%%%%%%%%%%%%%%%%%%%%%%%%%%%%%%%%%%%%%%%%%%%%%%%%%%%%%%%%%%%%%%%%%%%%%%%%%%%%%%%%%%%%%%%%%%%%%
\section{Constitución: Sección tercera, Comunidades indígenas Art. 66 - 70}
\begin{enumerate}
    \item Buscan garantizar a los indígenas ciertas cosas.
    \item Tengan igualdad ante la ley.
\end{enumerate}

%%%%%%%%%%%%%%%%%%%%%%%%%%%%%%%%%%%%%%%%%%%%%%%%%%%%%%%%%%%%%%%%%%%%%%%%%%%%%%%%%%%%%%%%%%%%%%%%
\section{Constitución: Educación, Sección cuarta, Art. 71 - 81}
\begin{enumerate}
    \item Vela por los asuntos de educación.
\end{enumerate}

%%%%%%%%%%%%%%%%%%%%%%%%%%%%%%%%%%%%%%%%%%%%%%%%%%%%%%%%%%%%%%%%%%%%%%%%%%%%%%%%%%%%%%%%%%%%%%%%
\section{Constitución: }


AUDIO 01:09

%%%%%%%%%%%%%%%%%%%%%%%%%%%%%%%%%%%%%%%%%%%%%%%%%%%%%%%%%%%%%%%%%%%%%%%%%%%%%%%%%%%%%%%%%%%%%%%%    
\section{Diferencias entre los trabajadores del estado y los trabajadores privados}
\begin{enumerate}
    \item 
\end{enumerate}
