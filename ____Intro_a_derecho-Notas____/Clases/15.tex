\section{Estado de derecho}
\begin{enumerate}
    \item El estado de derecho presupone una limitación de las funciones del estado por el derecho.
    \begin{itemize}
        \item \emph{\textbf{Recordar lo siguiente: }el estado tiene sus elementos y funciones el derecho busca limitar las funciones del estado.}
        \item \emph{\textbf{Interesante:} Los funcionarios públicos solo pueden hacer lo que la ley le permite, los ciudadanos podemos hacer todo lo que la ley no nos impide, los funcionarios solo pueden hacer lo que la ley acata.}
        \item \emph{\textbf{Interesante:} La lista de las libertades de los ciudadanos sería infinita pero las libertades de los funcionarios como el presidente o ministros }
        \item Diferencia entre libertades y libertad de Hayek: 
            \begin{itemize}
                \item Libertades $\Rightarrow$ gobernantes \emph{\textbf{Definición de ``libertades":} tienen una lista de libertades contempladas en la constitución.}
                \item Libertad $\Rightarrow$ gobernados \emph{\textbf{Definición de ``libertad":} a pesar que no es el plural de libertades, libertad comprende una más amplia gama de libertad que se extiende a una lista casi infinita de libertades. }
            \end{itemize}
        \item Hay delitos contra la administración públicas y hay delitos que los funcionarios tienen que los cuidadanos no tienen.
    \end{itemize}
    \item Implica:
        \begin{enumerate}
            \item Reconocimiento de derechos individuales: como el derecho a la vida, a la libre expresión.
            \item Limitación en el ejercicio del poder: sí limita la capacidad de los gobernantes. 
            \item Separación de poderes (funciones): ejecutivo legislativo y judicial.
            \item Principio de legalidad (sujeción a la ley): en general... que estamos sujetos a la ley.
            \item Exigencia de una constitución (Parte dogmática/Parte Orgánica): si existe una constitución, con sus dos partes orgánica y dogmática (algunas incluyen una tercera parte la parte pragmática).
            \item Sufragio universal: poder ejercer el voto. 
        \end{enumerate}
    
    \item Artículos constitucionales:
        \begin{itemize}
            \item Art. 141: separación de poderes y prohibición de subordinación de poderes. \emph{\textbf{Recordar lo siguiente: }Separación de poderes}.
            \item Art. 152: poder público \emph{\textbf{Recordar lo siguiente: }limitación.}
            \item Art. 153: el imperio de la ley, \emph{\textbf{Recordar lo siguiente: }principio de legalidad.}
            \item Art. 154: Los funcionarios pública \emph{\textbf{Recordar lo siguiente: } sujeción de la ley}.
            \item Art. 136: sufragio \emph{\textbf{Recordar lo siguiente: }sufragio universal}
            \item Art. 182: Limitación al presidente y al ejecutivo. \emph{\textbf{Recordar lo siguiente: }limitación de poderes.}
        \end{itemize}
    
    \item En un estado de derecho la actividad del estado está limitada por la constitución. Actividad limitada por la constitución política: 
        \begin{itemize}
            \item Normas claras, estables, iguales. \emph{\textbf{Recordar lo siguiente: }sujeción a la ley.}
            \item Reducción al mínimo posible de la coerción a los ciudadanos. \emph{\textbf{Interesante:} \emph{\textbf{Recordar lo siguiente: }Hayek y su propuesta que si alguien se debe someter a las reglas o someterse a la coaccionarlo, pero esta propuesta se debe reducir al mínimo.} \emph{\textbf{Caso} ``pena de muerte exonerada para la mujer": \textbf{Nos preguntamos:} ¿por qué?, debería de ser aplicable a todas las personas por igual dicen algunos.}}
            \item Respeto a la propiedad privada. \emph{\textbf{Recordar lo siguiente: } esto se persigue como el bien común. } \emph{\textbf{Ejemplo: }Alcalde de la antigua aprobó las obras en antigua que tenían un concepto anticolonial, el alcalde está en la cárcel}\emph{\textbf{Recordar lo siguiente: }está limitada por la expropiación según se pueda en conformidad con la ley, el patrimonio cultural, el subsuelo.} \emph{\textbf{Ejemplo: }el asunto de la zona 1 que está protegido y la propiedad privada está limitada por ley, cosas como pintar o construir es ilegal aun que uno sea el dueño.}
            \item Libertad producir, comerciar, consumir. \emph{\textbf{Recordar lo siguiente: } esto persigue el bien común.} \emph{\textbf{Interesante:} época de telgua, el estado tenía el monopolio de telefonía. El monopolio de la USAC. }
        \end{itemize}  
\end{enumerate}
%%%%%%%%%%%%%%%%%%%%%%%%%%%%%%%%%%%%%%%%%%%%%%%%%%%%%%%%%%%%%%%%%%%%%%%%%%%%%%%%%%%%%%%%%%%%%%%%
\section{Estado de legalidad}
Aquel que se fundamenta en la ley.
\begin{enumerate}
    \item La ley se utiliza como un vehículo para otorgar privilegios a unos en detrimento de otros.  
    \item No existen reglas claras, estables y universales, iguales para todos. 
    \item No garantiza ejercicio de derechos, ni libre actividad económica o libre competencia.
\end{enumerate}

%%%%%%%%%%%%%%%%%%%%%%%%%%%%%%%%%%%%%%%%%%%%%%%%%%%%%%%%%%%%%%%%%%%%%%%%%%%%%%%%%%%%%%%%%%%%%%%%
\section{\textbf{Nos preguntamos:} ¿GT es más un estado de derecho o legalista?}
\begin{enumerate}
    \item No es perfecto pero es considerado donde 
\end{enumerate}
