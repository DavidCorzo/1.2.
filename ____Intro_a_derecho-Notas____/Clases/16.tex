\section{Definición}
\begin{enumerate}
    \item \emph{\textbf{Definición de ``Derecho civil":} Conjunto de normas hechos, valores; que regula las relaciones fundamentales o más básicas de los individuo desde su nacimiento hasta posterior a su muerte.}
    \item \emph{\textbf{Ejemplo: }Los infantes pueden ser otorgados derechos retroactivos como que si tuvieran personalidad formal desde antes de nacer.}
    \item Derecho civil no es derecho mercantil:
        \begin{center}
           \begin{tabular}{ | p{5cm} | p{5cm} | }
               \hline
               Derecho civil & Derecho mercantil \\ 
               \hline
                    El derecho civil o derecho común regula las actividades aisladas, la diferencia está entre la repetición de dicho comportamiento & Cuando algo se vuelve habitual es derecho mercantil \\
                   El derecho civil es lento es tedioso no busca lucro & El derecho mercantil se ven transacciones en cantidad de transacciones en los miles de millones, como el ánimo es lucro son fuertemente incentivados a ser rápido y sin mayor trámite es mucho más ágil. \\ 
                   \emph{\textbf{Ejemplo: }en un testamento se incluye hasta la hora, en el derecho civil si yo quiero cambiar mi nombre lo puedo hacer, concluye todo con un testamento y esto está regulado por el derecho civil.} & \\ 
               \hline
           \end{tabular}
        \end{center}

        
    \item La división tradicional del derecho:
        \begin{itemize}
            \item Puede ser Francés o Alemán en GT se trata un poco de todos.
            \item La división está divida en 5.
        \end{itemize}
    
    \item La persona: 
        \begin{itemize}
            \item La identificación: nombre, nacionalidad, genero, etc.
            \item Persona jurídica: hereda la personalidad de lo que es una persona.
            \item Matrimonio, cosas como a quien le pertenecen los bienes en un matrimonio, a la hora del divorcio es útil saber. \emph{\textbf{Interesante:} el contrato hay que redactarlo pensando en incumplimiento.}
            \item Parentesco: quién es mi pariente y quién no, hay por sanguinida que es el pariente por sangre en GT se permite hasta el 4to grado de parentesco, afinidad es por que hubo matrimonio por ejemplo ``brother in law'', por adopción el único pariente es el adoptante. \emph{\textbf{Ejemplo: }Sandra Torres tiene parentesco por eso no puede.}
            \item Alimentos: no es solo dar de comer es amenidades como educación.
        \end{itemize}
    
    
    \item Derecho reales: conjunto de facultades que uno posee para poder disponer de algo:
        \begin{itemize}
            \item Implica la propiedad.
            \item Características de la propiedad privada.
                \begin{itemize}
                    \item Poseerlo
                    \item Poseer lo que se pueda derivar de él.
                    \item Derecho de poder venderlo o transmitirlo.
                \end{itemize}
            
            \item Saldo insoluto: 
                \begin{itemize}
                    \item El saldo que se pierde cuando no se puede paga un prestamo y se confiscan los bienes de garantía.
                    \item Esto está establecido en la ley.
                    \item La garantía puede ser mobiliaria, inmobiliaria o fiduciaria.
                \end{itemize}
            
            \item Venta en pública subasta: 
                \begin{itemize}
                    \item Yo no me puedo adueñar del bien de garantía un juez lo pondrá en pública subasta.
                \end{itemize}
            
            \item Segunda hipoteca: 
                \begin{itemize}
                    \item Dos hipotecas con vínculos al mismo bien.
                    \item Para ver si un bien ya esta hipotecado, se ve en el registro de la propiedad.
                \end{itemize}
            
            \item Sub-hipoteca: 
                \begin{itemize}
                    \item Transmite los derechos de hipotecas a otros bancos.
                    \item Es como un derecho accesorio.
                    \item La sub-hipoteca tiene que ser menor al plazo de la hipoteca. 
                \end{itemize}
            
            \item Remate: 
                \begin{itemize}
                    \item La venta del bien de garantía.
                \end{itemize}

        \end{itemize}

    
    \item Obligaciones:
        \begin{itemize}
            \item Pago por consignación: 
                \begin{itemize}
                    \item Cuando por alguna razón no aparece o no quiere pagar el préstamo, entonces le paga al tribunal
                \end{itemize}
            
            \item Pago por cesión:
                \begin{itemize}
                    \item Cuando se paga por medio de bienes por que no puedo pagar con dinero.
                    \item Es exclusivo a cuando no hay consignación.
                \end{itemize}
            
            \item Mora:
                \begin{itemize}
                    \item Penalización por daños y perjuicios al incumplir una obligación.
                \end{itemize}
            
            \item Clausula de indemnización: 
                \begin{itemize}
                    \item Se ponen de acuerdo con el daño y perjuicios desde el principio.
                \end{itemize}
            
            \item Transmisión de bienes:
                \begin{itemize}
                    \item Se transmiten obligaciones.
                \end{itemize}
        \end{itemize}
\end{enumerate}
