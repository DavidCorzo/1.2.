\documentclass{article}
\title{Derecho Civil}
\author{David Gabriel Corzo Mcmath}
\date{2019-Nov-06 22:54:44}
%%%%%%%%%%%%%%%%%%%%%%%%%%%%%%%%%%%%%%%%%%%%%%%%%%%%%%%%%%%%%%%%%%%%%%%%%%%%%%%%%%%%%%%%%%%%%%%%%%%%%%%%%%%%%%%%%%%%%%%%%%%%%%%%%%%%%%%%%%%%%%%
\usepackage[margin = 1in]{geometry}
\usepackage{graphicx}
\usepackage{fontenc}
\usepackage{pdfpages}
\usepackage[spanish]{babel}
\usepackage{amsmath}
\usepackage{amsthm}
\usepackage[utf8]{inputenc}
\usepackage{enumitem}
\usepackage{mathtools}
\usepackage{import}
\usepackage{xifthen}
\usepackage{pdfpages}
\usepackage{transparent}
\usepackage{color}
\usepackage{fancyhdr}
\usepackage{lipsum}
\usepackage{sectsty}
\usepackage{titlesec}
\usepackage{calc}
\usepackage{lmodern}
\usepackage{xpatch}
\usepackage{blindtext}
\usepackage{bookmark}
\usepackage{fancyhdr}
\usepackage{xcolor}
\usepackage{tikz}
\usepackage{blindtext}
\usepackage{hyperref}
\usepackage{listing}
\usepackage{spverbatim}
\usepackage{fancyvrb}
\usepackage{fvextra}
%%%%%%%%%%%%%%%%%%%%%%%%%%%%%%%%%%%%%%%%%%%%%%%%%%%%%%%%%%%%%%%%%%%%%%%%%%%%%%%%%%%%%%%%%%%%%%%%%%%%%%%%%%%%%%%%%%%%%%%%%%%%%%%%%%%%%%%%%%%%%%%
\begin{document}
\maketitle

\section{El derecho civil como disciplina}
\begin{enumerate}
    \item El derecho civil emana del derecho privado.
    \item Surge espontáneamente a partir de las interacciones humanas.
    \item Constituye esta rama el paradigma de la autonomía de la voluntad, terreno fértil donde se desarrolla el libre juego de las manifestaciones de las personas.
    \item La única limitación son el intercabio de objetos ilegales o materia prohibida.
\end{enumerate}

%%%%%%%%%%%%%%%%%%%%%%%%%%%%%%%%%%%%%%%%%%%%%%%%%%%%%%%%%%%%%%%%%%%%%%%%%%%%%%%%%%%%%%%%%%%%%%%%
\section{El derecho civil y el derecho mercantil}
\begin{enumerate}
    \item A partir de un repetido acto de las personas se debe elaborar un derecho que lo regule.
    \item Con forme se va ampliando el \emph{quehacer} humano se van formando nuevas formas de regirlo con el derecho.
    \item Importante: el derecho mercantil se es una rama especializada al comercio que \textbf{se deriva del derecho civil}.
\end{enumerate}

%%%%%%%%%%%%%%%%%%%%%%%%%%%%%%%%%%%%%%%%%%%%%%%%%%%%%%%%%%%%%%%%%%%%%%%%%%%%%%%%%%%%%%%%%%%%%%%%
\section{División tradicional del derecho mercantil}
\begin{itemize}
    \item Se divide en 6 libros:
        \begin{enumerate}
            \item Las personas y familia 
            \item Los bienes, propiedad y demás derechos reales 
            \item La sucesión hereditaria 
            \item El registro de la propiedad 
            \item Las obligaciones 
            \item Contratos en particular
        \end{enumerate}
    
    \item Las personas: 
        \begin{itemize}
            \item ``Ius civile'', el derecho propio del ciudadano.
            \item El punto de partida del derecho civil es \textbf{la persona}.
            \item Cosas como matrimonio, filiación, parentezco, nombre.
        \end{itemize}
    
    \item Propiedad: 
        \begin{itemize}
            \item Todas las manifestaciones de propiedad 
            \item Ejemplo: Bienes inmuebles, propiedad privada
        \end{itemize}
    
    \item Sucesión hereditaria:
        \begin{itemize}
            \item La adquisición de bienes posterior al fallecimiento de las personas.
        \end{itemize}
    
    \item Obligaciones: 
        \begin{itemize}
            \item Relación a la actuación de los individuos
            \item Desenvolvimiento de actividades humanas
        \end{itemize}
\end{itemize}

%%%%%%%%%%%%%%%%%%%%%%%%%%%%%%%%%%%%%%%%%%%%%%%%%%%%%%%%%%%%%%%%%%%%%%%%%%%%%%%%%%%%%%%%%%%%%%%%
\section{Los derechos reales}
\begin{itemize}
    \item Aquellos derechos que se relacionan con algo tangible el derecho es subsumido por ente material.
\end{itemize}

\subsection{Derechos reales de garantía}
\begin{itemize}
    \item Derechos reales vinculados con un bien que se utilizan como garantía de otra obligación.
    \item Se utiliza en \textbf{Hipotecas} para garantizar compensación usualmente si se incumple una deuda de préstamo.
    \item \textbf{Los saldos insolutos} son cuando una persona no puede pagar el préstamo y se le quita un bien de garantía, cuando se remata dicho bien de garantía usualmente no se recupera todo lo que se prestó en el préstamo por ende este saldo faltante es incobrable o insoluto.
    \item \textbf{Venta de inmueble hipotecado}: Cuando un inmueble se compromete con hipoteca debe inscribirse en el Registro de la Propiedad en donde habrá de aparecer la respectiva inscripción. Esto para que se pueda vender.
    \item \textbf{Segunda hipoteca:} cuando un bien inmueble puede estar como garantía en dos diferentes hipotecas, esto pasa usualmente con bienes de mucho valor. \emph{El primer prestamista tiene prioridad y si queda un saldo insoluto será de la segunda hipoteca.} \emph{\textbf{Ejemplo: }los bancos no aceptan segundas hipotecas.}
    \item \textbf{Sub-hipotéca}: cuando el bien es un crédito puede sub-hipotecarse. Se crea una figura en el medio, que por una parte es acreedor de alguien que le dio garantía hipotecaria pero al mismo tiempo es deudor de otra persona a quien le dio en garantía la hipoteca. Puede entonces proceder al remate del derecho de crédito hipotecario; en este caso ese tercero vendría a sustituir al original acreedor en las mismas condiciones en que pactó con el dueño del inmueble gravado.
    \item \textbf{Remate}: al incumplir con los términos de una hipoteca el prestamista puede adueñarse del bien asegurado como garantía y con permisos otorgados por el acreedor al inicio del contrato rematar ese bien en el mercado.
    \item \textbf{Vencimiento de la hipoteca}: si el prestamista no exige el pago de su hipoteca en un plazo mayor a 10 años puede cancelarse esa deuda y el prestamista quedarse sin su dinero. Esto es una defensa válida en caso de demanda por el prestamista.
\end{itemize}

%%%%%%%%%%%%%%%%%%%%%%%%%%%%%%%%%%%%%%%%%%%%%%%%%%%%%%%%%%%%%%%%%%%%%%%%%%%%%%%%%%%%%%%%%%%%%%%%
\section{Obligaciones en general}
\begin{itemize}
    \item Obligaciones == Compromisos   
\end{itemize}

\subsection{Cumplimiento de obligaciones}
\begin{itemize}
    \item \textbf{Pago}: no cumplir con lo que se comprometió en el contrato
    \item \textbf{Pago por consignación}: hay casos cuando el acreedor no quiere o no puede recibir el pago, en este caso el pago se hace a un tribunal como incidente de consignación. 
    \item \textbf{Pago por cesión de bienes}: la contra parte debe aceptar la opción de pagar por medio de bienes se llegase a dar el caso que no se pueda pagar con dinero. Hay dos formas de hacerlo la judicial o la extra judicial.
\end{itemize}

\subsection{Incumplimiento de obligaciones}
\begin{itemize}
    \item \textbf{Mora}: cuando el deudos está obligado a pagar al acreedor daños y perjuicios resultantes del retardo del pago de sus obligaciones. Las moras pueden estar establecidas en contrato o en ley. Se entra en mora un día después que venza el plazo.
    \item \textbf{Cláusula de indemnización}: Se puede fijar anticipadamente una cantidad que deberá pagar quien deje de cumplir la obligación. Esa indemnización compensa los daños y perjuicios. En algunos casos se conocen estas cláusulas como punitivas o penales.
\end{itemize}

\subsection{Transmisión de obligaciones}
\begin{itemize}
    \item \textbf{Cesión de derechos}: 
    \item \textbf{Subrogación}: 
    \item \textbf{Transmisión de deudas}: 
\end{itemize}

\subsection{Extinción de obligaciones}
\begin{itemize}
    \item \textbf{Novación}: Cuando se modifican las partes de la obligación original y se sustituye por otra, termina el compromiso inicial y este es sustituido por otro. Se extingue la primera, y las primeras obligaciones. La reducción del plazo no es novación. No es novación la reducción del tipo de interés.
    \item \textbf{Prescripción extintiva}: es una forma legal de:
        \begin{itemize}
            \item Adquirir derechos por el transcurso del tiempo
            \item Perder derechos por el transcurso del tiempo
        \end{itemize}
        usualmente es perder derechos o prescripción negativa, prescripción positiva para cuando se adquieren derechos. 
\end{itemize}

%%%%%%%%%%%%%%%%%%%%%%%%%%%%%%%%%%%%%%%%%%%%%%%%%%%%%%%%%%%%%%%%%%%%%%%%%%%%%%%%%%%%%%%%%%%%%%%%
\section{El contrato civil}
\begin{itemize}
    \item El contrato es el punto de partida de todas las instituciones que se comprenden dentro de esta disciplina jurídica. Si no hay contratos no hay operaciones ni negocios.
\end{itemize}

\subsection{Concepto de contrato}
\begin{itemize}
    \item Básicamente cuando entre dos personas se pacta el cumplimiento de una obligación.
    \item Esto todo es voluntario de las dos partes.
    \item La ley establece como se estructura un contrato para que tenga validez jurídica.
    \item Basta con que las personas manifieste su acuerdo en el contrato para que surga el contrato.
\end{itemize}

\subsection{Cumplimiento de buena fe}
\begin{itemize}
    \item El empresario debe cumplir con las obligaciones con buena fe.
    \item La persona que aprovechándose de la posición que ocupe, o la necesidad, inexperiencia o ignorancia de la otra parte induciéndola a conceder ventajas usurarias o a contraer obligaciones notoriamente perjudiciales a sus intereses, queda obligada a devolver lo recibido, con los daños y perjuicios, una vez declarada judicialmente la nulidad del convenio
    \item Derechos de trabajo supervisa esto.
\end{itemize}

\subsection{Forma de los contratos}
\begin{itemize}
    \item  Por escritura pública que consiste en un documento público que autoriza un Notario Público e impreso en papel especial, numerado y controlado, llamado Protocolo y que tiene en exclusiva y bajo su custodia cada Notario. \textbf{Deben de quedar manifestado el acuerdo de las dos partes.} El ``testimonio'' o copias legalizadas para registrar la operación son extendidas a los interesados.
    \item \textbf{Documento privado}, es un simple papel que pueden elaborar los contratantes, deben de incluir las firmas y estas ser revisadas por un notario.
    \item \textbf{Acta} levantada ante el alcalde del lugar.
    \item \textbf{Correspondencia} correo electrónico, \emph{\textbf{Interesante:} ya se aceptan las firmas electrónicas.} 
    \item \textbf{Verbalmente}: son contratos pactados verbalmente usualmente de poco valor, es difícil determinar qué pactaron. 
\end{itemize}

\subsection{De los contratos en particular}
\begin{itemize}
    \item \textbf{Compraventa}: compra y venta entre particulares.
    \item \textbf{Arrendamiento}: cuando se alquila un lugar para un negocio, recepción, fábrica, etc.
    \item \textbf{Opción y promesa de venta}: es una promesa unilateral, son unilateral es la opción por cuyo medio un propietario ofrece vender un inmueble bajo ciertas condiciones en un determinado plazo; en otras palabras el potencial comprador tiene la opción de hacerlo o no, hay promesas bilaterales las dos partes se comprometen: uno se obliga a vender y el otro a comprar, al punto de que si alguno incumple entonces se aplican las disposiciones del contrato o de ley en el sentido de obligar al vendedor a que se realice la venta opor el contrario, retener las arras que el comprador haya anticipado.
    \item \textbf{Mutuo}: comprende actividades de garantía como la personal o real, garantía fiduciaria, hipotécas (entre bancos y particulares) y prendas(muebles).
    \item \textbf{Obra}: contratos de construcción, el contratista contrata a una persona para encargarse de la construcción.
    \item \textbf{Mandato}: cuando una persona delega a otra la capacidad de hacer gestiones en su nombre.
\end{itemize}




\end{document}
