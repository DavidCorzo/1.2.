\documentclass[openany]{book}
\author{David Gabriel Corzo Mcmath}
\title{Matemática Discreta Aplicada \\ \normalsize Notas de clase}
\date{2019-08-03 02:29}
% % % % % % % % % % % % % % % % % % % % % % % % % % % % % % % % % % % % % % % % % % % % % % % % % % %
\usepackage[margin = 1in]{geometry}
\usepackage{graphicx}
\usepackage{fontenc}
\usepackage{pdfpages}
\usepackage[spanish]{babel}
\usepackage{amsmath}
\usepackage{amsthm}
\usepackage[utf8]{inputenc}
\usepackage{enumitem}
\usepackage{mathtools}
\usepackage{import}
\usepackage{xifthen}
\usepackage{pdfpages}
\usepackage{transparent}
\usepackage{color}
\usepackage{fancyhdr}
\usepackage{lipsum}
\usepackage{sectsty}
\usepackage{titlesec}
\usepackage{calc}
\usepackage{lmodern}
\usepackage{xpatch}
\usepackage{blindtext}
\usepackage{bookmark}

%%%%%%%%%%%%%%%%%%%%%%%%%%%%%%%%%%%%%%%%%%%%%%%%%%%%%%%%%%%%%%%%%%%%%%%%%%%%%%%%%%%%%%%%%%%%%%%%
\input{settings/page.tex}
\input{settings/chapter_head_font_whole_page.tex}
% \newcommand*\circled[1]{\tikz[baseline=(char.base)]{
%             \node[shape=circle,fill=gray!50,inner sep=2pt] (char) {#1};}}

% % header style
% \pagestyle{fancy}
% \fancyhf{}
% \fancyhead[EL]{\nouppercase\leftmark}
% \fancyhead[OR]{\nouppercase\rightmark}
% \fancyfoot[C]{\circled{\thepage}}
\newcommand*\circled[1]{\tikz[baseline=(char.base)]{
            \node[shape=circle,fill=gray!50,inner sep=2pt] (char) {#1};}}

% header style
\pagestyle{fancy}
\fancyhf{}
\fancyhead[EL]{\nouppercase\leftmark}
\fancyhead[OR]{\nouppercase\rightmark}
\fancyfoot[C]{\circled{\thepage}}

\fancypagestyle{plain}{%
  \fancyhf{}
  \fancyfoot[C]{\circled{\thepage}}
  \renewcommand{\headrulewidth}{0pt}
}



% % % % % % % % % % % % % % % % % % % % % % % % % % % % % % % % % % % % % % % % % % % % % % % % % % %
\begin{document}
\maketitle
\tableofcontents

%%%%%%%%%%%%%%%%%%%%%%%%%%%%%%%%%%%%%%%%%%%%%%%%%%%%%%%%%%%%%%%%%%%%%%%%%%%%%%%%%%%%%%%%%%%%%%%%

\part{Notas de Clases del semestre}
\chapter{Clase del Día: 2019-07-22; clase introductoria, ¿qué es la matemática discreta?, juegos de lógica fáciles}
\includepdf[pages=-,pagecommand={\thispagestyle{plain}}]{Clases/2019-07-22.pdf}

\chapter{Clase del Día: 2019-07-24; Lógica proposicional, juegos de lógica más complejos, ejemplos de juegos de lógica}
\includepdf[pages=-,pagecommand={\thispagestyle{plain}}]{Clases/2019-07-24.pdf}


\chapter{Clase del Día: 2019-07-29; Equivalencias lógicas, tautología, contradicción, contingencia, jerarquías operacionales lógicas}
\includepdf[pages=-,pagecommand={\thispagestyle{plain}}]{Clases/2019-07-29.pdf}


\chapter{Clase del Día: 2019-07-31; Inferencia, reglas de inferencia, }
\includepdf[pages=-,pagecommand={\thispagestyle{plain}}]{Clases/2019-07-31.pdf}
\includepdf[pages=-,pagecommand={\thispagestyle{plain}}]{pdf/F.pdf}


\chapter{Clase del Día: 2019-08-05; Primer laboratorio de lógica proposicional, equivalencias lógicas, reglas de inferencia}
\includepdf[pages=-,pagecommand={\thispagestyle{plain}}]{Clases/2019-08-05.pdf}


\chapter{Clase del Día: 2019-08-07; Técnicas de demostración, prueba directa, sets de números}
\includepdf[pages=-,pagecommand={\thispagestyle{plain}}]{Clases/2019-08-07.pdf}


\chapter{Clase del Día: 2019-08-12; Continuación de técnicas de demostración, prueba directa, prueba por contra-recíproca, prueba por casos(exhausión), prueba por contradicción}
\includepdf[pages=-,pagecommand={\thispagestyle{plain}}]{Clases/2019-08-12.pdf}


\chapter{Clase del Día: 2019-08-14; Más ejemplos de técnicas de demostración, introducción a la \textbf{inducción matemática}}
\includepdf[pages=-,pagecommand={\thispagestyle{plain}}]{Clases/2019-08-14.pdf}


\chapter{Clase del Día: 2019-08-19; Continuación inducción
matemática, más ejemplos con las técnicas de demostración presentadas hasta el momento, \textbf{técnicas de conteo}, principio de la suma, principio del producto}
\includepdf[pages=-,pagecommand={\thispagestyle{plain}}]{Clases/2019-08-19.pdf}


\chapter{Clase del Día: 2019-08-21; definición de P.S. \$ P.P., ejemplos, ¿cómo complementan estos principios a las técnicas de demostración?, \textbf{Permutaciones}}
\includepdf[pages=-,pagecommand={\thispagestyle{plain}}]{Clases/2019-08-21.pdf}

\chapter{Clase del Día: 2019-08-26; Combinatoria, ejercicios con Álvaro}
\includepdf[pages=-,pagecommand={\thispagestyle{plain}}]{Clases/2019-08-26.pdf}

\chapter{Clase del Día: 2019-08-28, Combinaciónes, ejemplos}
\includepdf[pages=-,pagecommand={\thispagestyle{plain}}]{Clases/2019-08-28.pdf}

\chapter{Clase del Día: 2019-09-02; Permutaciónes y combinatoria generalizada}
\includepdf[pages=-,pagecommand={\thispagestyle{plain}}]{Clases/2019-09-02.pdf}

\chapter{Clase del Día: 2019-09-16; Teoría de números, teorema de euclides, teorema fundamental de la aritmética, interesante: numeros primos, mcd(a,b)}
\includepdf[pages=-,pagecommand={\thispagestyle{plain}}]{Clases/2019-09-16.pdf}


\chapter{Clase del Día: 2019-09-18; Continuación, refutación del método mcd(a,b) de la mis, mínimo común multiplo (mcm(a,b)), }
\includepdf[pages=-,pagecommand={\thispagestyle{plain}}]{Clases/2019-09-18.pdf}

\chapter{Clase del Día: 2019-09-23; Identidad de Bézout}
\includepdf[pages=-,pagecommand={\thispagestyle{plain}}]{Clases/2019-09-23.pdf}

\chapter{Clase del Día: 2019-09-25; Ecuación Diofantiana}
\includepdf[pages=-,pagecommand={\thispagestyle{plain}}]{Clases/2019-09-25.pdf}

\chapter{Clase del Día: 2019-09-30; Aritmética modular}
\includepdf[pages=-,pagecommand={\thispagestyle{plain}}]{Clases/2019-09-30.pdf}

\chapter{Clase del Día: 2019-10-07 ; Continuación de aritmética modular}
\includepdf[pages=-,pagecommand={\thispagestyle{plain}}]{Clases/2019-10-07.pdf}

\chapter{Clase del Día: 2019-10-09 ; Cálculo de inversos multiplicativos módulo $n$}
\includepdf[pages=-,pagecommand={\thispagestyle{plain}}]{Clases/2019-10-09.pdf}

\chapter{Clase del Día: 2019-10-09 ; 2019-10-16 }
\includepdf[pages=-,pagecommand={\thispagestyle{plain}}]{Clases/2019-10-16.pdf}

\chapter{Clase del Día: 2019-10-09 ; Cálculo de inversos}
\includepdf[pages=-,pagecommand={\thispagestyle{plain}}]{Clases/CalculoDeInversos.pdf}

\chapter{Clase del día: 2019-10-30 ; Cifrado Vigenirer}
\includepdf[pages=-,pagecommand={\thispagestyle{plain}}]{Clases/2019-10-30.pdf}

\chapter{Clase del día: 2019-11-06 ; RSA, teoría}
\includepdf[pages=-,pagecommand={\thispagestyle{plain}}]{Clases/2019-11-06.pdf}

\chapter{Clase del día: 2019-11-11 ; RSA, ejemplo}
\includepdf[pages=-,pagecommand={\thispagestyle{plain}}]{Clases/2019-11-11.pdf}





\end{document}
