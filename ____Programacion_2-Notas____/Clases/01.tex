\section{Unity como objetos}
\begin{enumerate}
    \item \emph{\textbf{Definición de ``Inspector":} contiene los atributos públicos que son modificables por el usuario.}
    \item \emph{\textbf{Definición de ``Métodos públicos":} Son los atributos que se ven o son accesibles por el usuario.}
    \item \emph{\textbf{Definición de ``Métodos privados":} Son atributos que no son accesibles por el usuario.}
    \item Collider en unity: permite dar características físicas a un objeto, si no se tiene esto activado el objeto es como transparente.
    \item Por ejemplo un objeto perro:
        \begin{center}
        \begin{tabular}{ | p{5cm} | p{5cm} | }
         \hline
         Perro & gato \\
         \begin{itemize}
            \item $\Rightarrow$ Raza
            \item $\Rightarrow$ Color 
            \item $\Rightarrow$ Tamaño 
            \item $\Rightarrow$ Peso    
            \item $\Rightarrow$ Edad 
            \item $\Rightarrow$ \# Patas
            \item Call $\Rightarrow$ habar(); $\Rightarrow$ ``Woof''
        \end{itemize} & 
        \begin{itemize}
            \item $\Rightarrow$ Raza
            \item $\Rightarrow$ Color 
            \item $\Rightarrow$ Tamaño  
            \item Call $\Rightarrow$ hablar(); $\Rightarrow$ ``Miau''
        \end{itemize}
         \\ 
        \multicolumn{2}{|c|}{Como hay similaridades entre los objeetos cremos una clase ``animal''} \\ 
        \hline
        \multicolumn{2}{|c|}{Hagamos una definición general de animal y le enviamos atributos de cara uno a la clase perro y clase gato y hereda} \\
         \hline
        \end{tabular}
        \end{center}
        \begin{center}
        \begin{tabular}{ | p{5cm} | p{5cm} |  }
         \hline
         \multicolumn{2}{|c|}{Clase animal;} \\
         \hline
         Clase perro: & Clase gato: \\ 
         \hline
         $\Rightarrow$ Método Schitzu & $\Rightarrow$  Método egipcio \\
         \hline
         $\Rightarrow$ Método Salchicha & $\Rightarrow$ Método gato común \\  
         \hline
        \end{tabular}
        \end{center}
        
        
        \item En Unity se hereda a todos los objetos de la clase game object.
        \item Mover el sphere a main camera:
            \begin{itemize}
                \item Los valores de la clase padre los adoptan las clases hijas.
            \end{itemize}
        \item Para crear un padle:
            \begin{itemize}
                \item Crear un ``new element''
                \item insertar un cubo, modificar el parámetro del paddle para hacerlo más un rectángulo.
                \item Modificar el pad para que el usuario pueda observar el movimiento.
                \item Agregamos un comportamiento a la pelota, $\Rightarrow$ Crear un script de C\#, \emph{\textbf{(Paréntesis ``using, include, import'':}en programació se refiere a importar código de otro archivo\textbf{)}}
                \item Behavior usar polimorfismo, puedo incluir los métodos particulares para definiciones particulares.
                \item \emph{\textbf{Definición de ``dis":} cuando se usa un ``dis'' se hace referencia al mismo objeto que está en definición.}
            \end{itemize}
        
\end{enumerate}
