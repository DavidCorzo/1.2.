\documentclass{article}
\title{Notas Progración 2}
\author{David Gabriel Corzo Mcmath}
\date{2019-Oct-31 07:47:43}
%%%%%%%%%%%%%%%%%%%%%%%%%%%%%%%%%%%%%%%%%%%%%%%%%%%%%%%%%%%%%%%%%%%%%%%%%%%%%%%%%%%%%%%%%%%%%%%%%%%%%%%%%%%%%%%%%%%%%%%%%%%%%%%%%%%%%%%%%%%%%%%
\usepackage[margin = 1in]{geometry}
\usepackage{graphicx}
\usepackage{fontenc}
\usepackage{pdfpages}
\usepackage[spanish]{babel}
\usepackage{amsmath}
\usepackage{amsthm}
\usepackage[utf8]{inputenc}
\usepackage{enumitem}
\usepackage{mathtools}
\usepackage{import}
\usepackage{xifthen}
\usepackage{pdfpages}
\usepackage{transparent}
\usepackage{color}
\usepackage{fancyhdr}
\usepackage{lipsum}
\usepackage{sectsty}
\usepackage{titlesec}
\usepackage{calc}
\usepackage{lmodern}
\usepackage{xpatch}
\usepackage{blindtext}
\usepackage{bookmark}
\usepackage{fancyhdr}
\usepackage{xcolor}
\usepackage{tikz}
\usepackage{blindtext}
\usepackage{hyperref}
\usepackage{listing}
\usepackage{spverbatim}
\usepackage{fancyvrb}
\usepackage{fvextra}
%%%%%%%%%%%%%%%%%%%%%%%%%%%%%%%%%%%%%%%%%%%%%%%%%%%%%%%%%%%%%%%%%%%%%%%%%%%%%%%%%%%%%%%%%%%%%%%%%%%%%%%%%%%%%%%%%%%%%%%%%%%%%%%%%%%%%%%%%%%%%%%
\begin{document}
\maketitle

\begin{center}
    \begin{tabular}{ | p{11cm} | p{2cm} | p{2cm} | }
     \hline
     Story 000: C warmup & 0.35 & \\ 
     \hline
     Story 001: Lenguaje interpretado &  1.00 & \\ 
     \hline
    Story 002: Video sobre lenguajes interpretados y compilados & 1.00 & \\ 
    \hline
    Story 003: Implementar: Roll a Ball & 1.00 & \\ 
    \hline
    Story 004: Cuadro comparativo OOP - P. Estructurada & 1.00 & \\ 
    \hline
    Story 005: Doodle sobre arquitectura Java & 1.00 & \\ 
    \hline
    Examen Parcial I & 0.00 & \\ 
    \hline
    Story 006: Queue and Stack & 0.25 & \\ 
    \hline
    Story 007: Impacto de Java en la actualidad & 1.00 & \\ 
    \hline
    Story 008: Cuestionaro sobre caracteristicas de Java & 0.98 & \\ 
    \hline
    Story 009: Conocer los tipos de dato primitivos de Java & 1.00  & \\ 
    \hline
    Story 010: Entender las formas de utilizacion del sistema de I/O de Java & 1.00  &  \\ 
    \hline
    Story 011: Conocer las estructuras de control de flujo (if, while, for) & 0.90  & \\ 
    \hline
    Story 012: Ejercicios &  & \\ 
    \hline
    Story 013: Comprender la estructura de metodos en Java &   & \\ 
    \hline
    Story 014: Comprender la diferencia entre Clase y Objeto. &   & \\ 
    \hline
    Story 015: Constructores y Destructores &   & \\ 
    \hline
    Story 016: Comprender las opciones de encapsulamiento de metodos y atributos (Visibilidad) &  &  \\ 
    \hline
    Story 017: Aplicación de conceptos en Unity &   & \\ 
    \hline
    Story 018: Ejercicios &   & \\ 
    \hline
    Story 019: Comprender el concepto de sobrecarga &   & \\ 
    \hline
    Story 020: Comprender la diferencia entre asignar un objeto y clonarlo. &  &  \\ 
    \hline
    Story 021: Comprender la imlementacion de funciones recursivas en Java. &   & \\ 
    \hline
    Story 022: Comprender el uso de las variables static y comprender el concepto de final &   & \\ 
    \hline
    Story 023: ¿Como funciona el garbage collector? &   & \\ 
    \hline
    Story 024: Comprender los tipos de relaciones que existen entre las clases. &  &  \\ 
    \hline
    Story 025: Herencia &   & \\ 
    \hline
    Story 026: Arrays &   & \\ 
    \hline
    Story 027: Arrays of Objects &  &  \\ 
    \hline
    Story 028: Objects of Arrays &  & \\ 
    \hline 
    \end{tabular}
    \end{center}




\end{document}
