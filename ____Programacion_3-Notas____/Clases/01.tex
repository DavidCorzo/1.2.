\section{API's \& sus peculiaridades}
\begin{itemize}
\item \textbf{Nos preguntamos:} ¿Por que razón puedo ver el XML en el browser? Es por que estamos usando el método GET.
\item AJAX no permite no tiene la versitabilidad de parser. Este es el defecto de beautiful Soup, para el tipo de interacciónes con AJAX se necesita usar elenioum.
\end{itemize}

\section{Flask}
\begin{itemize}
    \item Flase es un framework, no es como Django que uno es obligado a usar cirtas cosas oblgatoriamente, \emph{\textbf{Definición de ``framework":} es un set de herramientas.}
    \item Usualmente utiliza menos memoria usar ``from <librería> import <funciones o clases a usar>''.
    \item Está corriendo en un puerto.
    \item \emph{\textbf{Definición de ``Socket":} la combinación de una IP  y un puerto}.
    \item Con hostname:
    \begin{verbatim}
        DAVIDCORZO@DESKTOP-73D7DE2 /cygdrive/c/Users/DAVIDCORZO
        $ hostname
            DESKTOP-73D7DE2
    \end{verbatim}
    
    \item app.run(host$=$"0.0.0.0",port=55) el host= 0.0.0.0. permite ver por medio de la red aplicaciones corriendo en otras computadoras.
    \item \emph{\textbf{Definición de ``puerto":} permite cambiar el socket, tiene un máximo de 65,535 puertos}.
    \item Debug igual True, uno de los beneficios que permite es correr la app sin tener que iterar el ciclo guardar,correr,ver\_resultados, pero el chiste es debugging.
    \item En python ``@'' es un decorador, es una manera implicia de llamar funciones.
    \item Cambiémos la ruta con el decorador a ``@app.route("/alumnos)''
    \item En flask hay dos formas de render: \emph{\textbf{Definición de ``Server side rendering":} fui a la base de datos y la respuesta solo al sabe mi aplicación, soy capaz por ende de imprimir información ingresada en mi aplicación.} \emph{\textbf{Definición de ``client side rendering":} este render lo hace el browser.}
    \item Lenguages de renderización, pintar de manera bonita en el browser.
    
    \item In jinja2 se referencia a una variable asi: ``\{\{ <variable> \}\}'', para hacer un for loop \{\% for i in algo \%\}.
\end{itemize}

\section{Averiguar después}
\begin{enumerate}
    \item decorators ``@''
    \item CSS Bootstrap
    \item jinja2 en Flask 
    \item Server side render, client side render.
    \item Leer Redis.
\end{enumerate}
