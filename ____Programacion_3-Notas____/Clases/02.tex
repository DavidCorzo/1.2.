\section{Usar el kanban}
\begin{itemize}
    \item Con esto se puede registrar en GitLab la actividad hasta el momento.
    \item Metodolgía ágil.
\end{itemize}

\section{Peculiaridades de Jinja}
\begin{enumerate} 
    \item \emph{\textbf{(Paréntesis ``historial'':}en bash o en shell se puede buscar el historial con el comando ``history''\textbf{)}}
    \item Permite no replicar código
    \item usar para un diccionario \{\%for students in ages.items()\%\}, posteriormente \{\%endfor\%\}, jinja intera en los elementos.
    \item Como jinja no tiene el control sobre los elementos se usa \{\{loop.index\}\}.
    \item Los parentesis con \% es para if's, fors por ejemplo.
    \item Puedo usar JS en mi página.
\end{enumerate}


\section{Gitlab}
\begin{itemize}
    \item ``git checkout master'', me saca del branch actual.
\end{itemize}

\section{Buenas prácticas de programación}
Agregar requirements.txt, if I add the flask library add the version.
\rule{16cm}{1pt}\newline 
\begin{itemize}
    \item Correr ``python -m pip install -r requirements.txt'' 
\end{itemize}
\rule{16cm}{1pt}\newline 

\section{Dockerfile \& .dockerignore}
\subsection{Dockerfile}
Correr: docker build -rn -f ``Dockerfile -t <nombre\_de\_archivo>:latest''\newline 
\rule{16cm}{1pt}
\textbf{EN EL Dockerfile}\newline 
\begin{verbatim}
    FROM python:3-alphine 

    WORKDIR / <app>
    COPY requirements.txt
    
    RUN pip intall -r requirements.txt 

    COPY . ./
    
    CMD ["Python","<app>.py>"]

    EXPOSE 5000 
\end{verbatim}
\rule{16cm}{1pt}\newline 


\subsection{.dockerignore}
No incluye los archivos especificados, \newline 
Ejemplo:\newline 
\rule{16cm}{1pt}\newline 
\textbf{EN EL .dockerignore}
\begin{verbatim}
    .vscode
    Dockerfile 
    README.md 
    *.pyc
\end{verbatim}
\rule{16cm}{1pt}\newline 


\section{.gitlab-ci.yml}
Sirve para compilar, para hacer pruebas dentro de GitLab, \emph{Citación:``no aprendan GitLabCI"}, \emph{\textbf{Definición de ``CI":} Continous Integration}.
\begin{itemize}
    \item \emph{\textbf{Definición de ``except master":} ejecutará el job en no master.}
    \item \emph{\textbf{Definición de ``only master":} ejecutará el job \textbf{solo} en master.}
    \item \emph{\textbf{Definición de ``CI":} permite probar e identificar fallas, es una metodología realmente para integración continual de mi código.}
\end{itemize}
\rule{16cm}{1pt}\newline 
\textbf{EN EL .gitlab-ci.yml}
\begin{verbatim}
    #Jenskinsfile
    #CircleCI
    #Travis CI 

    stages:
        -test 
        -build 
        -deploy
    
    <insertar un template de git lab>
\end{verbatim}
\rule{16cm}{1pt}\newline 

\section{Anuncios}
Invitación al innovation del entreprenour. \newline 
Martes 1 octubre conferencia de Docker. \newline 
Peach competition.


\section{Posterior a la clase investigar}
\begin{enumerate}
    \item html
    \item css
    \item bootstrap 
    \item navigation bar 
    \item jinja commands
    \item CI 
    \item Dockerfile 
\end{enumerate}
