\section{Diferencias entre UX/UI}
\subsection{UX}
\begin{enumerate}
    \item User Research
    \item User Personas 
    \item User flow diagrams
    \item Usability testing
    \item A/B testing 
    \item Measurements
\end{enumerate}
%%%%%%%%%%%%%%%%%%%%%%%%%%%%%%%%%%%%%%%%%%%%%%%%%%%%%%%%%%%%%%%%%%%%%%%%%%%%%%%%%%%%%%%%%%%%%%%%
\subsection{UI}
\begin{enumerate}
    \item Colors 
    \item Typography
    \item Color constrast 
    \item Accesibility 
    \item High fidelity prototypes 
    \item Overall look and feel
\end{enumerate}
\emph{\textbf{(Paréntesis:}Entrarían html y css, si tenés tiempo.\textbf{)}}
%%%%%%%%%%%%%%%%%%%%%%%%%%%%%%%%%%%%%%%%%%%%%%%%%%%%%%%%%%%%%%%%%%%%%%%%%%%%%%%%%%%%%%%%%%%%%%%%
\section{Proceso}
Hay varios métodos pero el más conocido y usado es el siguiente:
\begin{enumerate}
    \item Research: una de las partes más importantes para determinar qué podemos hacer, evaluar cómo opera la competencia, 
    \item Sketched: Los bocetos, cuando uno hace el research se le ocurren ideas, estos son machotes, son una lluvia de ideas que se utiliza para descartar ideas, normal mente son a mano.
    \item Wireframe: es una versión más refinada de los sketches, en los wireframes no se emite la versión terminada pero tenes que pensar más en lo realístico al producto final, vas a tomar en cuenta el dispositivo el cual tu app va a correrse.
    \item Mockups: Es la versión terminada de los mockups, se define casi que nada, esta es la version \underline{final} del producto.
    \item Prototyping: el prototipo es agarrar todos los mockups que tenemos y prototiparlo.
    \item Testing: esto es para medir qué tan buenos resultados están los resultados de los mockups, si no están al gusto del project manajer se repite la iteración.
\end{enumerate}
%%%%%%%%%%%%%%%%%%%%%%%%%%%%%%%%%%%%%%%%%%%%%%%%%%%%%%%%%%%%%%%%%%%%%%%%%%%%%%%%%%%%%%%%%%%%%%%%
\section{Aplicación a diseño en la vida real}
A continuación 
\begin{enumerate}
    \item Resource: qué tanto personal contamos con.
    \item Requerimientos: definen el ``qué hacer'', son los requisitos básicos que tienen que tener el producto, a veces uno propone algo mejor pero hay que respetar que tenemos esos requisitos.
    \item Deadline: uno empieza a economizar y ver si se puede saltar pasos en el proceso, probablemente saltar de sketches a mockups, por ejemplo.
    \item Availability: quiénes van a estar disponible, hay feriados en la iteración, alguien va a estar de vacaciones.
    \item Product type: qué estoy tratando de lograr desde lo que estoy diseñando, \emph{\textbf{Ejemplo:}un botón, una pagina de checkout}
\end{enumerate}
%%%%%%%%%%%%%%%%%%%%%%%%%%%%%%%%%%%%%%%%%%%%%%%%%%%%%%%%%%%%%%%%%%%%%%%%%%%%%%%%%%%%%%%%%%%%%%%%
\section{Buenos y malos ejemplos de UX}
\begin{itemize}
    \item Claridad en el producto, los parqueos y las señales de parqueo por ejemplo.
    \item \emph{\textbf{Ejemplo:} Formulario, la gran lista de países.}
    \item \emph{\textbf{Ejemplo:} Por ejemplo la eliminación de mensajes de whatsapp}
\end{itemize}
%%%%%%%%%%%%%%%%%%%%%%%%%%%%%%%%%%%%%%%%%%%%%%%%%%%%%%%%%%%%%%%%%%%%%%%%%%%%%%%%%%%%%%%%%%%%%%%%
\section{Trabajando con Project manajers y developers}
\textbf{Nos preguntamos:} ¿Cómo trabajo con tantas personas sin hacer un gran problema?
\begin{enumerate}
    \item Tener un timeline
    \item Definir prioridades
    \item Trabajar en equipo
    \item Mantener a todo el equipo al tanto en todo momento
\end{enumerate}
Teninendo un timeline y prioridades se puede evitar conflicto, \textbf{Nos preguntamos:} ¿qué pasa cuando dos productos prioritarios? \emph{\textbf{La respuesta a esta pregunta es: }tenemos que priorizar sólo uno de primero, normalmente se hace esto entre PM y usualmente solo se le informa al developer que deje de hacer lo que está haciendo y que se trabaje en lo que está prioridades}
%%%%%%%%%%%%%%%%%%%%%%%%%%%%%%%%%%%%%%%%%%%%%%%%%%%%%%%%%%%%%%%%%%%%%%%%%%%%%%%%%%%%%%%%%%%%%%%%
\section{Herramientas}
\begin{enumerate}
    \item Figma: es una herramienta multi-plataforma, es administrador de versiones, tiene versión web, tiene plugins que ayudan a diseñar más rápido.
    \item Sketch: sólo MAC, plug-ins, smart layouts, es pagado.
    \item Adobe XD, MAC/Windows, plugins, es gratis.
\end{enumerate}
%%%%%%%%%%%%%%%%%%%%%%%%%%%%%%%%%%%%%%%%%%%%%%%%%%%%%%%%%%%%%%%%%%%%%%%%%%%%%%%%%%%%%%%%%%%%%%%%
\section{\textbf{Nos preguntamos:} ¿Se puede programar y diseñar?}
\emph{\textbf{La respuesta a esta pregunta es: }Sí, es aún mejor tener los conocimientos para ser más ágiles a la hora de tener conflictos, tener en cuenta qué se puede hacer y qué no, entre más sepa de diseño mejor, mientras más sepa de programación mejor.}
%%%%%%%%%%%%%%%%%%%%%%%%%%%%%%%%%%%%%%%%%%%%%%%%%%%%%%%%%%%%%%%%%%%%%%%%%%%%%%%%%%%%%%%%%%%%%%%%
\section{Ventajas de diseñar y programar}
\begin{enumerate}
    \item Diseñar teniendo en cuenta la parte técnica.
    \item No vamos a diseñar cosas complejas ni para los usuarios ni para los devs.
    \item Diseño teniendo en cuenta la implementación.
    \item Hablar el lenguaje de los desarrolladores.
    \item Asegurarse que el diseño hecho quede igual una vez implementado.
    \item Ofrecer ayuda.
\end{enumerate}
%%%%%%%%%%%%%%%%%%%%%%%%%%%%%%%%%%%%%%%%%%%%%%%%%%%%%%%%%%%%%%%%%%%%%%%%%%%%%%%%%%%%%%%%%%%%%%%%
\section{\textbf{Nos preguntamos:} ¿Qué debemos hacer para programar?}
\begin{enumerate}
    \item HTML: estructura
    \item CSS: estilo
    \item JavaScript (un poquito solamente lo necesario), hay diseñadores que le tienen miedo a JavaScript.
    \item Responsive device
\end{enumerate}
%%%%%%%%%%%%%%%%%%%%%%%%%%%%%%%%%%%%%%%%%%%%%%%%%%%%%%%%%%%%%%%%%%%%%%%%%%%%%%%%%%%%%%%%%%%%%%%%
\section{Tres pilares de csss}
\begin{enumerate}
    \item Herencia: los hijos heredan los estilos de los padres, para ahorrar código.
    \item Especifidad: hay elementos en nuestro código son especificamente modificados para esos elementos selectos para que no hereden las características estipuladas si no que se comporten de una manera diferente.
    \item Cascada: dos clases que se llaman lo mismo y la que se va a aplicar va a ser la última. 
\end{enumerate}
Ver: EDteam.com 
%%%%%%%%%%%%%%%%%%%%%%%%%%%%%%%%%%%%%%%%%%%%%%%%%%%%%%%%%%%%%%%%%%%%%%%%%%%%%%%%%%%%%%%%%%%%%%%%
\section{\textbf{Nos preguntamos:} ¿Qué es responsive Web Design?}
\begin{enumerate}
    \item Es básicamente condicionales, tener en cuenta qué dispositivos estarán usando la aplicación.
    \item Una condicional que no afecte nada más que defina el comportamiento en diferentes dispositivos para acomodar bien todo según el dispositivo esto es para responsive, la condicional ``media query''.
\end{enumerate}
%%%%%%%%%%%%%%%%%%%%%%%%%%%%%%%%%%%%%%%%%%%%%%%%%%%%%%%%%%%%%%%%%%%%%%%%%%%%%%%%%%%%%%%%%%%%%%%%
\section{Resources}
\subsection{U-en-línea}
\begin{enumerate}
    \item Product design, udacity
    \item design.io
    \item learnux.io 
    \item platzi.com 
    \item ed.team 
    \item codigofacilito.com 
    \item udemy.com 
    \item udacity.com 
    \item youtube.com 
\end{enumerate}
\subsection{Inspiración}
\begin{enumerate}
    \item dribble.com 
    \item behance.net 
    \item uplabs.com 
    \item material.io 
    \item pttrns.com 
\end{enumerate}
