\section{ECMA Script - Katherine Garcia}
\begin{enumerate}
    \item JavaScript, el lenguaje de JavaScript se define bajo un estándar de ECMA, ECMAScript se creó para tener un estándar común entre los navegadores.
    \item Son un set de reglas, detalles y pautas que describen cómo es un lenguaje script.
    \item Las versiones: 
    \item Son transpilados los lenguajes ECMA, compila el código y lo transcribe a funciones que sí funcionan en ECMAScript, generalmente se transpila a una versión de ECMAScript en una versión.
    \item Se pueden mandar parametros por default, dentro de una función puede tomar un valor y fuera puede tomar otro.
    \item Es un set de standards.
    \item Transpilar, \emph{\textbf{Definición de ``transpilación":} es traducir código que no es complient con el ECMA standard.}
    \item ``By reading the ECMAScript specification, you learn how to create a scripting language. By reading the JavaScript documentation, you learn how to use a scripting language.''
    \item En python hay su propio ECMA compliance, es PEP, escribir en PEP es como vanila python.
    \item Hay muchos lenguajes tienen estos estándares, \emph{\textbf{Interesante:} ECMAScript versión 6, introdujo arrow functions, para los que vienen de un contexto como C\#}.
\end{enumerate}

%%%%%%%%%%%%%%%%%%%%%%%%%%%%%%%%%%%%%%%%%%%%%%%%%%%%%%%%%%%%%%%%%%%%%%%%%%%%%%%%%%%%%%%%%%%%%%%%
\section{Weakly typed / Strongly typed}
\begin{enumerate}
    \item Strong: ``siempre que que un objeto pase de una función de llamada una a función llamada, su tipo debe ser compatible con el tipo declarado e la función llamada''.
    \item \emph{\textbf{Ejemplo: }Java, Pascal, C*, and Lisp.}
    \item \emph{\textbf{Interesante:} En C si se utiliza punteros uno usa la memoria directamente de la computadora por ende se puede ver una característica de un weakly typed.}
    \item Weak: ``En lenguaje no cheque que los tipos de datos y solo asume que sí va a comportar como le dice el developer''.
    \item \emph{\textbf{Ejemplo: }JavaScript, Perl.} 
    \item Dynamic Static, un lenguaje strongly typed  tienden a ser Static, los weakly typed tienden a ser dynamic. No son synonimos, 
    
    \item Característica básicas de strongly typed:
        \begin{itemize}
            \item Type safety
            \item Memory safety 
            \item Static type checking
        \end{itemize}

    \item Aprender typescript asíncrono.
    \item \emph{\textbf{Interesante:} Codling, go, }
\end{enumerate}

\begin{itemize}
    \item Statically typed: no es restrictivo.
    \item Python es un hybrido.
    
    \item Dynamycally Strongly typed: restrictivo en qué tipos de datos se pueden operar.
    \item Usualmente los lenguajes compilados en algún nivel tienden a ser strongly typed.
\end{itemize}

%%%%%%%%%%%%%%%%%%%%%%%%%%%%%%%%%%%%%%%%%%%%%%%%%%%%%%%%%%%%%%%%%%%%%%%%%%%%%%%%%%%%%%%%%%%%%%%%
\section{Asynchronous - Call Stack}
\begin{enumerate}
    \item Asynchronous: 
        \begin{itemize}
            \item Varias operaciones están ocurriendo en threads separados.
            \item JavaScript, funciona en base a una naturaleza síncrona pero se puede volver asíncrono.
        \end{itemize}

    \item Call stack: 
        \begin{itemize}
            \item Es una parte del event loop, está diciendo que funciones o variables están metidos en el stack.
            \item ¿Asínrono y paralelo? asíncrono , \textbf{no blockea}, paralelo \textbf{todo se ejecuta al mismo tiempo.}
            \item Call stack $\neq$ quee.
            \item Un stack es apilar algo, se puede ver como un to-do list de funciones. 
            \item LIFO, Last in first out.
        \end{itemize}
\end{enumerate}

%%%%%%%%%%%%%%%%%%%%%%%%%%%%%%%%%%%%%%%%%%%%%%%%%%%%%%%%%%%%%%%%%%%%%%%%%%%%%%%%%%%%%%%%%%%%%%%%
\section{Event loop - Alejandra Lemus}
\begin{enumerate}
    \item El single thread en V8.
    \item Cuando no estamos usando el browser sino node.js podemos usar librerías de C.
    \item JavaScript si es un single thread, es un single threads, si tenemos ocho cores JavaScript solo va a utilizar uno, sin embargo se puede programar para que se haga una ejecución más inteligente.
    \item Event loop revisa si hay algo en el call stack para poder ejecutarlo, el event loop revisa el call-stack y si no hay nada ve el quee para meterlo al call stack.
    \item Verlo como un reloj, revisa cada ciclo de lo que está en el quue y en el stack.
    \item Promesas, eran para evitar hacer un montón de call-backs.
    \item Call-back hell, es un montón de call-backs.
    \item ``Programar en promesas''
    \item \emph{\textbf{Definición de ``promesas":} es una forma más bonita de hacer callbacks.}
    \item \emph{\textbf{Interesante:} cosas como un thread se queda esperando a otro thread es un asunto de ponerle atención a multiples variables, cosas como un thread blockea a otro aun que sea asíncrono, en Java se puede trabajar con single threads.}
    \item \emph{\textbf{Interesante:} GoldRoutine, uno puede mandar multiples threads pero requieren de un orquestador.}    
\end{enumerate}
%%%%%%%%%%%%%%%%%%%%%%%%%%%%%%%%%%%%%%%%%%%%%%%%%%%%%%%%%%%%%%%%%%%%%%%%%%%%%%%%%%%%%%%%%%%%%%%%

\section{Love-my-movies}
\begin{itemize}
    \item Usen JS en el proyecto.
    \item \emph{\textbf{Ojo: considerar lo siguiente...} usen JavaScript para las alertas y para las cosas del client side.}
\end{itemize}


\rule{16cm}{1pt}
